Una vez soldada la estructura y atornilladas las maderas, se procedio a hacer los agujeros en el caño de PVC en donde luego se colocarian los vasos con las lechugas.\\
Los caños se montaron sobre escuadras metálicas con sus respectivos topes para que no se produjera la rodadura de los mismos.\\
Para comenzar con el montaje vertical, se colocó cada rodamiento en el soporte correspondiente, luego se montaron las varillas lisas y por ultimo la varilla trapezoidal acoplada al motor.\\
Una vez armada la estructura vertical, sin el brazo robótico, se procede a montar el carro sobre el caño y luego el sistema de correa y poleas ya acopladas a sus respectivos motores paso a paso.\\
Se procede con el ajuste de la correa y prueba de montaje.\\
Una vez montados correctamente ambos sistemas de traslación se monta el brazo robótico para hacer pruebas integrales.\\