El sistema cosechador automático de lechugas hidropónicas cumplió los objetivos planteados, validando la viabilidad técnica de automatización en agricultura vertical mediante integración de mecatrónica, visión artificial e inteligencia artificial.

\subsection*{Logros técnicos}

\underline{Control jerárquico:} Arquitectura con separación funcional entre nivel regulatorio (control de actuadores en tiempo real) y supervisor (gestión de secuencias y coordinación con IA). Demostró ser robusta, escalable y modular.

\underline{Visión e IA:} Métricas de desempeño: detección >90\%, clasificación de madurez 97\%, repetibilidad 88\%. Algoritmos basados en procesamiento morfológico y análisis estadístico de descriptores geométricos.

\underline{Sistema mecánico:} Estructura cartesiana XZ + brazo serie 2 GDL. Precisión de posicionamiento 1\% en lazo abierto, tasa de agarres exitosos 82\% con realimentación visual.

\subsection*{Escalabilidad}

Software y algoritmos de IA son transferibles a escala industrial sin modificaciones estructurales. La arquitectura modular permite escalar mediante ajustes paramétricos (conversiones pasos/mm, límites, perfiles de velocidad).

\subsection*{Limitaciones identificadas}

\underline{Rigidez estructural:} Material PETG del prototipo presenta flexibilidad excesiva generando vibraciones. Requiere transición a materiales de mayor módulo elástico.

\underline{Iluminación:} Sistema funciona entre 800-1200 lux pero es sensible a variaciones. Requiere iluminación estabilizada o algoritmos adaptativos.

\subsection*{Viabilidad económica}

Costo de desarrollo: \$6.200.000 ARS. Período de recupero proyectado: 4 años (capacidad 55-60 lechugas/mes a \$2.500/unidad). Se requiere análisis detallado incluyendo costos operativos y escenarios de escalamiento.