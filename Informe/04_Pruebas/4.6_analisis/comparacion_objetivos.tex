\subsection{Cumplimiento de objetivos y transferibilidad}

La validación experimental del prototipo confirmó el cumplimiento de los objetivos planteados y permitió establecer la transferibilidad de cada subsistema hacia el sistema a escala real.

\subsubsection{Sistema de visión artificial}

El sistema de visión cumplió su objetivo de detección y clasificación, alcanzando tasas superiores al 90\% en detección de elementos, 85\% en clasificación de madurez, y 82\% en agarres exitosos con integración de control. Los algoritmos desarrollados son directamente transferibles al sistema real sin modificaciones estructurales, requiriendo únicamente recalibración de parámetros de cámara y ajuste de relaciones píxel-mundo según la nueva geometría.

La limitación crítica identificada es la sensibilidad a condiciones de iluminación, que reduce tasas de detección ante variaciones significativas de intensidad o direccionalidad de luz.

\subsubsection{Sistema de control coordinado}

La arquitectura de control jerárquico alcanzó plenamente su objetivo, demostrando sincronización efectiva entre ejes, precisión del 1\% en posicionamiento a lazo abierto sin errores acumulativos, y respuesta inmediata ante señales de seguridad. El software de control (nivel regulatorio y supervisor) es transferible al sistema real, requiriendo únicamente actualización de parámetros: conversión pasos/mm, límites de carrera, perfiles de velocidad según masas reales, y timeouts ajustados a distancias mayores.

\subsubsection{Arquitectura modular}

El diseño modular validó exitosamente la separación funcional entre subsistemas mediante interfaces bien definidas, facilitando modificaciones de componentes sin afectar otros módulos y permitiendo ajustes paramétricos sin reestructuración de software.

\subsection{Limitaciones identificadas y consideraciones de diseño}

\subsubsection{Aspectos mecánicos}

\textbf{Robot serie:} El material del brazo (PETG) presenta flexibilidad excesiva para velocidades elevadas, generando vibraciones residuales que afectan precisión final, tiempo de estabilización y velocidad segura de operación.

\textbf{Oscilaciones estructurales:} Se observaron oscilaciones menores durante la activación del brazo, atribuidas al juego en perfiles utilizados y rigidez lateral insuficiente de la solución implementada. El análisis de elementos finitos indica que el principio estructural es válido para el sistema real, requiriendo únicamente escalamiento dimensional apropiado. La limitación se proyecta eliminable mediante perfiles de mayor rigidez lateral (fig. \ref{fig:barra_lateral}).

\textbf{Fricción horizontal:} El sistema de desplazamiento horizontal (fig. \ref{fig:perfil_porton}, \ref{fig:carro_porton}) presentó fricción superior a la esperada, manifestándose en fuerza de arranque elevada y variaciones según posición del carro.

\subsubsection{Validación estructural}

A pesar de las limitaciones mencionadas, la estructura general del prototipo demostró rigidez suficiente para extrapolación al sistema real. Los soportes fabricados en PLA presentaron resistencia adecuada, el sistema de tres varillas proporcionó estabilidad sin rotaciones indeseadas, los dos motores sincronizados fueron efectivos, y no se observaron deformaciones permanentes ni fatiga de materiales.

El concepto estructural validado sirve como base sólida, requiriendo escalamiento proporcional y especificación de materiales según cargas incrementadas. Las limitaciones identificadas (rigidez del brazo, fricción horizontal, oscilaciones estructurales) son inherentes al prototipo de validación y han sido caracterizadas para guiar el diseño a escala real. La arquitectura de control y algoritmos de visión constituyen activos transferibles, validando el enfoque modular y escalable propuesto.
