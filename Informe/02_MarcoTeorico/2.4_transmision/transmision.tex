% Transmisión

\subsection{Sistemas de transmisión}

Se cuenta con diversos sistemas de transmisión, en este proyecto se emplea dos tipos principales de transmisión para convertir el movimiento rotacional de los motores en movimiento lineal y rotacional según los requerimientos de cada eje:

\subsubsection{Transmisión por correa dentada y poleas}

La transmisión por correa dentada consiste en una correa flexible con dientes en su superficie interna que engranan con poleas dentadas solidarias a los ejes motor y conducido. Este sistema permite transmitir potencia y movimiento entre ejes paralelos con alta eficiencia y sin deslizamiento.\\

Ventajas del sistema
\begin{itemize}[label=$\bullet$]
    \item \underline{Transmisión sincrónica:} Los dientes de la correa engranan positivamente con las poleas, eliminando el deslizamiento y garantizando una relación de transmisión exacta
    \item \underline{Alta eficiencia:} Eficiencia típica del 96-98\%, superior a sistemas de correas trapezoidales
    \item \underline{Bajo mantenimiento:} No requiere lubricación y presenta mínimo desgaste en condiciones normales de operación
    \item \underline{Operación silenciosa:} El engrane de los dientes minimiza vibraciones y ruido
    \item \underline{Flexibilidad de diseño:} Permite implementar diferentes relaciones de transmisión mediante la selección de poleas de distintos diámetros
    \item \underline{Bajo momento de inercia:} La correa es liviana, lo que facilita arranques y paradas rápidas\\
\end{itemize}

Componentes del sistema:
\begin{enumerate}
    \item \underline{Polea motriz:} Montada en el eje del motor, transmite el movimiento rotacional a la correa. Su diámetro primitivo determina la velocidad de salida junto con el diámetro de la polea conducida.

    \item \underline{Polea conducida:} Montada en el eje a accionar, recibe el movimiento desde la correa. La relación de diámetros entre ambas poleas define la relación de transmisión:
    \begin{equation}
        i = \frac{D_2}{D_1} = \frac{n_1}{n_2} = \frac{z_2}{z_1}
    \end{equation}
    donde $D$ son los diámetros primitivos, $n$ las velocidades angulares y $z$ el número de dientes de las poleas motriz (1) y conducida (2).

    \item \underline{Correa dentada:} Elemento flexible con perfil dentado estandarizado. Los perfiles más comunes son:
    \begin{itemize}[label=$\bullet$]
        \item GT2 (2\,mm de paso): Utilizado en aplicaciones de precisión como impresoras 3D y CNC
        \item GT3 (3\,mm de paso): Mayor capacidad de carga que GT2
        \item HTD (High Torque Drive): Para aplicaciones de alta potencia
        \item T-series y AT-series: Perfiles trapezoidales para uso general\\
    \end{itemize}
\end{enumerate}

Cálculo de la relación de transmisión\\
Para un sistema con polea motriz de $z_1$ dientes y polea conducida de $z_2$ dientes:
\begin{equation}
    n_2 = n_1 \cdot \frac{z_1}{z_2}
\end{equation}

La longitud de correa requerida para una distancia entre centros $C$ se calcula como:
\begin{equation}
    L = 2C + \frac{\pi(D_1 + D_2)}{2} + \frac{(D_2 - D_1)^2}{4C}
\end{equation}

Tensado de la correa
El tensado adecuado es crítico para el funcionamiento correcto:
\begin{itemize}[label=$\bullet$]
    \item \underline{Tensión insuficiente:} Produce saltos de dientes y pérdida de sincronismo
    \item \underline{Tensión excesiva:} Incrementa el desgaste de la correa y aumenta la carga sobre los rodamientos
\end{itemize}

La tensión estática recomendada se calcula como:
\begin{equation}
    T_s = \frac{1.5 \cdot P}{v} + m \cdot v^2
\end{equation}
donde $P$ es la potencia transmitida, $v$ la velocidad lineal de la correa y $m$ la masa por unidad de longitud.

\subsubsection{Transmisión por varilla roscada}

La transmisión por varilla roscada, también conocida como sistema tornillo-tuerca, convierte el movimiento rotacional en movimiento lineal mediante el principio de la rosca helicoidal. Cuando la varilla roscada gira, la tuerca acoplada a ella se desplaza linealmente a lo largo del eje de la varilla.\\

Tipos de roscas
\begin{enumerate}
    \item \underline{Rosca trapezoidal (ACME o TR):}
    \begin{itemize}[label=$\bullet$]
        \item Perfil trapezoidal con ángulo de flanco de 30° (ACME) o 15° (TR métrica)
        \item Alta resistencia al desgaste
        \item Buena capacidad de carga axial
        \item Eficiencia típica: 30-50\%
        \item Auto-frenante en la mayoría de configuraciones (ángulo de hélice bajo)
        \item Aplicación: Movimiento vertical del sistema
    \end{itemize}

    \item \underline{Husillo de bolas:}
    \begin{itemize}[label=$\bullet$]
        \item Recirculación de bolas entre la rosca y la tuerca
        \item Mínima fricción (contacto rodante vs. deslizante)
        \item Eficiencia típica: 90-95\%
        \item No es auto-frenante
        \item Mayor precisión y vida útil
        \item Costo significativamente mayor\\
    \end{itemize}
\end{enumerate}

Relación de transmisión.\\
\noindent
El desplazamiento lineal por revolución está determinado por el paso de la rosca:
\begin{equation}
    \Delta x = n \cdot p
\end{equation}
donde $n$ es el número de revoluciones y $p$ es el paso de la rosca.

Para rosca de múltiples entradas:
\begin{equation}
    p_{\text{efectivo}} = N_e \cdot p_{\text{nominal}}
\end{equation}
donde $N_e$ es el número de entradas.

El torque necesario para mover una carga $F$ mediante una varilla roscada incluye dos componentes:

1. Torque de subida:
\begin{equation}
T_{\text{subida}} = \frac{F \cdot d_m}{2} \cdot k_R + \frac{F \cdot \mu_c \cdot d_c}{2}
\label{eq:torque_subida}
\end{equation}

2. Torque de bajada:
\begin{equation}
T_{\text{bajada}} = \frac{F \cdot d_m}{2} \cdot k_L + \frac{F \cdot \mu_c \cdot d_c}{2}
\label{eq:torque_bajada}
\end{equation}

El torque necesario resulta:
\begin{equation}
    T_{\text{total}} = max(T_{\text{subida}}, T_{\text{bajada}})
    \label{eq:torque_total_varilla}
\end{equation}

donde:
\begin{itemize}[label=$\bullet$]
    \item F: Carga a desplazar.
    \item $d_m$: diámetro medio de la rosca
    \item $d_c$: diámetro medio del collarín
    \item $\mu_c$:coeficiente de fricción del collarín
    \item $k_R$:  factor de rosca para subida, incluye efecto de cuña y fricción
    \item $k_L$: factor de rosca para bajada
\end{itemize}

La eficiencia de la transmisión se define como:
\begin{equation}
    \eta = \frac{\tan(\alpha)}{\tan(\alpha + \phi)}
\end{equation}

Para roscas trapezoidales típicas con $\mu \approx 0.1-0.2$ (lubricadas), la eficiencia varía entre 30\% y 50\%.\\

Condición de auto-frenado.\\
Un sistema de tornillo-tuerca es auto-frenante (la carga no puede hacer girar la varilla) cuando:
\begin{equation}
    \alpha < \phi \quad \Rightarrow \quad \tan(\alpha) < \mu
\end{equation}

Esto es deseable en aplicaciones verticales donde se requiere mantener la posición sin energía en el motor.\\

Carga crítica y pandeo. \\
Para varillas largas trabajando a compresión, debe verificarse la carga crítica de Euler:
\begin{equation}
    P_{\text{cr}} = \frac{\pi^2 \cdot E \cdot I}{(K \cdot L)^2}
\end{equation}

donde $E$ es el módulo de Young, $I$ el momento de inercia de la sección, $L$ la longitud y $K$ el factor de longitud efectiva según las condiciones de apoyo.\\

Vida útil y desgaste.\\
\noindent
El desgaste en roscas trapezoidales depende de:
\begin{itemize}[label=$\bullet$]
    \item Presión de contacto entre rosca y tuerca
    \item Velocidad de deslizamiento
    \item Calidad de la lubricación
    \item Material de la tuerca (bronce, polímeros, acero)
\end{itemize}

El sistema trabaja principalmente a tracción (varilla fija superior, carga colgante) lo que minimiza problemas de pandeo, aunque se verifica la estabilidad estructural para garantizar seguridad ante cualquier eventualidad.
