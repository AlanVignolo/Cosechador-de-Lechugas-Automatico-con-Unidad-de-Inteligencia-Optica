En sistemas de control jerárquico, es común utilizar distintas plataformas de hardware según las capacidades de procesamiento requeridas en cada nivel. Los microcontroladores se especializan en tareas de control en tiempo real y manejo directo de periféricos, mientras que los microprocesadores ofrecen mayor capacidad de cómputo para algoritmos complejos y gestión de alto nivel.

La selección del microcontrolador o computador de placa única (SBC) adecuado depende de los requisitos específicos del sistema de control. A continuación se presenta una comparación de las plataformas más utilizadas en proyectos de automatización y robótica:

\begin{table}[H]
\centering
\small
\begin{tabular}{|p{2.5cm}|p{2cm}|p{2cm}|p{2.5cm}|p{2cm}|p{2cm}|}
\hline
\textbf{Plataforma} & \textbf{Procesador} & \textbf{Frecuencia} & \textbf{Memoria} & \textbf{GPIO/PWM} & \textbf{Voltaje} \\
\hline
Arduino Mega 2560 & ATmega2560 8-bit & 16 MHz & Flash: 256 KB\newline SRAM: 8 KB & 54 digitales\newline 15 PWM\newline 16 analógicos & 5V \\
\hline
Arduino Uno R3 & ATmega328P 8-bit & 16 MHz & Flash: 32 KB\newline SRAM: 2 KB & 14 digitales\newline 6 PWM\newline 6 analógicos & 5V \\
\hline
ESP32 & Xtensa LX6 32-bit Dual-Core & 240 MHz & Flash: hasta 16 MB\newline SRAM: 520 KB & 34 GPIO\newline 16 PWM\newline 18 analógicos & 3.3V \\
\hline
Raspberry Pi 5 & ARM Cortex-A76 64-bit Quad-Core & 2.4 GHz & RAM: 4/8 GB LPDDR4X & 40 pines GPIO\newline Hardware PWM & 5V (GPIO 3.3V) \\
\hline
Raspberry Pi 4 & ARM Cortex-A72 64-bit Quad-Core & 1.5 GHz & RAM: 2/4/8 GB LPDDR4 & 40 pines GPIO\newline Hardware PWM & 5V (GPIO 3.3V) \\
\hline
STM32F4 & ARM Cortex-M4 32-bit & 168 MHz & Flash: 512 KB - 1 MB\newline SRAM: 192 KB & Hasta 140 GPIO\newline Timer PWM & 3.3V \\
\hline
\end{tabular}
\caption{\textit{Comparación de características principales de microcontroladores}}
\label{tab:comparacion_micros}
\end{table}

\begin{table}[H]
\centering
\small
\begin{tabular}{|p{2.5cm}|p{3cm}|p{3cm}|p{4cm}|}
\hline
\textbf{Plataforma} & \textbf{Comunicación} & \textbf{Conectividad} & \textbf{Aplicaciones típicas} \\
\hline
Arduino Mega 2560 & 4x UART, I²C, SPI & USB & Control tiempo real, actuadores, sensores \\
\hline
Arduino Uno R3 & 1x UART, I²C, SPI & USB & Proyectos básicos, prototipado \\
\hline
ESP32 & 3x UART, I²C, SPI, CAN & Wi-Fi, Bluetooth & IoT, control inalámbrico, sensores remotos \\
\hline
Raspberry Pi 5 & UART, I²C, SPI & Ethernet, Wi-Fi, Bluetooth, USB 3.0 & IA, visión, supervisión, interfaz gráfica \\
\hline
Raspberry Pi 4 & UART, I²C, SPI & Ethernet, Wi-Fi, Bluetooth, USB 3.0 & Procesamiento de imágenes, servidor, HMI \\
\hline
STM32F4 & Hasta 6x UART, I²C, SPI, CAN & USB, Ethernet (según modelo) & Control industrial, sistemas embebidos críticos \\
\hline
\end{tabular}
\caption{\textit{Comunicación y aplicaciones de microcontroladores}}
\label{tab:aplicaciones_micros}
\end{table}

Los microcontroladores soportan diversos protocolos de comunicación, cada uno con características específicas que los hacen adecuados para diferentes aplicaciones:

\begin{table}[H]
\centering
\small
\begin{tabular}{|p{2cm}|p{2.5cm}|p{2cm}|p{2cm}|p{2.5cm}|p{2cm}|}
\hline
\textbf{Protocolo} & \textbf{Tipo} & \textbf{Líneas} & \textbf{Velocidad} & \textbf{Distancia} & \textbf{Dispositivos} \\
\hline
UART (Serial) & Asíncrono punto a punto & 2 (TX, RX) & 9600 - 115200 bps (típico)\newline hasta 1 Mbps & Corta (<15 m) & 2 (1 a 1) \\
\hline
I²C & Síncrono multi-maestro & 2 (SDA, SCL) & Standard: 100 kbps\newline Fast: 400 kbps\newline High-speed: 3.4 Mbps & Muy corta (<1 m) & Hasta 128 (direccionamiento 7-bit) \\
\hline
SPI & Síncrono maestro-esclavo & 4 (MOSI, MISO, SCK, CS) + CS por esclavo & Hasta 10-50 Mbps & Muy corta (<3 m) & 1 maestro, múltiples esclavos \\
\hline
CAN Bus & Diferencial multi-maestro & 2 (CAN-H, CAN-L) & 125 kbps - 1 Mbps & Larga (hasta 40 m a 1 Mbps) & Hasta 110 nodos \\
\hline
RS-485 & Diferencial multi-punto & 2 o 4 (A, B) & Hasta 10 Mbps & Muy larga (hasta 1200 m) & Hasta 32 (sin repetidores) \\
\hline
USB & Serial diferencial & 4 (D+, D-, VCC, GND) & USB 2.0: 480 Mbps\newline USB 3.0: 5 Gbps & Media (5 m sin hub) & 1 host, hasta 127 dispositivos \\
\hline
\end{tabular}
\caption{\textit{Comparación de protocolos de comunicación}}
\label{tab:protocolos_comunicacion}
\end{table}

%\subsubsection{Comunicación entre Raspberry Pi y Arduino Mega}

%La comunicación entre una Raspberry Pi y un Arduino Mega mediante UART permite implementar una arquitectura de control jerárquico donde:

%\begin{itemize}
  %  \item La \textbf{Raspberry Pi} actúa como nivel supervisor, ejecutando algoritmos de alto nivel, procesamiento de imágenes, toma de decisiones y coordinación global del sistema
 %   \item El \textbf{Arduino Mega} funciona como nivel regulatorio, gestionando el control en tiempo real de actuadores, lectura de sensores y ejecución de comandos de bajo nivel
%\end{itemize}

%\textbf{Configuración de la conexión:}

%\begin{enumerate}
    %\item \textbf{Conexión física:}
    %\begin{itemize}
       % \item TX de Raspberry Pi $\rightarrow$ RX del Arduino (GPIO14/TX)
      %  \item RX de Raspberry Pi $\leftarrow$ TX del Arduino (GPIO15/RX)
     %   \item GND común entre ambos dispositivos
    %\end{itemize}

    %\item \textbf{Consideración de niveles lógicos:}
    %\begin{itemize}
    %    \item Raspberry Pi opera a 3.3V mientras Arduino Mega opera a 5V
    %    \item Alternativamente, los pines RX del Arduino son tolerantes a 3.3V en la mayoría de los casos
    %\end{itemize}

    %\item \textbf{Configuración de parámetros:}
    %\begin{itemize}
   %     \item Ambos dispositivos deben configurarse con la misma velocidad de transmisión (típicamente 115200 baudios)
  %      \item Configuración estándar: 8 bits de datos, sin paridad, 1 bit de parada (8N1)
 %   \end{itemize}
%\end{enumerate}

%\textbf{Protocolo de comunicación:}

%Para una comunicación robusta, es recomendable implementar un protocolo de aplicación sobre UART que incluya:

%\begin{itemize}
    %\item \textbf{Delimitadores de mensaje:} Caracteres especiales que indican inicio y fin de mensaje.
    %\item \textbf{Estructura de comandos:} Formato definido para envío de instrucciones.
   % \item \textbf{Checksum o CRC:} Para verificación de integridad de datos.
  %  \item \textbf{Confirmaciones (ACK/NACK):} Para asegurar recepción correcta de mensajes.
 %   \item \textbf{Manejo de errores:} Reintentos y timeout en caso de fallas de comunicación.
%\end{itemize}

%Esta arquitectura de comunicación permite que la Raspberry Pi se enfoque en tareas complejas de alto nivel mientras que el Arduino garantiza el control de bajo nivel, comunicándose de forma eficiente mediante un canal UART confiable.
