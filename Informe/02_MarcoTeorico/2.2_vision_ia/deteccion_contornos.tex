\subsubsection{Extracción de contornos}

El algoritmo implementado en OpenCV mediante la función \texttt{findContours} detecta contornos recorriendo la imagen binaria y rastreando las fronteras. El modo de recuperación RETR\_EXTERNAL extrae únicamente contornos externos de nivel superior, suficiente para aplicaciones donde interesan objetos individuales sin necesidad de analizar estructura interna.

\subsubsection{Descriptores geométricos}

Los descriptores cuantifican propiedades estructurales de contornos. El área encerrada se calcula mediante:

\begin{equation}
A = \frac{1}{2}\left|\sum_{i=0}^{n-1}(x_i y_{i+1} - x_{i+1}y_i)\right|
\end{equation}

donde $(x_n, y_n) = (x_0, y_0)$ cierra el contorno. El rectángulo delimitador es el rectángulo alineado a ejes que contiene completamente al contorno: $\text{BBox} = (x_{min}, y_{min}, w, h)$. El centroide representa el centro de masa geométrico:

\begin{equation}
\bar{x} = \frac{1}{6A}\sum_{i=0}^{n-1}(x_i + x_{i+1})(x_i y_{i+1} - x_{i+1}y_i), \quad \bar{y} = \frac{1}{6A}\sum_{i=0}^{n-1}(y_i + y_{i+1})(x_i y_{i+1} - x_{i+1}y_i)
\end{equation}

\subsubsection{Filtrado y selección de contornos}

El filtrado elimina contornos no deseados según criterios del problema. El filtrado por área establece límites: $A_{min} \leq A_{contorno} \leq A_{max}$. El filtrado por posición descarta contornos que tocan bordes de la imagen o que no contienen el punto central. Cuando múltiples contornos cumplen filtros básicos, se implementa un sistema de scoring que pondera múltiples criterios:

\begin{equation}
S_{total} = w_1 \cdot S_{área} + w_2 \cdot S_{posición} + w_3 \cdot S_{forma}
\end{equation}

donde los pesos $w_i$ (con $\sum w_i = 1$) reflejan importancia relativa de cada aspecto.
