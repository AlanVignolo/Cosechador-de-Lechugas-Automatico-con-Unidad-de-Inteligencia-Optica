\subsection{Detección y Análisis de Contornos}

La detección de contornos identifica las fronteras entre regiones con propiedades visuales distintas en imágenes binarias, proporcionando información estructural esencial para reconocimiento y clasificación de elementos. Un contorno se define matemáticamente como la secuencia ordenada de píxeles que forman el límite de una región conexa: $C = \{(x_i, y_i)\}_{i=1}^{n}$ donde $(x_i, y_i) \in \partial R$, siendo $\partial R$ la frontera topológica de la región $R$. Formalmente, un punto pertenece a la frontera si es parte de la región pero tiene al menos un vecino que no pertenece a ella. La conectividad del contorno se define por el vecindario considerado: 4-conectividad (vecinos arriba, abajo, izquierda, derecha) o 8-conectividad (incluyendo diagonales, generando contornos más suaves).

\subsubsection{Algoritmo de Suzuki-Abe}

El algoritmo de Suzuki-Abe, implementado en OpenCV mediante la función \texttt{findContours}, detecta simultáneamente contornos externos e internos organizándolos jerárquicamente. El algoritmo recorre la imagen binaria en orden raster y, al encontrar un píxel de objeto no visitado, inicia el rastreo del contorno siguiendo la frontera. La jerarquía resultante representa relaciones de anidamiento entre contornos mediante estructura de árbol.

Los modos de recuperación determinan qué contornos se extraen: \textbf{RETR\_EXTERNAL} extrae solo contornos externos de nivel superior (suficiente para aplicaciones donde interesan objetos individuales), \textbf{RETR\_LIST} extrae todos los contornos sin jerarquía, y \textbf{RETR\_TREE} construye la jerarquía completa de contornos anidados. El parámetro de aproximación controla la representación: \textbf{CHAIN\_APPROX\_NONE} almacena todos los puntos preservando información geométrica completa, mientras que \textbf{CHAIN\_APPROX\_SIMPLE} comprime segmentos colineales reduciendo uso de memoria.

\subsubsection{Descriptores Geométricos}

Los descriptores cuantifican propiedades estructurales de contornos. El \textbf{área} encerrada se calcula mediante la fórmula del Shoelace:

\begin{equation}
A = \frac{1}{2}\left|\sum_{i=0}^{n-1}(x_i y_{i+1} - x_{i+1}y_i)\right|
\end{equation}

donde $(x_n, y_n) = (x_0, y_0)$ cierra el contorno. Este método tiene complejidad $\mathcal{O}(n)$ y proporciona área exacta del polígono. El \textbf{perímetro} se calcula sumando distancias euclidianas entre puntos consecutivos: $P = \sum_{i=0}^{n-1}\sqrt{(x_{i+1}-x_i)^2 + (y_{i+1}-y_i)^2}$.

El \textbf{rectángulo delimitador} (bounding box) es el rectángulo alineado a ejes que contiene completamente al contorno: $\text{BBox} = (x_{min}, y_{min}, w, h)$ donde $x_{min} = \min_i x_i$, $y_{min} = \min_i y_i$, $w = \max_i x_i - x_{min}$ y $h = \max_i y_i - y_{min}$. Este descriptor permite filtrar contornos por relación de aspecto o dimensiones absolutas.

El \textbf{centroide} representa el centro de masa geométrico del contorno:

\begin{equation}
\bar{x} = \frac{1}{6A}\sum_{i=0}^{n-1}(x_i + x_{i+1})(x_i y_{i+1} - x_{i+1}y_i), \quad \bar{y} = \frac{1}{6A}\sum_{i=0}^{n-1}(y_i + y_{i+1})(x_i y_{i+1} - x_{i+1}y_i)
\end{equation}

Este punto resulta fundamental para cálculos de desviación espacial en sistemas de corrección de posicionamiento visual.

\subsubsection{Filtrado y Selección de Contornos}

Las imágenes binarias reales contienen múltiples contornos, muchos correspondientes a ruido o elementos irrelevantes. El filtrado elimina contornos no deseados según criterios del problema. El \textbf{filtrado por área} establece límites: $A_{min} \leq A_{contorno} \leq A_{max}$. Los umbrales se definen según el tamaño esperado de objetos; por ejemplo, $A_{min} = 0.01 \cdot A_{imagen}$ filtra ruido puntual mientras que $A_{max} = 0.6 \cdot A_{imagen}$ elimina regiones que abarcan casi toda la imagen.

El \textbf{filtrado por posición} descarta contornos que tocan bordes de la imagen (generalmente incompletos) o que no contienen el punto central (bajo asunción de objeto centrado en campo de visión). El \textbf{filtrado por forma} emplea relación de aspecto $AR = w/h$ del bounding box para discriminar orientaciones: valores $AR > 2$ indican objetos horizontales, $AR < 0.5$ objetos verticales, y valores intermedios formas aproximadamente cuadradas.

Cuando múltiples contornos cumplen filtros básicos, se implementa un \textbf{sistema de scoring} que pondera múltiples criterios para seleccionar el mejor candidato. Un score típico combina área normalizada, distancia del centroide al centro de imagen, y métricas de calidad de forma:

\begin{equation}
S_{total} = w_1 \cdot S_{área} + w_2 \cdot S_{posición} + w_3 \cdot S_{forma}
\end{equation}

donde los pesos $w_i$ reflejan importancia relativa de cada aspecto según el problema. Para detección de cintas de referencia, la posición central recibe mayor peso ($w_2 = 0.4$) dado que las desviaciones esperadas son pequeñas. Para clasificación de cultivos, el área domina ($w_1 = 0.5$) siendo el descriptor más discriminativo.

\subsubsection{Análisis de Regiones Parciales}

En aplicaciones donde la geometría varía entre zonas del objeto, analizar regiones específicas del contorno proporciona descriptores más robustos que el contorno completo. Por ejemplo, para detectar cintas de referencia con bases rectangulares pero formas irregulares superiores, se analiza únicamente el 10\% inferior del contorno: $R_{base} = \{(x,y) \in C : y \geq y_{max} - 0.1h\}$ donde $h$ es la altura del bounding box. Sobre esta región se calcula el ancho efectivo $w_{base} = \max_{(x,y) \in R_{base}} x - \min_{(x,y) \in R_{base}} x + 1$ y la ocupación (fracción de píxeles presentes respecto al rectángulo delimitador de la base). Bases sólidas rectangulares presentan ocupación superior a 0.8, mientras que formas irregulares tienen menor ocupación, permitiendo discriminación efectiva.

Este enfoque de análisis por regiones resulta particularmente valioso cuando las condiciones de captura introducen variabilidad en ciertas zonas del objeto (como hojas de lechuga que se extienden irregularmente) pero otras zonas mantienen geometría consistente (como bases de recipientes).
