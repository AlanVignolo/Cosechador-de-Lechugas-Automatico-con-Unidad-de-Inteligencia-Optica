\subsection{Detección y Análisis de Contornos}

La detección de contornos constituye una técnica fundamental en visión por computadora para la identificación y caracterización de objetos en imágenes. Los contornos representan las fronteras entre regiones con propiedades visuales distintas, proporcionando información estructural esencial para el reconocimiento y clasificación de elementos.

\subsubsection{Definición Matemática de Contorno}

Un contorno se define como la secuencia ordenada de píxeles que forman el límite de una región conexa en una imagen binaria:

\begin{equation}
C = \{(x_i, y_i)\}_{i=1}^{n} \quad \text{donde } (x_i, y_i) \in \partial R
\end{equation}

siendo $\partial R$ la frontera topológica de la región $R$. Formalmente, un punto $(x,y)$ pertenece a $\partial R$ si:

\begin{equation}
(x,y) \in R \quad \land \quad \exists (x',y') \in \mathcal{N}(x,y) : (x',y') \notin R
\end{equation}

donde $\mathcal{N}(x,y)$ representa el vecindario del píxel (usualmente 4-conectividad u 8-conectividad).

\subsubsection{Algoritmos de Detección de Contornos}

\textbf{Método de Seguimiento de Bordes (Border Following)}

El algoritmo de Moore-Neighbor tracing recorre el perímetro de objetos binarios siguiendo el borde en sentido horario o antihorario:

\begin{enumerate}
\item Localizar primer píxel de objeto (barrido izquierda-derecha, arriba-abajo)
\item Examinar vecinos en orden circular desde el píxel de entrada
\item Seleccionar primer vecino que pertenezca al objeto
\item Repetir hasta retornar al píxel inicial
\end{enumerate}

La complejidad computacional es $\mathcal{O}(n)$ donde $n$ es el número de píxeles del perímetro.

\textbf{Algoritmo de Suzuki-Abe}

Este método detecta simultáneamente contornos externos e internos (huecos), organizándolos jerárquicamente. Define dos tipos de bordes:

\begin{itemize}
\item \textbf{Border tipo 0}: Borde externo de objeto rodeado por fondo
\item \textbf{Border tipo 1}: Borde interno de hueco rodeado por objeto
\end{itemize}

La jerarquía se representa mediante un árbol donde cada nodo almacena:
\begin{equation}
\text{Nodo}_i = \{\text{Contorno}_i, \text{Padre}_i, \text{Hijos}_i\}
\end{equation}

\begin{figure}[h]
\centering
\includegraphics[width=0.7\textwidth]{imagenes/jerarquia_contornos.png}
\caption{Jerarquía de contornos mostrando relación padre-hijo entre contornos externos e internos}
\label{fig:jerarquia_contornos}
\end{figure}

\subsubsection{Aproximación y Simplificación de Contornos}

Los contornos detectados suelen contener información redundante. La aproximación poligonal reduce el número de puntos preservando la forma esencial.

\textbf{Algoritmo de Douglas-Peucker}

Este método recursivo aproxima una curva mediante segmentos de línea recta:

\begin{enumerate}
\item Conectar primer y último punto del contorno
\item Calcular distancia perpendicular de todos los puntos intermedios a esta línea
\item Si $d_{max} > \epsilon$, dividir contorno en el punto de máxima distancia
\item Aplicar recursivamente a cada segmento
\item Si $d_{max} \leq \epsilon$, aproximar segmento con línea recta
\end{enumerate}

La tolerancia $\epsilon$ controla el nivel de simplificación:

\begin{equation}
\epsilon_{optimal} = \alpha \cdot P_{contorno}
\end{equation}

donde $P_{contorno}$ es el perímetro y $\alpha \in [0.01, 0.05]$ es un factor empírico.

\begin{figure}[h]
\centering
\includegraphics[width=0.8\textwidth]{imagenes/aproximacion_contorno.png}
\caption{Aproximación poligonal de contorno mediante algoritmo Douglas-Peucker}
\label{fig:aproximacion_contorno}
\end{figure}

\subsubsection{Descriptores Geométricos de Contornos}

Los descriptores cuantifican propiedades estructurales de los contornos, permitiendo clasificación y reconocimiento de formas.

\textbf{Área}

El área encerrada por un contorno se calcula mediante la fórmula del Shoelace:

\begin{equation}
A = \frac{1}{2}\left|\sum_{i=0}^{n-1}(x_i y_{i+1} - x_{i+1}y_i)\right|
\end{equation}

donde $(x_n, y_n) = (x_0, y_0)$ para cerrar el contorno.

\textbf{Perímetro}

La longitud del contorno se aproxima sumando distancias euclidianas entre puntos consecutivos:

\begin{equation}
P = \sum_{i=0}^{n-1}\sqrt{(x_{i+1}-x_i)^2 + (y_{i+1}-y_i)^2}
\end{equation}

\textbf{Centroide}

El centro de masa del contorno se calcula como:

\begin{equation}
\bar{x} = \frac{1}{6A}\sum_{i=0}^{n-1}(x_i + x_{i+1})(x_i y_{i+1} - x_{i+1}y_i)
\end{equation}

\begin{equation}
\bar{y} = \frac{1}{6A}\sum_{i=0}^{n-1}(y_i + y_{i+1})(x_i y_{i+1} - x_{i+1}y_i)
\end{equation}

\textbf{Momentos de Imagen}

Los momentos proporcionan información sobre la distribución espacial del contorno:

\begin{equation}
M_{pq} = \sum_{i=0}^{n-1} x_i^p y_i^q
\end{equation}

Los momentos centrales, invariantes a traslación, se definen como:

\begin{equation}
\mu_{pq} = \sum_{i=0}^{n-1} (x_i - \bar{x})^p (y_i - \bar{y})^q
\end{equation}

\textbf{Razón de Aspecto}

La relación entre dimensiones principales se calcula mediante el rectángulo delimitador mínimo:

\begin{equation}
AR = \frac{W_{bbox}}{H_{bbox}}
\end{equation}

donde $W_{bbox}$ y $H_{bbox}$ son ancho y alto del bounding box.

\begin{figure}[h]
\centering
\includegraphics[width=0.6\textwidth]{imagenes/bounding_box.png}
\caption{Rectángulo delimitador y ejes principales de un contorno}
\label{fig:bounding_box}
\end{figure}

\textbf{Circularidad}

Mide cuán similar es un contorno a un círculo:

\begin{equation}
C = \frac{4\pi A}{P^2}
\end{equation}

Propiedades:
\begin{itemize}
\item $C = 1$ para círculos perfectos
\item $C < 1$ para formas no circulares
\item $C \to 0$ para formas muy irregulares
\end{itemize}

\textbf{Convexidad}

La relación entre el área del contorno y su envolvente convexa:

\begin{equation}
\text{Convexity} = \frac{A_{convexHull}}{A_{contour}}
\end{equation}

Valores cercanos a 1 indican formas convexas; valores mayores indican concavidades.

\textbf{Solidez}

Mide la proporción del contorno que ocupa su envolvente convexa:

\begin{equation}
\text{Solidity} = \frac{A_{contour}}{A_{convexHull}}
\end{equation}

\subsubsection{Filtrado de Contornos}

En aplicaciones reales, las imágenes contienen múltiples contornos, muchos correspondientes a ruido. El filtrado elimina contornos irrelevantes según criterios definidos.

\textbf{Filtrado por Área}

Eliminar contornos cuya área esté fuera de un rango esperado:

\begin{equation}
\text{Contorno válido} \Leftrightarrow A_{min} \leq A_{contour} \leq A_{max}
\end{equation}

Los umbrales se establecen según el tamaño esperado de objetos de interés.

\textbf{Filtrado por Jerarquía}

Seleccionar solo contornos de nivel específico en la jerarquía:
\begin{itemize}
\item Nivel 0: Contornos externos principales
\item Nivel 1: Huecos dentro de objetos
\item Niveles superiores: Estructuras anidadas
\end{itemize}

\textbf{Filtrado por Forma}

Utilizar descriptores geométricos como criterio:

\begin{equation}
\text{Validez} = \begin{cases}
\text{True} & \text{si } C > C_{min} \land AR \in [AR_{min}, AR_{max}] \\
\text{False} & \text{en caso contrario}
\end{cases}
\end{equation}

\begin{figure}[h]
\centering
\includegraphics[width=0.8\textwidth]{imagenes/filtrado_contornos.png}
\caption{Proceso de filtrado de contornos por área y forma}
\label{fig:filtrado_contornos}
\end{figure}

\subsubsection{Análisis de Regiones Internas}

Una vez identificados contornos relevantes, el análisis del contenido interior proporciona información adicional para clasificación.

\textbf{Extracción de Región de Interés (ROI)}

Dado un contorno, extraer la región rectangular que lo contiene:

\begin{equation}
\text{ROI} = I[y_{min}:y_{max}, x_{min}:x_{max}]
\end{equation}

donde $(x_{min}, y_{min})$ y $(x_{max}, y_{max})$ son las coordenadas extremas del contorno.

\textbf{Máscara de Contorno}

Crear una máscara binaria donde píxeles internos al contorno valen 1:

\begin{equation}
M(x,y) = \begin{cases}
1 & \text{si } (x,y) \in \text{Interior}(C) \\
0 & \text{en caso contrario}
\end{cases}
\end{equation}

Esta máscara permite análisis estadístico de píxeles contenidos.

\textbf{Conteo de Píxeles}

El número de píxeles dentro del contorno proporciona una medida de área alternativa:

\begin{equation}
N_{pixels} = \sum_{(x,y) \in \text{ROI}} M(x,y)
\end{equation}

Esta métrica es menos sensible a irregularidades del perímetro que el cálculo mediante coordenadas del contorno.

\subsubsection{Invariancia a Transformaciones}

Para robustez ante variaciones de escala, rotación y traslación, se emplean descriptores invariantes.

\textbf{Momentos de Hu}

Los siete momentos invariantes de Hu se construyen a partir de momentos centrales normalizados:

\begin{equation}
\eta_{pq} = \frac{\mu_{pq}}{\mu_{00}^{1 + (p+q)/2}}
\end{equation}

El primer momento de Hu, por ejemplo:

\begin{equation}
h_1 = \eta_{20} + \eta_{02}
\end{equation}

Estos momentos permanecen constantes bajo rotación, escala y traslación.

\textbf{Descriptores de Fourier}

La transformada de Fourier del contorno proporciona representación en frecuencia invariante a transformaciones:

\begin{equation}
F_n = \sum_{k=0}^{N-1} z_k e^{-i2\pi nk/N}
\end{equation}

donde $z_k = x_k + iy_k$ es la representación compleja del contorno. Los coeficientes de baja frecuencia capturan la forma global, mientras los de alta frecuencia representan detalles.
