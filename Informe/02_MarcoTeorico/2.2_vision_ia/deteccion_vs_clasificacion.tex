\subsection{Detección vs Clasificación de Objetos}

\subsubsection{Enfoques Principales}

\begin{table}[h]
\centering
\begin{tabular}{|l|p{4cm}|p{4cm}|}
\hline
\textbf{Método} & \textbf{Output} & \textbf{Aplicación} \\ \hline
Clasificación & Etiqueta de clase & Identificar QUÉ hay \\ \hline
Detección & Bounding boxes + clase & Identificar QUÉ y DÓNDE \\ \hline
Segmentación & Máscara pixel-level & Forma exacta del objeto \\ \hline
\end{tabular}
\end{table}

\subsubsection{Justificación del Enfoque Híbrido}

En este proyecto se implementó un \textbf{enfoque híbrido}:

\begin{equation}
\text{Pipeline} = \text{Análisis Morfológico} \rightarrow \text{Clasificación CNN}
\end{equation}

\textbf{Ventajas vs YOLO/Faster R-CNN:}
\begin{itemize}
    \item \textbf{Menor carga computacional}: No requiere red de propuestas de regiones
    \item \textbf{Mejor adaptación}: Análisis morfológico explota geometría conocida de plántulas
    \item \textbf{Interpretabilidad}: Se pueden validar detecciones geométricas
\end{itemize}

\textbf{Trade-offs:}
\begin{itemize}
    \item Requiere calibración de parámetros morfológicos
    \item Menos robusto a objetos con formas variables
    \item Mayor dependencia de condiciones de iluminación controladas
\end{itemize}

\begin{equation}
\text{Tiempo}_{\text{híbrido}} \approx 0.3 \cdot \text{Tiempo}_{\text{YOLO}} \quad \text{(en Raspberry Pi 4)}
\end{equation}