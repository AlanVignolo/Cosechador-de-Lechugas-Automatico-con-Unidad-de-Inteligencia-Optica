\subsection{Clasificación Basada en Análisis Morfológico}

La clasificación morfológica de objetos utiliza características geométricas y estadísticas extraídas de contornos e imágenes para asignar categorías a elementos detectados. A diferencia de métodos de aprendizaje profundo que requieren extensos conjuntos de entrenamiento y elevada capacidad computacional, este enfoque se fundamenta en reglas explícitas derivadas del conocimiento del dominio y análisis estadístico de muestras representativas. El método resulta particularmente efectivo cuando las clases presentan diferenciación clara en características medibles como tamaño, forma o color, y cuando se dispone de entornos controlados donde las condiciones de captura son predecibles.

\subsubsection{Clasificación por Umbrales sobre Descriptores}

El método más directo establece límites de decisión basados en valores de descriptores geométricos. Para un descriptor $d$ (como área, perímetro, o relación de aspecto) y un conjunto de clases $\mathcal{C} = \{c_1, c_2, ..., c_n\}$, la regla de clasificación se define mediante umbrales $t_0, t_1, ..., t_n$ que particionan el espacio de características:

\begin{equation}
\text{Clase}(x) = c_i \quad \text{si} \quad t_{i-1} < d(x) \leq t_i
\end{equation}

donde $x$ es el objeto a clasificar y $d(x)$ su valor de descriptor.

Para objetos que se diferencian principalmente por tamaño, el área en píxeles constituye un descriptor efectivo. La robustez del método depende de la \textbf{separabilidad entre clases}: la distancia entre centroides de clases adyacentes debe ser significativa respecto a la dispersión intra-clase para minimizar errores de clasificación. Matemáticamente, se busca que:

\begin{equation}
|\mu_i - \mu_{i+1}| \gg \sigma_i + \sigma_{i+1}
\end{equation}

donde $\mu_i$ y $\sigma_i$ son la media y desviación estándar del descriptor para la clase $i$.

La selección óptima de umbrales requiere análisis estadístico de una base de datos representativa. Para cada clase $c_i$, se calculan la media $\mu_i$ y desviación estándar $\sigma_i$ del descriptor mediante:

\begin{equation}
\mu_i = \frac{1}{N_i}\sum_{j=1}^{N_i} d_j, \quad \sigma_i = \sqrt{\frac{1}{N_i-1}\sum_{j=1}^{N_i}(d_j - \mu_i)^2}
\end{equation}

donde $N_i$ es el número de muestras de la clase $i$ y $d_j$ los valores del descriptor observados. Asumiendo distribuciones normales con varianzas similares, el umbral óptimo entre clases consecutivas que minimiza la probabilidad de error es el punto medio entre sus medias:

\begin{equation}
t_i^* = \frac{\mu_i + \mu_{i+1}}{2}
\end{equation}

Este criterio resulta de igualar las densidades de probabilidad gaussianas de ambas clases y resolver para el punto de intersección.

\subsubsection{Clasificación Multi-Criterio con Distancia Normalizada}

Cuando un único descriptor no proporciona separabilidad suficiente, se emplean múltiples características simultáneamente. Un enfoque común combina descriptores geométricos con información de color o textura. Dado un contorno detectado, se pueden calcular múltiples características como:

\begin{equation}
r_{color} = \frac{N_{píxeles\_color\_objetivo}}{N_{píxeles\_totales}}, \quad I_{media} = \frac{1}{N}\sum_{(x,y) \in R} I(x,y)
\end{equation}

donde $N$ es el número total de píxeles en la región $R$ delimitada por el contorno, $N_{píxeles\_color\_objetivo}$ es el número de píxeles que satisfacen un criterio de color específico, e $I(x,y)$ es la intensidad del píxel.

Cada objeto se representa como un vector en el espacio de características $\mathbf{x} = [d_1, d_2, ..., d_m]^T$ donde cada $d_j$ es un descriptor (área, relación de aspecto, ratio de color, intensidad media, etc.). Para cada clase $c_i$, se define un \textbf{prototipo} mediante las medias $\boldsymbol{\mu}_i = [\mu_{1,i}, \mu_{2,i}, ..., \mu_{m,i}]^T$ y desviaciones estándar $\boldsymbol{\sigma}_i = [\sigma_{1,i}, \sigma_{2,i}, ..., \sigma_{m,i}]^T$ calculadas sobre un conjunto de muestras representativas (conjunto de entrenamiento).

La clasificación de un objeto nuevo se realiza calculando su distancia normalizada a cada prototipo de clase. La normalización por desviación estándar garantiza que descriptores con diferentes rangos contribuyan equitativamente:

\begin{equation}
D_i(\mathbf{x}) = \sqrt{\sum_{k=1}^{m} \left(\frac{x_k - \mu_{k,i}}{\sigma_{k,i} + \epsilon}\right)^2}
\end{equation}

donde $m$ es el número de descriptores, $x_k$ el valor del descriptor $k$ para el objeto, $\mu_{k,i}$ y $\sigma_{k,i}$ los parámetros de la clase $i$, y $\epsilon$ un término pequeño ($10^{-6}$) que evita división por cero. Esta métrica implementa la distancia de Mahalanobis simplificada bajo la asunción de independencia entre descriptores. El objeto se asigna a la clase con menor distancia:

\begin{equation}
\text{Clase}(\mathbf{x}) = \arg\min_{i} D_i(\mathbf{x})
\end{equation}

La confianza de la clasificación se estima mediante la relación entre distancias:

\begin{equation}
\text{Confianza} = 1 - \frac{D_{min}}{D_{max} + D_{min}}
\end{equation}

donde $D_{min}$ es la distancia a la clase predicha y $D_{max}$ la distancia a la clase más lejana. Valores cercanos a 1 indican alta confianza (objeto claramente perteneciente a una clase); valores cercanos a 0.5 sugieren ambigüedad (objeto equidistante entre clases). Esta información de confianza puede emplearse en el nivel supervisor para implementar estrategias de verificación adicional cuando la clasificación es incierta.

\subsubsection{Ventajas y Limitaciones del Enfoque}

\textbf{Resumen:}

El enfoque morfológico para clasificación presenta \textbf{ventajas significativas} como su interpretabilidad (reglas explícitas), eficiencia computacional (funciona en tiempo real sin GPUs), requisitos mínimos de datos (decenas vs miles de imágenes), control directo de parámetros y bajo costo de implementación.

Sin embargo, tiene \textbf{limitaciones importantes}: requiere diseño manual de características con conocimiento experto, tiene capacidad expresiva limitada para patrones complejos, es sensible a variaciones de iluminación y perspectiva, y funciona mejor solo con 2-5 categorías bien diferenciadas, degradándose con más clases o cuando hay solapamiento entre ellas.

En esencia: es un método eficiente, interpretable y económico para problemas de clasificación simples en entornos controlados, pero carece de la flexibilidad y capacidad de generalización del aprendizaje profundo.

Bajo estas condiciones, la combinación de análisis morfológico con métodos estadísticos robustos permite alcanzar exactitudes superiores a 90\% sin la complejidad y requisitos computacionales de métodos de aprendizaje profundo.
