\subsection{Detección de Bordes y Líneas}

La detección de bordes identifica transiciones abruptas de intensidad en imágenes, correspondientes a límites entre objetos o regiones con propiedades visuales distintas. Estos bordes representan características estructurales fundamentales para reconocimiento de formas, medición dimensional y localización de elementos en el espacio. En aplicaciones de robótica agrícola, la detección de bordes permite identificar límites de tubos de cultivo, bordes de recipientes y estructuras de soporte sin depender exclusivamente del color o textura.

\subsubsection{Algoritmo de Canny}

El algoritmo de Canny, propuesto por John Canny en 1986, constituye uno de los métodos más efectivos y ampliamente utilizados para detección de bordes. El algoritmo se fundamenta en tres criterios de optimización: buena detección (baja tasa de falsos positivos y negativos), buena localización (bordes detectados próximos a los bordes reales), y respuesta única (un único punto de borde por cada borde real). La implementación sigue una secuencia de cinco etapas que progresivamente refinan la información hasta obtener bordes delgados y precisos.

\textbf{Reducción de Ruido mediante Filtrado Gaussiano}

La primera etapa aplica un filtro gaussiano bidimensional para suavizar la imagen y reducir ruido que podría generar bordes espurios. El kernel gaussiano se define como:

\begin{equation}
G(x,y) = \frac{1}{2\pi\sigma^2} e^{-\frac{x^2 + y^2}{2\sigma^2}}
\end{equation}

donde $\sigma$ controla el grado de suavizado. La imagen suavizada $I_s$ resulta de la convolución: $I_s = I \ast G$. Valores típicos de $\sigma$ entre 1.0 y 1.5 proporcionan balance entre reducción de ruido y preservación de bordes genuinos.

\textbf{Cálculo de Gradientes}

La segunda etapa calcula la magnitud y dirección del gradiente de intensidad en cada píxel. El gradiente se aproxima mediante operadores de Sobel:

\begin{equation}
G_x = \begin{bmatrix} -1 & 0 & 1 \\ -2 & 0 & 2 \\ -1 & 0 & 1 \end{bmatrix} \ast I_s, \quad G_y = \begin{bmatrix} 1 & 2 & 1 \\ 0 & 0 & 0 \\ -1 & -2 & -1 \end{bmatrix} \ast I_s
\end{equation}

La magnitud del gradiente $M = \sqrt{G_x^2 + G_y^2}$ indica la fuerza del borde, mientras que la dirección $\theta = \arctan(G_y/G_x)$ indica su orientación. Píxeles con alta magnitud de gradiente son candidatos a pertenecer a bordes.

\textbf{Supresión de No Máximos}

La tercera etapa adelgaza los bordes candidatos preservando únicamente píxeles que son máximos locales en la dirección del gradiente. Para cada píxel, se compara su magnitud de gradiente con la de sus dos vecinos en dirección perpendicular al borde. Si la magnitud del píxel no es mayor que ambos vecinos, se descarta. Esta operación garantiza que los bordes tengan grosor de un píxel, mejorando la localización.

\textbf{Doble Umbralización con Histéresis}

La cuarta etapa aplica dos umbrales para clasificar píxeles en tres categorías: bordes fuertes (magnitud > umbral alto), bordes débiles (magnitud entre umbral bajo y alto), y no bordes (magnitud < umbral bajo). Los bordes fuertes se aceptan inmediatamente. Los bordes débiles se aceptan únicamente si están conectados directa o indirectamente a un borde fuerte, mediante un algoritmo de seguimiento de contornos. Esta estrategia de histéresis reduce falsos positivos permitiendo que bordes genuinos con secciones débiles sean preservados, mientras que bordes débiles aislados (probablemente ruido) son rechazados.

La relación típica entre umbrales es $T_{alto} = 2 \cdot T_{bajo}$ o $3 \cdot T_{bajo}$, estableciéndose los valores absolutos experimentalmente según las características de la imagen. Para imágenes de cultivos hidropónicos con buena iluminación, valores como $T_{bajo} = 50$ y $T_{alto} = 150$ (en escala 0-255) resultan efectivos.

\subsubsection{Transformada de Hough para Detección de Líneas}

Una vez detectados los bordes mediante Canny, la transformada de Hough permite identificar estructuras geométricas específicas como líneas rectas. El método se basa en representar cada punto de borde en un espacio de parámetros donde líneas rectas se manifiestan como puntos. Una línea recta en el espacio de imagen se parametriza mediante su distancia perpendicular al origen $\rho$ y el ángulo de la perpendicular $\theta$:

\begin{equation}
\rho = x \cos\theta + y \sin\theta
\end{equation}

Cada píxel de borde $(x_i, y_i)$ genera una curva sinusoidal en el espacio $(\theta, \rho)$ representando todas las líneas que podrían pasar por ese punto. Líneas colineales en el espacio de imagen producen curvas que se intersectan en un punto común del espacio de parámetros. Detectar estas intersecciones mediante acumulación de votos en un arreglo discretizado permite identificar las líneas presentes en la imagen.

La transformada de Hough probabilística optimiza este proceso muestreando aleatoriamente puntos de borde en lugar de procesarlos todos, reduciendo significativamente el costo computacional manteniendo precisión adecuada. El algoritmo retorna los puntos extremos de los segmentos de línea detectados, permitiendo filtrar líneas por longitud mínima y separación máxima entre segmentos.

\subsubsection{Aplicación a Detección de Estructuras de Cultivo}

En el contexto de robótica agrícola para cultivos hidropónicos, la combinación de Canny y Hough permite detectar líneas horizontales correspondientes a bordes superior e inferior de tubos de cultivo. El sistema captura una imagen del tubo, aplica Canny para extraer bordes, y ejecuta Hough configurado para detectar líneas con orientación horizontal (ángulos cercanos a 0° o 180°). Los parámetros clave incluyen la longitud mínima de línea (para filtrar segmentos cortos irrelevantes) y el gap máximo permitido entre segmentos (para unificar líneas fragmentadas por oclusiones parciales).

La detección robusta de ambas líneas (superior e inferior) indica que el tubo está completamente visible en el campo de visión de la cámara, información empleada por el sistema de mapeo autónomo para registrar la posición de estaciones de cultivo. Un sistema de confirmación temporal que requiere detección consistente durante múltiples frames consecutivos (típicamente 5 frames) implementa un debouncing que rechaza detecciones espurias causadas por reflejos o sombras temporales, garantizando robustez en el mapeo del entorno.
