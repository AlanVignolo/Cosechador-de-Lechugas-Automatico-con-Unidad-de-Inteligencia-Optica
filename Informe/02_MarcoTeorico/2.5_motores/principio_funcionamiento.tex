\subsection{Fundamentos de Motores Paso a Paso}

Los motores paso a paso (stepper motors) son actuadores electromecánicos que convierten pulsos eléctricos digitales en movimientos angulares discretos y precisos. A diferencia de los motores de corriente continua convencionales, los motores paso a paso se mueven en incrementos angulares fijos llamados pasos, lo que permite un control de posición en lazo abierto sin necesidad de retroalimentación encoders, aunque estos pueden añadirse para aplicaciones críticas.

\subsubsection{Principio de Funcionamiento}

El funcionamiento de un motor paso a paso se basa en la conmutación secuencial de las corrientes en sus bobinas estatóricas, generando campos magnéticos que interactúan con el rotor magnetizado o dentado. Cada pulso de control provoca un movimiento angular discreto del rotor hacia la siguiente posición de equilibrio magnético estable.

El ángulo de paso $\alpha$ define la resolución angular del motor y se expresa como:

\begin{equation}
\alpha = \frac{360^\circ}{N_p}
\end{equation}

donde $N_p$ es el número de pasos por revolución. Los valores típicos de pasos por revolución son 200 ($1.8^\circ$ por paso), 400 ($0.9^\circ$ por paso), aunque mediante técnicas de microstepping se pueden alcanzar resoluciones mucho mayores.

\subsubsection{Tipos de Motores Paso a Paso}

\paragraph{Motor de Reluctancia Variable (VR):}

Posee un rotor multipolar de hierro dulce (no magnetizado permanentemente) y un estator con bobinas. El rotor tiende a alinearse con el campo magnético del estator para minimizar la reluctancia del circuito magnético. Características principales:

\begin{itemize}
    \item Ángulos de paso típicos: $7.5^\circ$ a $15^\circ$
    \item No produce torque residual (holding torque) sin corriente
    \item Menor costo pero menor precisión
    \item Mayor tendencia a resonancia
\end{itemize}

\paragraph{Motor de Imán Permanente (PM):}

El rotor está compuesto por un imán permanente cilíndrico magnetizado radialmente. El número de polos del rotor determina el ángulo de paso. Características:

\begin{itemize}
    \item Ángulos de paso típicos: $7.5^\circ$ a $90^\circ$
    \item Produce torque de retención sin alimentación
    \item Menor resolución que motores híbridos
    \item Costo moderado
\end{itemize}

\paragraph{Motor Híbrido (HY):}

Combina las ventajas de los motores de reluctancia variable y de imán permanente. El rotor contiene un imán permanente axialmente magnetizado con dientes en sus extremos. Es el tipo más utilizado en aplicaciones industriales y de precisión. Características:

\begin{itemize}
    \item Ángulos de paso típicos: $0.9^\circ$ a $3.6^\circ$ (más comúnmente $1.8^\circ$)
    \item Alto torque de retención y dinámico
    \item Excelente precisión y repetibilidad
    \item Mayor costo pero mejor desempeño
\end{itemize}

\subsection{Modos de Excitación y Control}

El modo de excitación define cómo se energizan las bobinas del motor, afectando directamente el torque, la precisión, el consumo de energía y la suavidad del movimiento.

\subsubsection{Excitación de Paso Completo (Full Step)}

\paragraph{Modo de Una Fase (Wave Drive):}

Solo una bobina se energiza a la vez. La secuencia de activación para un motor bipolar de dos fases es:

\begin{table}[ht]
\centering
\caption{Secuencia de excitación Wave Drive}
\begin{tabular}{|c|c|c|}
\hline
\textbf{Paso} & \textbf{Fase A} & \textbf{Fase B} \\
\hline
1 & +I & 0 \\
2 & 0 & +I \\
3 & -I & 0 \\
4 & 0 & -I \\
\hline
\end{tabular}
\end{table}

Características:
\begin{itemize}
    \item Menor consumo de corriente (50\% respecto a dos fases)
    \item Torque reducido (aproximadamente 70\% del torque nominal)
    \item Mayor resonancia a ciertas velocidades
\end{itemize}

\paragraph{Modo de Dos Fases (Full Step):}

Dos bobinas se energizan simultáneamente. La secuencia de activación es:

\begin{table}[ht]
\centering
\caption{Secuencia de excitación Full Step}
\begin{tabular}{|c|c|c|}
\hline
\textbf{Paso} & \textbf{Fase A} & \textbf{Fase B} \\
\hline
1 & +I & +I \\
2 & -I & +I \\
3 & -I & -I \\
4 & +I & -I \\
\hline
\end{tabular}
\end{table}

Características:
\begin{itemize}
    \item Máximo torque disponible (100\% del torque nominal)
    \item Mayor consumo de corriente y generación de calor
    \item Mejor estabilidad posicional
\end{itemize}

\subsubsection{Excitación de Medio Paso (Half Step)}

Alterna entre la energización de una y dos fases, duplicando la resolución del motor. La secuencia combina los modos anteriores:

\begin{table}[ht]
\centering
\caption{Secuencia de excitación Half Step}
\begin{tabular}{|c|c|c|}
\hline
\textbf{Paso} & \textbf{Fase A} & \textbf{Fase B} \\
\hline
1 & +I & 0 \\
2 & +I & +I \\
3 & 0 & +I \\
4 & -I & +I \\
5 & -I & 0 \\
6 & -I & -I \\
7 & 0 & -I \\
8 & +I & -I \\
\hline
\end{tabular}
\end{table}

Para un motor de 200 pasos/rev ($1.8^\circ$), el half stepping proporciona 400 pasos/rev ($0.9^\circ$).

Características:
\begin{itemize}
    \item Resolución duplicada respecto al full step
    \item Torque variable entre pasos (70-100\%)
    \item Mayor suavidad de movimiento
    \item Reducción de resonancia
\end{itemize}

\subsubsection{Microstepping}

El microstepping es una técnica avanzada que subdivide cada paso completo en múltiples micropasos mediante la modulación sinusoidal de las corrientes en las bobinas. Las corrientes en las fases A y B se controlan según:

\begin{equation}
\begin{aligned}
I_A(\theta) &= I_{\max} \cos(\theta) \\
I_B(\theta) &= I_{\max} \sin(\theta)
\end{aligned}
\end{equation}

donde $\theta$ es el ángulo eléctrico deseado e $I_{\max}$ es la corriente máxima nominal.

Para $M$ micropasos por paso completo, el ángulo de micropaso es:

\begin{equation}
\alpha_{\mu} = \frac{\alpha}{M} = \frac{360^\circ}{N_p \cdot M}
\end{equation}

Niveles comunes de microstepping incluyen 1/4, 1/8, 1/16, 1/32, hasta 1/256 pasos.

\paragraph{Ventajas del Microstepping:}

\begin{itemize}
    \item Resolución extremadamente alta (hasta 51,200 pasos/rev con 1/256)
    \item Movimiento muy suave y silencioso
    \item Eliminación virtual de resonancia a bajas velocidades
    \item Mejor precisión en posicionamiento fino
\end{itemize}

\paragraph{Limitaciones del Microstepping:}

\begin{itemize}
    \item La resolución teórica no garantiza precisión posicional debido a no linealidades
    \item Reducción de torque en micropasos intermedios
    \item Mayor complejidad del driver y procesamiento requerido
    \item Sensibilidad a fricción y errores mecánicos
\end{itemize}

\subsection{Características de Desempeño}

\subsubsection{Curvas de Torque}

El desempeño de un motor paso a paso se caracteriza principalmente mediante dos curvas de torque:

\paragraph{Torque de Retención (Holding Torque):}

Es el torque máximo que el motor puede ejercer cuando está energizado pero estacionario. Representa la capacidad de mantener la posición contra cargas externas. Se expresa en N·m o kg·cm y es un parámetro fundamental de selección.

\paragraph{Torque Dinámico (Pull-out Torque):}

Es el torque máximo disponible durante el movimiento a una velocidad dada. Esta curva es crítica para determinar si el motor puede acelerar y mantener una carga específica a la velocidad deseada.

La relación entre torque y velocidad se aproxima por:

\begin{equation}
T(\omega) = T_0 \sqrt{1 - \left(\frac{\omega}{\omega_{\max}}\right)^2}
\end{equation}

donde $T_0$ es el torque a velocidad cero (aproximadamente el holding torque), $\omega$ es la velocidad angular, y $\omega_{\max}$ es la velocidad máxima sin carga.

\subsubsection{Inercia del Rotor}

La inercia rotacional del rotor $J_r$ es crucial para la dinámica del sistema. El torque requerido para acelerar el motor con una carga es:

\begin{equation}
T_{req} = (J_r + J_L) \alpha + T_L + T_f
\end{equation}

donde:
\begin{itemize}
    \item $J_r$: Inercia del rotor
    \item $J_L$: Inercia de la carga reflejada al eje del motor
    \item $\alpha$: Aceleración angular deseada
    \item $T_L$: Torque de carga
    \item $T_f$: Torque de fricción
\end{itemize}

Para optimizar el desempeño dinámico, se recomienda que la inercia total no exceda 10 veces la inercia del rotor:

\begin{equation}
J_L \leq 10 \cdot J_r
\end{equation}

\subsubsection{Resonancia}

Los motores paso a paso exhiben resonancia mecánica a ciertas velocidades, típicamente entre 100-300 Hz, donde el sistema puede perder pasos o vibrar excesivamente. Las frecuencias de resonancia natural se calculan mediante:

\begin{equation}
f_n = \frac{1}{2\pi}\sqrt{\frac{K_s}{J_r + J_L}}
\end{equation}

donde $K_s$ es la rigidez torsional del sistema.

Estrategias de mitigación incluyen:
\begin{itemize}
    \item Utilizar microstepping
    \item Implementar perfiles de aceleración rampeados
    \item Añadir amortiguadores mecánicos o viscosos
    \item Evitar velocidades resonantes mediante control de velocidad
\end{itemize}

\subsection{Drivers y Electrónica de Control}

\subsubsection{Tipos de Drivers}

\paragraph{Driver Unipolar:}

Diseñado para motores unipolares (generalmente con 5 o 6 cables). Utiliza transistores simples para conmutar las corrientes. Ventajas: simplicidad y bajo costo. Desventajas: menor eficiencia y torque (aproximadamente 70\% del equivalente bipolar).

\paragraph{Driver Bipolar:}

Para motores bipolares (4 cables). Requiere puentes H para invertir la polaridad de las corrientes. Mayor complejidad pero mejor desempeño. Topologías comunes:

\begin{itemize}
    \item \textbf{L/R Drive}: Control de voltaje simple, genera calor excesivo
    \item \textbf{Chopper Drive}: Modulación PWM de corriente, mayor eficiencia
    \item \textbf{Bipolar con control de corriente}: Regulación precisa mediante sensado de corriente
\end{itemize}

\subsubsection{Control de Corriente por Chopping}

El método de chopping mantiene la corriente en las bobinas mediante modulación PWM, comparando la corriente medida con una referencia:

\begin{equation}
D = \frac{t_{on}}{T_{PWM}} = \frac{R \cdot I_{ref} + L \frac{di}{dt}}{V_{supply}}
\end{equation}

donde $D$ es el ciclo de trabajo, $R$ y $L$ son la resistencia e inductancia de la bobina, $I_{ref}$ es la corriente de referencia, y $V_{supply}$ es el voltaje de alimentación.

Modos de chopping:
\begin{itemize}
    \item \textbf{Fast decay}: Ambos transistores del puente H se apagan, corriente decae rápidamente
    \item \textbf{Slow decay}: Un transistor permanece encendido, decaimiento lento
    \item \textbf{Mixed decay}: Combinación de ambos para optimizar desempeño
\end{itemize}

\subsubsection{Interfaces de Control}

\paragraph{Step/Direction:}

La interfaz más común. Dos señales digitales:
\begin{itemize}
    \item \textbf{STEP}: Cada pulso produce un paso (o micropaso)
    \item \textbf{DIRECTION}: Define el sentido de rotación (HIGH/LOW)
\end{itemize}

Adicionalmente pueden incluirse señales ENABLE (habilitar motor) y FAULT (indicación de error).

\paragraph{Control por Pulsos y Sentido:}

Variante donde pulsos en dos líneas diferentes determinan dirección:
\begin{itemize}
    \item Pulsos en CW: Rotación horaria
    \item Pulsos en CCW: Rotación antihoraria
\end{itemize}

\paragraph{Comunicación Serial/Bus:}

Drivers avanzados incorporan interfaces RS-485, CANbus, o EtherCAT para configuración de parámetros, diagnóstico y control centralizado en sistemas multi-eje.

\subsection{Generación de Perfiles de Movimiento}

\subsubsection{Perfil Trapezoidal}

El perfil de velocidad trapezoidal es el más utilizado por su simplicidad y efectividad. Consta de tres fases:

\begin{enumerate}
    \item \textbf{Aceleración}: Incremento lineal de velocidad desde reposo hasta velocidad objetivo
    \item \textbf{Velocidad constante}: Mantenimiento de la velocidad máxima
    \item \textbf{Deceleración}: Reducción lineal hasta detenerse
\end{enumerate}

Para un desplazamiento angular $\theta_{total}$, con aceleración $\alpha$ y velocidad máxima $\omega_{max}$:

Tiempo de aceleración:
\begin{equation}
t_{acc} = \frac{\omega_{max}}{\alpha}
\end{equation}

Ángulo durante aceleración:
\begin{equation}
\theta_{acc} = \frac{1}{2}\alpha t_{acc}^2 = \frac{\omega_{max}^2}{2\alpha}
\end{equation}

Si el movimiento permite alcanzar $\omega_{max}$:
\begin{equation}
\theta_{cruise} = \theta_{total} - 2\theta_{acc}
\end{equation}

Tiempo total:
\begin{equation}
t_{total} = 2t_{acc} + \frac{\theta_{cruise}}{\omega_{max}}
\end{equation}

Para movimientos cortos donde no se alcanza $\omega_{max}$, la velocidad pico es:
\begin{equation}
\omega_{peak} = \sqrt{\alpha \cdot \theta_{total}}
\end{equation}

\subsubsection{Perfil S-Curve}

Para reducir vibraciones y mejorar la suavidad, se utilizan perfiles con aceleración variable (jerk limitado). El perfil S-curve tiene siete fases: incremento de aceleración, aceleración constante, decremento de aceleración, velocidad constante, y las fases simétricas de deceleración.

El jerk máximo $j_{max}$ determina la suavidad:
\begin{equation}
\alpha(t) = \begin{cases}
j_{max} \cdot t & 0 \leq t < t_1 \\
\alpha_{max} & t_1 \leq t < t_2 \\
\alpha_{max} - j_{max}(t - t_2) & t_2 \leq t < t_3
\end{cases}
\end{equation}

\subsection{Criterios de Selección de Motores Paso a Paso}

La selección adecuada de un motor paso a paso requiere un análisis sistemático de los requisitos de la aplicación y las características del motor. El proceso se estructura en los siguientes pasos:

\subsubsection{Paso 1: Definición de Requisitos de la Aplicación}

\paragraph{Requisitos Cinemáticos:}

\begin{itemize}
    \item \textbf{Resolución angular requerida ($\alpha_{req}$)}: Determina el ángulo de paso básico y nivel de microstepping necesario
    
    \begin{equation}
    \alpha_{req} = \frac{\text{resolución lineal}}{\text{radio efectivo}} \quad \text{(para sistemas rotativos)}
    \end{equation}
    
    \item \textbf{Velocidad máxima ($\omega_{max}$)}: Velocidad de operación en RPM o rad/s
    
    \item \textbf{Aceleración requerida ($\alpha_{req}$)}: Para cumplir tiempos de ciclo especificados
    
    \item \textbf{Rango de movimiento}: Total de desplazamiento angular o lineal
\end{itemize}

\paragraph{Requisitos Dinámicos:}

\begin{itemize}
    \item \textbf{Torque de carga ($T_L$)}: Torque resistivo durante operación normal
    
    \item \textbf{Inercia de carga ($J_L$)}: Momento de inercia de todos los componentes móviles
    
    Para sistemas lineales con masa $m$ y radio de polea/husillo $r$:
    \begin{equation}
    J_L = m \cdot r^2
    \end{equation}
    
    Para sistemas con reducción de transmisión con relación $N$:
    \begin{equation}
    J_{L,ref} = \frac{J_L}{N^2}
    \end{equation}
    
    \item \textbf{Fricción ($T_f$)}: Torque de fricción estática y dinámica
\end{itemize}

\paragraph{Requisitos Ambientales y Operacionales:}

\begin{itemize}
    \item Temperatura de operación
    \item Nivel de protección requerido (IP rating)
    \item Ciclo de trabajo (continuo o intermitente)
    \item Vida útil esperada
    \item Restricciones de espacio y peso
\end{itemize}

\subsubsection{Paso 2: Cálculo de Torque Requerido}

El torque total requerido incluye componentes estáticos y dinámicos:

\begin{equation}
T_{req,total} = T_{acc} + T_L + T_f + T_{safety}
\end{equation}

\paragraph{Torque de Aceleración:}

\begin{equation}
T_{acc} = (J_r + J_L) \cdot \alpha
\end{equation}

donde $\alpha$ es la aceleración angular en rad/s².

\paragraph{Torque de Carga:}

Para aplicaciones lineales con fuerza resistiva $F$ y radio $r$:
\begin{equation}
T_L = F \cdot r \cdot \frac{1}{\eta}
\end{equation}

donde $\eta$ es la eficiencia del sistema de transmisión (típicamente 0.85-0.95 para sistemas con rodamientos de bolas).

Para elevación vertical:
\begin{equation}
T_L = m \cdot g \cdot r \cdot \frac{1}{\eta}
\end{equation}

\paragraph{Margen de Seguridad:}

Se recomienda aplicar un factor de seguridad de 1.5 a 2.0:

\begin{equation}
T_{motor,min} = T_{req,total} \cdot SF
\end{equation}

donde $SF$ es el factor de seguridad (típicamente 1.5-2.0).

\subsubsection{Paso 3: Verificación de Torque-Velocidad}

Utilizando las curvas torque-velocidad del fabricante, verificar que:

\begin{equation}
T_{available}(\omega_{max}) \geq T_{req}(\omega_{max})
\end{equation}

El torque disponible debe ser suficiente a la velocidad máxima de operación. Si no se cumple, considerar:

\begin{itemize}
    \item Seleccionar un motor de mayor tamaño
    \item Implementar reducción mecánica
    \item Reducir la velocidad de operación
    \item Utilizar un driver con mayor voltaje de alimentación
\end{itemize}

\subsubsection{Paso 4: Verificación de Relación de Inercias}

Calcular la relación de inercias:

\begin{equation}
R_J = \frac{J_L}{J_r}
\end{equation}

Criterios recomendados:
\begin{itemize}
    \item $R_J \leq 5$: Excelente respuesta dinámica
    \item $5 < R_J \leq 10$: Aceptable para la mayoría de aplicaciones
    \item $R_J > 10$: Considerar motor de mayor inercia o reducción mecánica
\end{itemize}

Si $R_J$ es excesivo, las opciones incluyen:
\begin{itemize}
    \item Seleccionar motor con mayor frame size (mayor $J_r$)
    \item Implementar reducción de transmisión (reduce $J_L$ por $N^2$)
    \item Rediseñar componentes móviles para reducir masa/inercia
\end{itemize}

\subsubsection{Paso 5: Selección de Driver y Voltaje de Alimentación}

\paragraph{Corriente del Driver:}

El driver debe soportar la corriente nominal del motor con margen:

\begin{equation}
I_{driver} \geq 1.2 \cdot I_{motor,nominal}
\end{equation}

\paragraph{Voltaje de Alimentación:}

Un voltaje mayor permite mejor desempeño a altas velocidades. La regla empírica para voltaje óptimo:

\begin{equation}
V_{supply,opt} = 32 \sqrt{L_{phase}}
\end{equation}

donde $L_{phase}$ es la inductancia de fase en mH, y $V_{supply}$ resulta en voltios.

Alternativamente, utilizar:
\begin{equation}
V_{supply} = (10 \text{ a } 20) \times V_{rated}
\end{equation}

donde $V_{rated}$ es el voltaje nominal del motor.

\paragraph{Nivel de Microstepping:}

Seleccionar según resolución requerida:

\begin{equation}
M = \frac{\alpha_{full}}{\alpha_{req}}
\end{equation}

Niveles comunes: 1 (full step), 2 (half step), 4, 8, 16, 32, 64, 128, 256.

\subsubsection{Paso 6: Consideraciones Térmicas}

Verificar que la disipación térmica del motor sea adecuada:

\paragraph{Potencia Disipada:}

\begin{equation}
P_{diss} = I_{RMS}^2 \cdot R_{phase} \cdot N_{phases}
\end{equation}

Para operación continua:
\begin{equation}
P_{diss} \leq P_{max,cont}
\end{equation}

Para operación intermitente con ciclo de trabajo $DC$:
\begin{equation}
P_{avg} = P_{diss} \cdot DC \leq P_{max,cont}
\end{equation}

\paragraph{Elevación de Temperatura:}

La temperatura del motor no debe exceder:
\begin{equation}
T_{motor} = T_{ambient} + \Delta T_{rise} \leq T_{max,rated}
\end{equation}

Típicamente, $T_{max,rated} = 80$--$100^\circ$C para motores estándar.

Si el análisis térmico indica sobrecalentamiento:
\begin{itemize}
    \item Reducir corriente de operación (sacrificando torque)
    \item Añadir disipadores o ventilación forzada
    \item Reducir ciclo de trabajo
    \item Seleccionar motor de mayor frame size
\end{itemize}

\subsubsection{Paso 7: Validación del Sistema de Transmisión}

Si se utiliza transmisión mecánica (poleas, husillos, reductores):

\paragraph{Para Sistemas con Husillo de Bolas:}

Paso del husillo $p$ (mm/rev), la velocidad lineal es:
\begin{equation}
v_{linear} = \omega_{motor} \cdot \frac{p}{2\pi}
\end{equation}

La fuerza disponible:
\begin{equation}
F_{available} = \frac{T_{motor} \cdot 2\pi \cdot \eta}{p}
\end{equation}

\paragraph{Para Sistemas de Poleas/Correas:}

Con diámetro de polea $D$ y relación de transmisión $N$:

\begin{equation}
v_{linear} = \omega_{motor} \cdot \frac{D}{2N}
\end{equation}

\begin{equation}
F_{available} = \frac{2 \cdot T_{motor} \cdot N \cdot \eta}{D}
\end{equation}

\subsubsection{Paso 8: Verificación de Resonancia}

Estimar frecuencias críticas y planificar estrategias de mitigación:

\begin{equation}
f_{step,resonance} = \frac{\omega_{resonance} \cdot N_p}{2\pi}
\end{equation}

Si la velocidad operativa coincide con resonancia, implementar:
\begin{itemize}
    \item Microstepping para suavizar excitación
    \item Perfiles de aceleración S-curve
    \item Cambio de velocidad de crucero
    \item Amortiguadores mecánicos
\end{itemize}

\subsection{Ejemplo de Aplicación del Proceso de Selección}

\paragraph{Especificaciones de Diseño:}

\begin{itemize}
    \item Sistema de posicionamiento lineal mediante husillo de bolas
    \item Masa de carga: $m = 5$ kg
    \item Recorrido: 300 mm
    \item Velocidad máxima: 100 mm/s
    \item Tiempo de aceleración: 0.2 s
    \item Paso de husillo: $p = 5$ mm/rev
    \item Eficiencia del husillo: $\eta = 0.9$
    \item Precisión requerida: $\pm 0.01$ mm
\end{itemize}

\paragraph{Paso 1 - Resolución:}

\begin{equation}
\alpha_{req} = \frac{0.01 \text{ mm}}{5 \text{ mm/rev}} \times 360^\circ = 0.72^\circ
\end{equation}

Motor de $1.8^\circ$ con microstepping de 1/4 proporciona: $1.8^\circ/4 = 0.45^\circ$ (suficiente).

\paragraph{Paso 2 - Velocidad del Motor:}

\begin{equation}
\omega_{max} = \frac{v_{max}}{p} = \frac{100 \text{ mm/s}}{5 \text{ mm/rev}} = 20 \text{ rev/s} = 1200 \text{ RPM}
\end{equation}

\paragraph{Paso 3 - Aceleración:}

\begin{equation}
a_{linear} = \frac{v_{max}}{t_{acc}} = \frac{0.1 \text{ m/s}}{0.2 \text{ s}} = 0.5 \text{ m/s}^2
\end{equation}

\begin{equation}
\alpha_{motor} = \frac{a_{linear}}{p/(2\pi)} = \frac{0.5}{0.005/(2\pi)} = 628.3 \text{ rad/s}^2
\end{equation}

\paragraph{Paso 4 - Torque de Carga:}

Considerando elevación vertical máxima (peor caso):
\begin{equation}
T_L = \frac{m \cdot g \cdot p}{2\pi \cdot \eta} = \frac{5 \times 9.81 \times 0.005}{2\pi \times 0.9} = 0.043 \text{ N·m}
\end{equation}

\paragraph{Paso 5 - Inercia Reflejada:}

\begin{equation}
J_L = m \left(\frac{p}{2\pi}\right)^2 = 5 \times \left(\frac{0.005}{2\pi}\right)^2 = 3.16 \times 10^{-6} \text{ kg·m}^2
\end{equation}

\paragraph{Paso 6 - Torque de Aceleración:}

Asumiendo motor NEMA 23 con $J_r = 5 \times 10^{-5}$ kg·m²:

\begin{equation}
T_{acc} = (J_r + J_L) \times \alpha = (5 \times 10^{-5} + 3.16 \times 10^{-6}) \times 628.3 = 0.033 \text{ N·m}
\end{equation}

\paragraph{Paso 7 - Torque Total con Factor de Seguridad:}

Estimando $T_f = 0.01$ N·m:

\begin{equation}
T_{req} = (0.043 + 0.033 + 0.01) \times 1.5 = 0.129 \text{ N·m}
\end{equation}

\paragraph{Paso 8 - Selección del Motor:}

Se selecciona un motor NEMA 23 con las siguientes especificaciones:
\begin{itemize}
    \item Holding torque: 1.9 N·m
    \item Ángulo de paso: $1.8^\circ$
    \item Corriente nominal: 2.8 A por fase
    \item Resistencia de fase: 1.2 $\Omega$
    \item Inductancia de fase: 3.2 mH
    \item Inercia del rotor: $5.4 \times 10^{-5}$ kg·m²
\end{itemize}

\paragraph{Paso 9 - Verificación de Relación de Inercias:}

\begin{equation}
R_J = \frac{J_L}{J_r} = \frac{3.16 \times 10^{-6}}{5.4 \times 10^{-5}} = 0.059
\end{equation}

Excelente relación ($R_J < 1$), garantiza respuesta dinámica óptima.

\paragraph{Paso 10 - Verificación de Curva Torque-Velocidad:}

A 1200 RPM, según curva del fabricante con alimentación de 48V, el torque disponible es aproximadamente 0.5 N·m, lo cual excede ampliamente el requerimiento de 0.129 N·m. Margen de seguridad real:

\begin{equation}
MS = \frac{T_{available}}{T_{req}} = \frac{0.5}{0.129} = 3.88
\end{equation}

\paragraph{Paso 11 - Selección del Driver:}

Driver con las siguientes características:
\begin{itemize}
    \item Corriente máxima: 4.0 A (cumple $4.0 > 1.2 \times 2.8 = 3.36$ A)
    \item Voltaje de alimentación: 48 V DC
    \item Microstepping: hasta 1/256 (se configurará en 1/4)
    \item Interfaz: Step/Direction
\end{itemize}

Verificación de voltaje óptimo:
\begin{equation}
V_{opt} = 32\sqrt{L_{phase}} = 32\sqrt{3.2} = 57.3 \text{ V}
\end{equation}

El voltaje de 48V es adecuado (dentro del rango recomendado).

\paragraph{Paso 12 - Análisis Térmico:}

Potencia disipada en operación continua:
\begin{equation}
P_{diss} = 2 \times (2.8)^2 \times 1.2 = 18.8 \text{ W}
\end{equation}

Si el motor tiene capacidad de disipación de 25 W en operación continua, el sistema operará dentro de límites térmicos seguros con margen del 33\%.

\subsection{Estrategias de Control Avanzadas}

\subsubsection{Control en Lazo Cerrado}

Aunque los motores paso a paso tradicionalmente operan en lazo abierto, las aplicaciones críticas implementan control en lazo cerrado mediante:

\paragraph{Encoders:}

Proporcionan retroalimentación de posición real. Tipos comunes:
\begin{itemize}
    \item Encoders incrementales: Generan pulsos proporcionales al movimiento
    \item Encoders absolutos: Proporcionan posición única en todo el rango
\end{itemize}

El error de seguimiento se define como:
\begin{equation}
e(t) = \theta_{desired}(t) - \theta_{actual}(t)
\end{equation}

\paragraph{Control PID:}

Un controlador PID ajusta la velocidad de comando basado en el error:

\begin{equation}
u(t) = K_p e(t) + K_i \int_0^t e(\tau)d\tau + K_d \frac{de(t)}{dt}
\end{equation}

donde $K_p$, $K_i$ y $K_d$ son las ganancias proporcional, integral y derivativa.

\paragraph{Detección de Pérdida de Pasos:}

Comparación continua entre pasos comandados y posición real del encoder:

\begin{equation}
\Delta\theta = |\theta_{commanded} - \theta_{encoder}| > \theta_{threshold}
\end{equation}

Si $\Delta\theta$ excede el umbral, se activa alarma o procedimiento de recuperación.

\subsubsection{Control de Torque y Corriente}

Los drivers avanzados implementan control dinámico de corriente para optimizar eficiencia y reducir calentamiento:

\paragraph{Reducción Automática de Corriente en Reposo:}

Cuando el motor está estacionario, la corriente se reduce a un porcentaje del valor nominal (típicamente 30-50\%):

\begin{equation}
I_{hold} = \beta \cdot I_{nominal}, \quad 0.3 \leq \beta \leq 0.5
\end{equation}

Esto mantiene el holding torque necesario mientras reduce disipación térmica.

\paragraph{Control Adaptativo de Corriente:}

La corriente se ajusta dinámicamente según la velocidad y carga:

\begin{equation}
I_{dynamic}(\omega) = I_{nominal} \times \left(1 - k \frac{\omega}{\omega_{max}}\right)
\end{equation}

donde $k$ es un factor de ajuste (típicamente 0.2-0.4).

\subsubsection{Compensación de Resonancia}

Técnicas avanzadas para mitigar resonancia incluyen:

\paragraph{Frequency Hopping:}

Variación automática de la frecuencia de pasos al detectar resonancia:

\begin{equation}
f_{adjusted} = f_{target} \pm \Delta f_{hop}
\end{equation}

\paragraph{Damping Electrónico:}

Inyección de componentes de corriente que generan amortiguamiento artificial:

\begin{equation}
I_{damping} = -K_d \cdot \omega_{rotor}
\end{equation}

donde $K_d$ es la constante de amortiguamiento y $\omega_{rotor}$ es la velocidad angular del rotor.

\subsection{Consideraciones Prácticas de Implementación}

\subsubsection{Conexionado y Cableado}

\paragraph{Identificación de Bobinas:}

Para motores bipolares de 4 cables, identificar pares de bobinas mediante medición de resistencia:
\begin{itemize}
    \item Entre cables del mismo par: $R = R_{phase}$ (típicamente 1-10 $\Omega$)
    \item Entre cables de pares diferentes: $R = \infty$ (circuito abierto)
\end{itemize}

\paragraph{Longitud de Cables:}

Cables largos aumentan inductancia y resistencia parásitas. Límites recomendados:
\begin{itemize}
    \item Cables de potencia del motor: < 10 m
    \item Señales Step/Direction: < 5 m (o usar señales diferenciales para distancias mayores)
\end{itemize}

La caída de voltaje en cables debe ser mínima:
\begin{equation}
V_{drop} = I_{motor} \times R_{cable} < 0.05 \times V_{supply}
\end{equation}

\paragraph{Apantallamiento y Tierra:}

Para reducir EMI (interferencia electromagnética):
\begin{itemize}
    \item Usar cable apantallado para señales de control
    \item Conectar el apantallamiento a tierra en un solo punto
    \item Separar cables de potencia de cables de señal
    \item Implementar ferrites en cables de motor
\end{itemize}

\subsubsection{Protecciones y Diagnóstico}

\paragraph{Protecciones del Driver:}

Los drivers modernos incorporan:
\begin{itemize}
    \item \textbf{Protección de sobrecorriente}: Desconexión si $I > I_{max}$
    \item \textbf{Protección térmica}: Shutdown si temperatura excede límite
    \item \textbf{Protección de bajo voltaje}: Alarma si $V_{supply} < V_{min}$
    \item \textbf{Protección de cortocircuito}: Detección de cortocircuitos fase-fase o fase-tierra
\end{itemize}

\paragraph{Señales de Diagnóstico:}

Monitoreo continuo de:
\begin{equation}
\begin{aligned}
\text{Alarmas} &= \{V_{supply}, I_{phase}, T_{driver}, \text{Fault codes}\} \\
\text{Status} &= \{\text{Enable}, \text{Position error}, \text{Motor stall}\}
\end{aligned}
\end{equation}

\subsubsection{Configuración Inicial y Puesta en Marcha}

\paragraph{Procedimiento de Configuración:}

\begin{enumerate}
    \item \textbf{Configurar corriente del motor}: Ajustar mediante potenciómetro o software
    \begin{equation}
    V_{ref} = \frac{I_{desired} \times R_{sense} \times 8}{K_{driver}}
    \end{equation}
    donde $R_{sense}$ es la resistencia de sensado y $K_{driver}$ es la ganancia del driver (consultar datasheet).
    
    \item \textbf{Seleccionar nivel de microstepping}: Mediante DIP switches o configuración digital
    
    \item \textbf{Ajustar decay mode}: Fast, slow o mixed según aplicación
    
    \item \textbf{Configurar señales de control}: Polaridad de Step/Direction, lógica de Enable
\end{enumerate}

\paragraph{Pruebas de Validación:}

\begin{enumerate}
    \item \textbf{Test de rotación libre}: Sin carga, verificar rotación suave sin vibraciones excesivas
    
    \item \textbf{Test de holding torque}: Aplicar torque manual y verificar que el motor mantiene posición
    
    \item \textbf{Test de velocidad}: Rampear velocidad gradualmente hasta máxima operativa
    
    \item \textbf{Test de aceleración}: Verificar que no hay pérdida de pasos durante aceleración máxima
    
    \item \textbf{Test térmico}: Operar en ciclo representativo y medir temperatura después de 30 min
    
    \item \textbf{Test de precisión}: Verificar repetibilidad posicional mediante múltiples ciclos
\end{enumerate}

\subsection{Análisis de Fallos Comunes}

\subsubsection{Pérdida de Pasos}

\paragraph{Causas:}

\begin{itemize}
    \item Torque requerido excede torque disponible
    \item Aceleración excesiva
    \item Resonancia mecánica
    \item Voltaje de alimentación insuficiente a alta velocidad
    \item Carga mecánica variable o impactos
\end{itemize}

\paragraph{Diagnóstico:}

Verificar mediante encoder o test repetitivo:
\begin{equation}
N_{cycles} = 100, \quad \text{error final} > N_{steps} \times 0.01
\end{equation}

\paragraph{Soluciones:}

\begin{itemize}
    \item Aumentar corriente del motor (si térmicamente posible)
    \item Reducir aceleración máxima
    \item Implementar microstepping
    \item Incrementar voltaje de alimentación
    \item Seleccionar motor de mayor torque
    \item Añadir reducción mecánica
\end{itemize}

\subsubsection{Resonancia y Vibración}

\paragraph{Identificación:}

Realizar barrido de frecuencias y detectar velocidades donde vibración es máxima:

\begin{equation}
\text{Vibration amplitude} = f(\omega) \rightarrow \text{peak at } \omega_{resonance}
\end{equation}

\paragraph{Mitigación:}

\begin{itemize}
    \item Cambiar a microstepping (mínimo 1/8)
    \item Evitar velocidades resonantes en trayectorias
    \item Implementar perfiles S-curve
    \item Añadir masa o amortiguadores mecánicos
    \item Utilizar drivers con anti-resonance features
\end{itemize}

\subsubsection{Sobrecalentamiento}

\paragraph{Límites Térmicos:}

La temperatura superficial del motor no debe exceder:
\begin{equation}
T_{surface} < T_{ambient} + \Delta T_{rated}
\end{equation}

Típicamente, $\Delta T_{rated} = 80^\circ$C sobre ambiente.

\paragraph{Soluciones:}

\begin{itemize}
    \item Reducir corriente en configuración del driver (sacrificando torque)
    \item Implementar reducción de corriente en reposo
    \item Mejorar ventilación o añadir disipadores
    \item Reducir ciclo de trabajo
    \item Seleccionar motor de mayor frame size con mejor capacidad térmica
\end{itemize}

\subsubsection{Ruido Acústico Excesivo}

\paragraph{Causas:}

\begin{itemize}
    \item Frecuencia de pasos en rango audible (< 20 kHz)
    \item Resonancia estructural
    \item Modo full step sin suavizado
    \item Componentes mecánicos sueltos
\end{itemize}

\paragraph{Reducción de Ruido:}

\begin{itemize}
    \item Implementar microstepping alto ( $\geq$ 1/16)
    \item Utilizar perfiles de aceleración suaves
    \item Verificar y ajustar fijaciones mecánicas
    \item Añadir materiales de amortiguamiento
    \item Reducir corriente si el torque lo permite
\end{itemize}

\subsection{Aplicaciones Típicas y Consideraciones Específicas}

\subsubsection{Impresoras 3D y Máquinas CNC}

\paragraph{Requisitos:}

\begin{itemize}
    \item Alta precisión (típicamente $\pm 0.01$ - $\pm 0.1$ mm)
    \item Operación continua por períodos prolongados
    \item Sincronización precisa de múltiples ejes
    \item Bajo ruido
\end{itemize}

\paragraph{Configuración Típica:}

\begin{itemize}
    \item Motores NEMA 17 o NEMA 23
    \item Microstepping 1/16 o superior
    \item Control por firmware (Marlin, GRBL)
    \item Drivers con control de corriente por chopper
\end{itemize}

\subsubsection{Telescopios y Sistemas de Seguimiento}

\paragraph{Requisitos:}

\begin{itemize}
    \item Movimiento extremadamente suave
    \item Velocidades muy bajas con alta precisión
    \item Minimización de vibraciones
    \item Operación silenciosa
\end{itemize}

\paragraph{Configuración Típica:}

\begin{itemize}
    \item Microstepping 1/64 a 1/256
    \item Engranajes de reducción de alta calidad
    \item Drivers con decay modes optimizados
    \item Control en lazo cerrado para tracking preciso
\end{itemize}

\subsubsection{Automatización Industrial y Pick-and-Place}

\paragraph{Requisitos:}

\begin{itemize}
    \item Alta velocidad y aceleración
    \item Repetibilidad (típicamente $\pm 0.05$ mm)
    \item Ciclos de trabajo intensivos
    \item Confiabilidad a largo plazo
\end{itemize}

\paragraph{Configuración Típica:}

\begin{itemize}
    \item Motores NEMA 23 o NEMA 34 de alto torque
    \item Voltajes de alimentación elevados (48-80V)
    \item Encoders para verificación de posición
    \item Sistemas de enfriamiento activo si es necesario
\end{itemize}

\subsubsection{Sistemas Médicos y de Laboratorio}

\paragraph{Requisitos:}

\begin{itemize}
    \item Precisión extrema ($< 1 \mu$m en algunos casos)
    \item Movimiento suave y libre de vibraciones
    \item Operación silenciosa
    \item Cumplimiento de normativas médicas
\end{itemize}

\paragraph{Configuración Típica:}

\begin{itemize}
    \item Motores de alta resolución con encoders de alta precisión
    \item Microstepping máximo disponible
    \item Control en lazo cerrado obligatorio
    \item Componentes con certificaciones médicas apropiadas
\end{itemize}

\subsection{Tendencias y Tecnologías Emergentes}

\subsubsection{Motores Paso a Paso en Lazo Cerrado}

Los servomotores híbridos combinan las ventajas de motores paso a paso tradicionales con control en lazo cerrado:

\begin{itemize}
    \item Encoder integrado en el motor
    \item Drivers con control digital avanzado
    \item Eliminación completa de pérdida de pasos
    \item Optimización dinámica de corriente y eficiencia
    \item Auto-tuning de parámetros de control
\end{itemize}

\subsubsection{Comunicación Industrial}

Integración con protocolos de comunicación industrial:

\begin{itemize}
    \item EtherCAT: Comunicación en tiempo real con latencias < 100 $\mu$s
    \item CANopen: Comunicación robusta en ambientes industriales
    \item Modbus RTU/TCP: Integración con PLCs
    \item IO-Link: Configuración y diagnóstico simplificados
\end{itemize}

\subsubsection{Drivers Inteligentes}

Nueva generación de drivers con características avanzadas:

\begin{itemize}
    \item Detección automática de parámetros del motor
    \item Compensación de resonancia adaptativa
    \item Monitoreo predictivo de mantenimiento
    \item Perfiles de movimiento programables internamente
    \item Interfaces de configuración gráficas
\end{itemize}

\subsection{Resumen de Especificaciones Clave}

\begin{table}[ht]
\centering
\caption{Comparación de tamaños NEMA comunes}
\begin{tabular}{|c|c|c|c|c|}
\hline
\textbf{NEMA} & \textbf{Holding Torque} & \textbf{Corriente} & \textbf{Inercia} & \textbf{Aplicaciones} \\
\textbf{Size} & \textbf{(N·m)} & \textbf{(A)} & \textbf{(kg·m²)} & \textbf{Típicas} \\
\hline
11 & 0.08-0.15 & 0.5-1.0 & $5 \times 10^{-7}$ & Micro-posicionamiento \\
\hline
14 & 0.15-0.25 & 0.8-1.5 & $1 \times 10^{-6}$ & Cámaras, óptica \\
\hline
17 & 0.20-0.55 & 1.0-2.0 & $3 \times 10^{-6}$ & Impresoras 3D \\
\hline
23 & 0.50-3.00 & 1.5-4.0 & $5 \times 10^{-5}$ & CNC, automatización \\
\hline
34 & 3.00-12.0 & 3.0-8.0 & $5 \times 10^{-4}$ & Robótica, industrial \\
\hline
42 & 12.0-40.0 & 5.0-15.0 & $3 \times 10^{-3}$ & Alta carga \\
\hline
\end{tabular}
\end{table}

Los motores paso a paso continúan siendo una solución óptima para aplicaciones que requieren posicionamiento preciso, control digital directo, y operación confiable sin sistemas de retroalimentación complejos. La selección adecuada mediante un proceso sistemático, combinada con configuración apropiada del driver y técnicas de control avanzadas, garantiza un desempeño óptimo y vida útil prolongada del sistema.