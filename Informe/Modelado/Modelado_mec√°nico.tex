Se cuenta con dos movimientos principales, horizontal y vertical. El movimiento horizontal se genera mediante un sistema de transmisión por correa dentada y poleas, accionado por dos motores paso a paso. La estructura se desplaza sobre un perfil rígido y dos carros con rodamientos, que soportan la carga y aseguran un movimiento estable y preciso. Este mecanismo permite posicionar el carro en el eje horizontal y, en consecuencia, trasladar la estructura vertical junto con el brazo robótico.\\
El movimiento vertical se basa en un mecanismo de varilla roscada trapezoidal con tuerca, de funcionamiento similar al empleado en impresoras 3D. El giro del motor paso a paso, acoplado a la varilla, se transforma en un desplazamiento lineal vertical. Para garantizar suavidad y evitar bloqueos, el sistema incorpora rodamientos lineales que guían el movimiento sobre dos varillas trafiladas de soporte. Asimismo, la varilla roscada se encuentra montada sobre rodamientos en sus extremos, lo que mejora la distribución de la carga y aumenta la vida útil del mecanismo.\\
Además, se cuenta con un brazo mecánico controlado por servomotores, el mismo tiene 3 grados de libertad. \textbf{CHEQUEAR}
\textbf{PONER FOTOS DE LOS MOVIMIENTOS- carritos, caño y motores, varillas y motor, foto de servo y sistema de transmisión de las art}\\
Se añadió para limitar el movimiento pendular de la estructura, una varilla trefilada guiada por rodamientos lineales unidos al soporte superior. \textbf{esto capaz se puede poner como un problema que se solucionó en otra sección, ya vemos}\\