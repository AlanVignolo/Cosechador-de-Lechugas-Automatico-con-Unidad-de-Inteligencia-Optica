\subsection{Costos de desarrollo}

El desarrollo del proyecto requiere aproximadamente 1000 horas de ingeniería, distribuidas en las siguientes actividades:

\begin{itemize}
    \item Investigación y selección de componentes y materiales
    \item Diseño mecánico y modelado 3D
    \item Montaje y ensamblaje de la estructura
    \item Desarrollo de software y programación de controladores
    \item Integración del sistema de visión e inteligencia artificial
    \item Pruebas de funcionamiento y ajustes
    \item Elaboración de la documentación técnica
\end{itemize}

Considerando el costo de las horas de ingeniería junto con el costo de los materiales empleados (componentes electrónicos, motores, estructura metálica, impresiones 3D, sistema de visión, entre otros), el desarrollo del proyecto completo asciende aproximadamente a \$6.200.000 (ARS).

\subsection{Análisis de retorno de inversión}

Para evaluar la viabilidad económica del proyecto, se realizó un análisis del período de recupero de la inversión. Considerando las siguientes premisas:

\begin{itemize}
    \item Precio de venta promedio de lechugas: \$2.500 por unidad
    \item Capacidad de cosecha mensual: entre 55 y 60 lechugas
    \item Ingreso mensual estimado: \$137.500 a \$150.000
    \item Ingreso anual estimado: \$1.650.000 a \$1.800.000
\end{itemize}

Se estima un período de recupero de la inversión de aproximadamente 4 años, tiempo en el cual el sistema habría generado ingresos suficientes para cubrir los costos de desarrollo.

Es importante destacar que este análisis no contempla costos de mantenimiento, consumo energético ni potenciales mejoras o actualizaciones del sistema, factores que deberían considerarse en un análisis económico más detallado para una implementación a escala productiva.
