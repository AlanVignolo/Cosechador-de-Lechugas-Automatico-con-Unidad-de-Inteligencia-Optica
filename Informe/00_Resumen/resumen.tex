El presente trabajo describe el diseño, desarrollo e implementación de un prototipo funcional de robot cosechador automático de lechugas para sistemas de cultivo hidropónico vertical. El proyecto surge como respuesta a la necesidad de superar las limitaciones operativas y de seguridad asociadas a la cosecha manual en estructuras verticales de difícil acceso, proponiendo una solución que integra visión artificial, control jerárquico y mecatrónica. El sistema desarrollado emplea una configuración híbrida que combina una estructura cartesiana de dos ejes (horizontal X y vertical Z) con un brazo robótico serie de dos grados de libertad. La estructura cartesiana, compuesta por un perfil superior con transmisión por correa dentada para el eje horizontal y un sistema de tornillo de potencia autofrenante para el eje vertical, proporciona el posicionamiento bidimensional frente a los cultivos. El brazo serie, equipado con un efector final tipo pinza, ejecuta las operaciones de aproximación, toma y retracción de las lechugas. La arquitectura de control implementa un esquema jerárquico de dos niveles claramente diferenciados: el nivel regulatorio, ejecutado en Arduino Mega 2560, gestiona el control de bajo nivel de motores paso a paso, servomotores y sensores de seguridad; mientras que el nivel supervisor, implementado en Raspberry Pi 5, coordina la toma de decisiones de alto nivel, la secuenciación de operaciones y la integración con el sistema de visión artificial. La comunicación entre niveles se realiza mediante protocolo UART con estructura de comandos robusta y mecanismos de detección de errores. El sistema de inteligencia artificial comprende tres módulos especializados de procesamiento de imágenes: un detector de marcadores de referencia basado en segmentación por color que permite la calibración espacial inicial, un detector de tubos de cultivo mediante procesamiento morfológico que identifica las posiciones de los contenedores hidropónicos, y un clasificador estadístico de madurez de cultivos basado en análisis de contornos que determina qué lechugas están listas para cosecha. El clasificador alcanzó una separabilidad entre clases excepcional con una exactitud del 97\% en pruebas de validación. La validación experimental del prototipo confirmó el cumplimiento de los objetivos establecidos. Las pruebas mecánicas demostraron sincronización efectiva entre motores acoplados, ausencia de errores acumulativos en posicionamiento, y funcionalidad del sistema de autobloqueo vertical. Las pruebas de visión artificial lograron tasas de detección superiores al 90\% para marcadores, tubos y lechugas bajo condiciones controladas, con consistencia del 88\% en 500 detecciones repetidas. Las pruebas de control mostraron precisión de posicionamiento del 1\% en lazo abierto y una tasa de agarres exitosos del 82\% con integración de realimentación visual. El análisis de resultados identificó aspectos transferibles al sistema a escala real y limitaciones inherentes al prototipo de validación. Los algoritmos de visión artificial y el software de control demostró ser escalable, requiriendo únicamente recalibración paramétrica para adaptarse a sistemas de mayor envergadura. Las limitaciones mecánicas identificadas se relacionan directamente con la naturaleza de prototipo académico y se proyecta su eliminación mediante el uso de materiales estructurales de mayor rigidez, sistemas de guiado de menor fricción y dimensionamiento adecuado de componentes en la implementación industrial. El proyecto demuestra la viabilidad técnica de la automatización de cosecha en agricultura hidropónica vertical mediante la integración efectiva de visión artificial, control coordinado multi-eje y arquitecturas de control jerárquico, estableciendo fundamentos sólidos para el desarrollo de sistemas de agricultura urbana automatizada de mayor escala.