El presente trabajo describe el diseño y desarrollo de un robot cosechador automático de lechugas para sistemas de cultivo hidropónico vertical. El proyecto surge como respuesta a la necesidad de automatizar la cosecha en estructuras verticales de difícil acceso, integrando visión artificial, control jerárquico y mecatrónica.

El desarrollo abarca dos etapas complementarias: un análisis teórico que fundamenta el diseño del sistema a escala real, y la implementación de un prototipo funcional para validación experimental. El análisis teórico incluye el dimensionamiento mecánico de estructura cartesiana y brazo robótico, el diseño de la arquitectura de control jerárquica de dos niveles, y el desarrollo de algoritmos de visión artificial para detección y clasificación de cultivos.

El prototipo implementado emplea una configuración híbrida con estructura cartesiana de dos ejes para posicionamiento bidimensional y brazo serie con pinza para manipulación. El nivel regulatorio gestiona actuadores y sensores, mientras el nivel supervisor coordina las operaciones y la visión artificial.

La validación experimental evaluó distintas áreas del sistema. Las pruebas mecánicas verificaron sincronización de motores, precisión de posicionamiento y velocidades de desplazamiento. Las pruebas de visión artificial validaron los algoritmos de detección bajo condiciones controladas. Las pruebas integradas evaluaron el ciclo completo de cosecha, confirmando la viabilidad del enfoque propuesto.