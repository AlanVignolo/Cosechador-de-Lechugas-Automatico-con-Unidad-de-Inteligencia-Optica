\subsection{Análisis energético y dimensionamiento de fuente}

La Tabla \ref{tab:consumo_energetico} presenta el consumo eléctrico de cada componente del sistema en diferentes estados operativos.

\begin{table}[H]
\centering
\small
\begin{tabular}{|l|c|c|c|c|}
\hline
\textbf{Componente} & \textbf{Voltaje} & \textbf{Reposo (A)} & \textbf{Operación (A)} & \textbf{Pico (A)} \\
\hline
\multicolumn{5}{|c|}{\textbf{Alimentación 24V}} \\
\hline
Driver TB6600 ($\times$2, NEMA 17) & 24V & 2.0 & 6.0 & 6.0 \\
\hline
Driver KTC-DM860H-0 ($\times$1, NEMA 34) & 24V & 2.5 & 6.0 & 6.0 \\
\hline
Regulador 24V$\rightarrow$5V (5A)& 24V & 0.35 & 0.89 & 1.16 \\
\hline
Regulador 24V$\rightarrow$12V (10A)& 24V & 0.17 & 5.5 & 5.5 \\
\hline
\multicolumn{2}{|l|}{\textbf{Subtotal 24V}} & \textbf{5.02} & \textbf{18.39} & \textbf{18.66} \\
\hline
\hline
\multicolumn{5}{|c|}{\textbf{Alimentación 5V (desde regulador 24V$\rightarrow$5V)}} \\
\hline
Arduino Mega 2560 & 5V & 0.2 & 0.35 & 0.5 \\
\hline
Raspberry Pi 5 & 5V & 1.0 & 3.0 & 4.0 \\
\hline
Cámara USB & 5V & 0.3 & 0.5 & 0.5 \\
\hline
\multicolumn{2}{|l|}{\textbf{Subtotal 5V}} & \textbf{1.5} & \textbf{3.85} & \textbf{5.0} \\
\hline
\hline
\multicolumn{5}{|c|}{\textbf{Alimentación 12V (desde regulador 24V$\rightarrow$12V)}} \\
\hline
Servomotores Dynamixel MX-106T/R ($\times$2) & 12V & 0.0 & 9.6 & 9.6 \\
\hline
Lógica drivers ($\times$3) & 12V & 0.3 & 0.3 & 0.3 \\
\hline
\multicolumn{2}{|l|}{\textbf{Subtotal 12V}} & \textbf{0.3} & \textbf{9.9} & \textbf{9.9} \\
\hline
\hline
\multicolumn{5}{|c|}{\textbf{Consumo total en fuente de 24V}} \\
\hline
\multicolumn{2}{|l|}{Idle} & \multicolumn{3}{c|}{5.02 A @ 24V = \textbf{120 W}} \\
\hline
\multicolumn{2}{|l|}{Operación normal} & \multicolumn{3}{c|}{18.39 A @ 24V = \textbf{441 W}} \\
\hline
\multicolumn{2}{|l|}{Pico máximo} & \multicolumn{3}{c|}{18.66 A @ 24V = \textbf{448 W}} \\
\hline
\end{tabular}
\caption{\textit{Consumo eléctrico del sistema por componente y estado operativo}}
\label{tab:consumo_energetico}
\end{table}
%\footnotetext[1]{Consumo calculado considerando eficiencia del 90\% según carga real en cada estado}

El consumo en operación normal (441 W) ocurre cuando los motores paso a paso están en movimiento continuo, los servomotores Dynamixel operan bajo carga, y el sistema de visión está procesando imágenes simultáneamente.

Los valores de consumo de los drivers consideran que los motores NEMA 17 y NEMA 34 alcanzan su máxima demanda de corriente durante operación a velocidad máxima. Específicamente, el motor NEMA 34 Medium del eje vertical, con su mayor masa inercial y corriente nominal de 4.5 A, presenta el consumo más elevado del sistema (6 A del driver KTC-DM860H-0 a máxima velocidad, comparado con 2.5 A en reposo).

Los picos de 448 W ocurren durante aceleraciones bruscas con los tres motores paso a paso y los dos servomotores iniciando movimiento simultáneamente.

Se selecciona una fuente conmutada de 24V DC con capacidad nominal de 20 A (480 W), proporcionando un margen de seguridad del 7.1\% sobre el pico máximo calculado. Este margen considera:

\begin{itemize}[label=$\bullet$]
\item Degradación de la fuente con el tiempo (típicamente 10-15\% de pérdida de capacidad tras 2 años de operación continua)
\item Transitorios de corriente durante arranque inicial del sistema
\item Posibles expansiones futuras del hardware (sensores adicionales, iluminación LED, etc.)
\end{itemize}

La eficiencia de los reguladores step-down (90\% para 24V$\rightarrow$5V y 90\% para 24V$\rightarrow$12V) ya está considerada en los cálculos de consumo de la fuente primaria. Las pérdidas totales por conversión de voltaje son de aproximadamente 15.3 W en operación normal (2.1 W en el regulador 24V$\rightarrow$5V y 13.2 W en el regulador 24V$\rightarrow$12V).
