\subsection{Microcomputador: Raspberry Pi 5}
\label{sec:Raspberry}
El nivel supervisor se implementa sobre una Raspberry Pi 5 con procesador ARM Cortex-A76 quad-core a 2.4 GHz y 8 GB de memoria RAM LPDDR4X. La selección se justifica por:

La cámara seleccionada requiere interfaz USB 3.0 para transmitir video Full HD . La Raspberry Pi 5 incorpora 2 puertos USB 3.0 nativos que proporcionan hasta 5 Gbps por puerto, garantizando ancho de banda suficiente para captura de imágenes de alta resolución en tiempo real. El sistema operativo Raspberry Pi OS proporciona soporte nativo para el controlador V4L2 (Video4Linux2), permitiendo integración directa con bibliotecas de procesamiento de imágenes sin desarrollo de drivers adicionales.

Los algoritmos de visión por computadora requieren operaciones matriciales intensivas sobre imágenes de 1920×1080 píxeles. El procesador quad-core a 2.4 GHz permite ejecutar el proceso completo de clasificación en tiempo real.

 Python 3.11 con bibliotecas científicas maduras (NumPy, OpenCV, PySerial) permite desarrollo ágil de algoritmos de visión. La compatibilidad con OpenCV 4.x está completamente validada en la plataforma, simplificando la implementación de pipelines de procesamiento de imágenes. El sistema operativo Linux proporciona capacidades multitarea para ejecución concurrente de procesos de control, visión, y comunicación.

La Raspberry Pi 5 opera a 5V DC mediante conector USB-C, con consumo típico de 3 A (15 W) durante procesamiento intensivo de imágenes, y mínimo de 1 A (5 W) en estado idle.
