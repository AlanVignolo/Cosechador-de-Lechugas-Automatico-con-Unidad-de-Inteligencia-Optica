\subsection{Cámara USB de visión}

Se implementó una cámara USB 3.0 con sensor de 2 MP, lente ojo de pez y sensor Sony IMX291. Las características principales son:

\textbf{Resolución.} El sensor proporciona imágenes de 1920×1080 píxeles (Full HD), resolución suficiente para detectar lechugas de 8-12 cm de diámetro a distancias de 15-25 cm con al menos 120 píxeles de ancho, permitiendo clasificación confiable por área.

\textbf{Lente ojo de pez.} La distorsión radial del lente gran angular proporciona campo de visión amplio (aproximadamente 120°) a distancias cortas, permitiendo capturar el área completa de una estación de cultivo (15×15 cm) desde 20 cm de distancia. Aunque introduce distorsión geométrica, el clasificador morfológico es robusto ante estas deformaciones dado que opera sobre área de contornos en píxeles.

\textbf{Sensibilidad.} El sensor Sony IMX291 presenta sensibilidad adecuada para las condiciones de iluminación del invernadero (200-800 lux), con rango dinámico suficiente para manejar variaciones de luz natural.

\textbf{Interfaz USB 3.0.} La conectividad USB 3.0 proporciona ancho de banda de hasta 5 Gbps, suficiente para transmitir video Full HD a 30 fps sin compresión. El protocolo USB garantiza robustez ante interferencia electromagnética generada por motores y drivers.

El consumo de la cámara es de aproximadamente 500 mA a 5V (2.5 W) durante captura continua de video.
