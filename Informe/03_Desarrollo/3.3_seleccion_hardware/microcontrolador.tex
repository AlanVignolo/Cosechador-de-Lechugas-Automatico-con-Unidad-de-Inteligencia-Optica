\subsection{Microcontrolador: Arduino Mega 2560}

El controlador de bajo nivel seleccionado es un Arduino Mega 2560 basado en el microcontrolador ATmega2560, que opera a 16 MHz con arquitectura de 8 bits. La selección se fundamenta en los siguientes criterios técnicos:

Los drivers TB6600 seleccionados requieren señales lógicas de 5V para las entradas STEP, DIR y ENABLE. El Arduino Mega 2560 proporciona salidas digitales de 5V nativas, asegurando compatibilidad directa sin necesidad de circuitos de adaptación de nivel lógico. 

El Arduino Mega proporciona 54 pines digitales (de los cuales 15 soportan PWM) y 16 entradas analógicas, superando ampliamente los 17 pines digitales requeridos.

La generación de perfiles de velocidad trapezoidales requiere interrupciones de alta frecuencia (hasta 15 kHz) para emisión de pulsos STEP con precisión temporal. El ATmega2560 incorpora 6 timers hardware (cuatro de 8 bits y dos de 16 bits) que permiten generar señales PWM y manejar interrupciones sin sobrecarga del procesador principal.

El microcontrolador dispone de 4 puertos UART hardware, permitiendo comunicación con el nivel supervisor a 115200 baudios sin consumo de ciclos de CPU para manejo de protocolo.

El Arduino Mega 2560 opera a 5V DC con consumo típico de 200 mA durante operación normal, alcanzando picos de 500 mA durante conmutación simultánea de múltiples salidas digitales.