El modelado tridimensional de los componentes estructurales del sistema se realizó mediante software CAD, permitiendo la visualización detallada de cada elemento, la verificación de interferencias geométricas y el análisis de esfuerzos mediante simulación por elementos finitos. En esta sección se presentan los tres soportes estructurales principales fabricados mediante impresión 3D, junto con el brazo robótico y su efector final. Se incluye además el análisis de esfuerzos del soporte superior, que constituye el elemento más crítico desde el punto de vista de cargas estructurales.

\subsubsection{Soporte Superior}

El soporte superior constituye el elemento estructural de mayor criticidad en el sistema, ya que soporta la totalidad de las cargas verticales (varillas, brazo, carga) y actúa como interfaz mecánica entre los subsistemas de movimiento horizontal y vertical. Su diseño integra múltiples funciones: acoplamiento a los carros deslizantes del perfil horizontal, fijación de los extremos superiores de las tres varillas verticales (una roscada y dos lisas), y montaje del motor del eje vertical.

La Figura \ref{fig:SuperiorReal_simplificado_vista} muestra la geometría general del soporte superior.

Para el análisis por elementos finitos se consideró la condición de carga más desfavorable: sistema completamente extendido con carga máxima en el extremo del brazo. La carga total aplicada sobre el soporte superior se compone de:

\begin{itemize}[label=$\bullet$]
    \item Peso de las tres varillas verticales (dos lisas de acero + una roscada): $\approx$ 3.8 kg
    \item Peso del soporte medio con brazo y cámara: $\approx$ 1.5 kg
    \item Peso del motor vertical: $\approx$ 2.2 kg
    \item Carga útil (lechuga) en extremo del brazo: $\approx$ 0.4 kg
    \item Peso propio del soporte superior: $\approx$ 1.1 kg
\end{itemize}

La carga total considerada para el análisis fue de 9 kg ($\approx$ 88 N), aplicada como fuerzas distribuidas sobre las superficies de contacto con las varillas y reacciones de apoyo en las zonas de montaje a los carros, como se muestra en la Figura \ref{fig:SuperiorReal_simplificado_fuerzas_app}.

\begin{figure}[H]
    \centering
    \begin{subfigure}{0.35\textwidth}
        \centering
        \includegraphics[width=0.7\textwidth]{img/SuperiorReal_simplificado_vista.jpg}
        \caption{\textit{Vista del soporte superior mostrando la configuración estructural, alojamientos de varillas y zonas de montaje.}}
        \label{fig:SuperiorReal_simplificado_vista}
    \end{subfigure}
    \hspace{0.5cm}
    \begin{subfigure}{0.35\textwidth}
        \centering
        \includegraphics[width=0.7\textwidth]{img/SuperiorReal_simplificado_fuerzas_app.jpg}
        \caption{\textit{Distribución de cargas aplicadas sobre el soporte superior.}}
        \label{fig:SuperiorReal_simplificado_fuerzas_app}
    \end{subfigure}
    \caption{\textit{Soporte superior. Vista y fuerzas aplicadas}}
\end{figure}

El análisis de esfuerzos mediante elementos finitos se realizó considerando material PLA con las siguientes propiedades mecánicas:

\begin{itemize}[label=$\bullet$]
    \item Módulo de Young: 3500 MPa
    \item Límite de fluencia: 50 MPa
    \item Coeficiente de Poisson: 0.36
\end{itemize}

La Figura \ref{fig:SuperiorReal_simplificado_tensiones} presenta los resultados del análisis, mostrando las deformaciones resultantes. Los resultados indican:

\begin{itemize}[label=$\bullet$]
    \item Deformación máxima: $\approx$ 0.15 mm (en zonas de mayor voladizo)
    \item Factor de seguridad: $\frac{50\,\text{MPa}}{2.8\,\text{MPa}} \approx 18$ (ampliamente seguro)
\end{itemize}

\begin{figure}[H]
    \centering
    \begin{subfigure}{0.48\textwidth}
        \centering
    \includegraphics[width=0.78\textwidth]{img/tensiones_Real.jpg}
    \caption{Distribución de tensiones en el soporte superior bajo carga máxima.}
    \label{fig:tensiones_Real}
    \end{subfigure}
    \hspace{0.5cm}
    \begin{subfigure}{0.48\textwidth}
        \centering
    \includegraphics[width=0.9\textwidth]{img/SuperiorReal_simplificado_tensiones.jpg}
    \caption{Distribución de deformaciones en el soporte superior bajo carga máxima.}
    \label{fig:SuperiorReal_simplificado_tensiones}
    \end{subfigure}
    \caption{\textit{Tensiones y deformaciones del soporte superior.}}
\end{figure}
Los resultados confirman que el diseño propuesto es estructuralmente robusto, con tensiones muy por debajo del límite de fluencia del material y deformaciones despreciables para los requerimientos operacionales del sistema.

\subsubsection{Soporte Medio}

El soporte medio constituye el elemento móvil del eje vertical, desplazándose a lo largo de las varillas mediante el giro de la varilla roscada. Este componente cumple funciones críticas en el sistema: montaje del brazo robótico serial, fijación de la cámara de visión artificial y mantenimiento del equilibrio dinámico mediante contrapeso.

\begin{itemize}[label=$\bullet$]
    \item \underline{Sistema de guiado:} Incorpora dos rodamientos lineales LM16UU para deslizamiento sobre las varillas lisas de 8 mm, garantizando movimiento suave sin rotaciones indeseadas
    \item \underline{Tuerca de transmisión:} Alojamiento central para tuerca TR16 que se acopla con la varilla roscada, convirtiendo el movimiento rotacional del motor en desplazamiento lineal
    \item \underline{Montaje del brazo:} Base horizontal plana con patrón de agujeros para fijación del servomotor base del brazo robótico
    \item \underline{Montaje de cámara:} Soporte superior para la cámara con ajuste angular para optimización del campo de visión
    \item \underline{Sistema de balanceo:} Contrapeso en el lado opuesto al brazo para reducir momentos flectores sobre las varillas y mejorar la estabilidad dinámica del sistema
\end{itemize}

\begin{figure}[H]
    \centering
    \includegraphics[width=0.5\textwidth]{img/MedioReal_simplificado_vista.jpg}
    \caption{Vista general del soporte medio mostrando el montaje del brazo robótico, alojamiento de la cámara, sistema de guiado lineal y contrapeso de equilibrio.}
    \label{fig:soporte_medio_Real}
\end{figure}

El contrapeso permite compensar el momento generado por el brazo extendido, reduciendo las cargas laterales sobre las guías lineales y mejorando la precisión de posicionamiento vertical. El diseño modular facilita el reemplazo del brazo o la cámara sin necesidad de rediseñar el soporte completo.

\subsubsection{Soporte Inferior}

El soporte inferior cierra la estructura vertical del sistema, proporcionando la base de fijación para los extremos inferiores de las tres varillas y alojando componentes auxiliares del sistema de control.\\

Funciones y características

\begin{itemize}[label=$\bullet$]
    \item \underline{Anclaje de varillas:} Alojamientos pasantes para las dos varillas lisas con fijación mediante tornillos prisioneros, y soporte del extremo libre de la varilla roscada con rodamiento axial
    \item \underline{Final de carrera inferior:} Montaje del microswitch de límite inferior del eje vertical, con placa de activación mecánica
\end{itemize}

\begin{figure}[H]
    \centering
    \includegraphics[width=0.5\textwidth]{img/InferiorReal_simplificado_vista.jpg}
    \caption{\textit{Soporte inferior, se muestra el alojamiento del final de carrera inferior.}}
    \label{fig:soporte_inferior_real}
\end{figure}

El diseño incorpora tolerancias ajustadas para los alojamientos de las varillas, garantizando la perpendicularidad del eje vertical respecto al plano horizontal de desplazamiento.

\subsubsection{Brazo robótico serie}

Se opta por el uso de PLA con fibra de carbono como material estructural proporciona mejor relación rigidez/peso que el PLA estándar, reduciendo las vibraciones residuales al finalizar movimientos rápidos y mejorando la precisión de posicionamiento del efector final.

\begin{figure}[H]
    \centering
    \includegraphics[width=0.5\textwidth]{img/brazo_completo.jpg}
    \caption{\textit{Vista completa del brazo robótico de 2 grados de libertad mostrando las dos articulaciones rotacionales y el efector final tipo pinza.}}
    \label{fig:brazo_Real}
\end{figure}

\subsubsection{Efector final}

El efector final implementa un mecanismo de pinza paralela actuado mediante transmisión piñón-cremallera, convirtiendo el movimiento rotacional del servomotor en desplazamiento lineal de las mordazas.\\

Principio de funcionamiento

\begin{itemize}[label=$\bullet$]
    \item \underline{Transmisión:} Sistema piñón-cremallera con relación 1:1, proporcionando movimiento sincronizado de ambas mordazas hacia el centro
    \item \underline{Rango de apertura:} 0-50 mm, adecuado para lechugas de tamaño pequeño a grande
    \item \underline{Mordazas:} Geometría curva con superficie texturizada para mejorar fricción sin penetrar las hojas
\end{itemize}

\begin{figure}[H]
    \centering
    \begin{subfigure}{0.35\textwidth}
        \centering
        \includegraphics[width=0.7\textwidth]{img/gripper.jpg}
        \caption{\textit{Sistema de transmisión piñón-cremallera del gripper mostrando el acoplamiento mecánico entre el servomotor y las mordazas móviles.}}
        \label{fig:gripper_Real_transmision}
    \end{subfigure}
    \hspace{0.5cm}
    \begin{subfigure}{0.35\textwidth}
        \centering
        \includegraphics[width=0.7\textwidth]{img/pinza_tapa.png}
        \caption{\textit{Vista del gripper ensamblado mostrando la configuración de mordazas paralelas y carcasa de protección.}}
        \label{fig:gripper_Real}
    \end{subfigure}
    \caption{\textit{Efector final tipo gripper con transmisión por piñón-cremallera para agarre de lechugas.}}
\end{figure}

