\subsubsection{Selección del perfil de movimiento horizontal}
El perfil horizontal soporta la mayor parte de la carga total del sistema, compuesta por el peso de la lechuga, las varillas (lisas y roscadas), y los soportes medio e inferior.

Se selecciona un perfil de aluminio V-Slot con ruedas de precisión, que ofrece:
\begin{itemize}
    \item Capacidad de carga de hasta 50\,kg
    \item Sistema de ajuste para eliminación de juego mecánico
    \item Alta precisión de movimiento
\end{itemize}

\paragraph{Cálculo de deflexión}
Para verificar la rigidez del perfil, se analiza como una viga simplemente apoyada bajo carga centrada:
\begin{equation}
\delta = \frac{F \cdot L^3}{48 \cdot E \cdot I}
\label{eq:deflexion_perfil}
\end{equation}

donde:
\begin{itemize}
    \item $E = 69\,000$\,N/mm$^2$ (módulo de Young del aluminio)
    \item $L = 3\,000$\,mm (longitud del perfil)
    \item $I$ = momento de inercia del perfil
    \item $F$ = carga aplicada
\end{itemize}

Para un perfil V-Slot 20$\times$80\,mm, con $I = 350\,000$\,mm$^4$, la deflexión calculada es:
\[\delta = 0.87\,\text{mm}\]

Este valor es aceptable considerando que representa menos del 0.03\% de la longitud total del perfil, cumpliendo con criterios de rigidez para aplicaciones de precisión moderada.

Como complemento, se seleccionan dos carros lineales estándar de 100\,mm de ancho, cada uno equipado con 4 ruedas tipo DELRIN que se desplazan sobre el perfil V-Slot.
\begin{figure}[H]
    \centering
    \begin{subfigure}{0.4\textwidth}
        \centering
        \includegraphics[width=\textwidth]{img/vslot_40x80.png}
        \caption{Perfil v-slot.}
        \label{fig:vslot_40x80}
    \end{subfigure}
    \hfill
    \begin{subfigure}{0.4\textwidth}
        \centering
        \includegraphics[width=\textwidth]{img/carro_perfilvslot.png}
        \caption{Sistema de ruedas para deslizamiento sobre perfil v-slot.}
        \label{fig:carro_perfilvslot}
    \end{subfigure}
    \caption{Sistema de deslizamiento con perfil v-slot.}
\end{figure}

\subsubsection{Sistema de transmisión por correa}
El sistema utiliza correa dentada GT2 con poleas de igual diámetro en ambos extremos. La longitud de correa necesaria se calcula como:
\begin{equation}
L = 2C + \pi D = 6.06\,\text{m}
\label{eq:longitud_correa}
\end{equation}
donde $C$ es la distancia entre centros de las poleas y $D$ el diámetro de las mismas.

\paragraph{Distribución de cargas en el sistema}
La carga total de 12.8\,kg se distribuye entre dos subsistemas mecánicos:

\begin{itemize}
    \item \textbf{Carga soportada por el perfil V-Slot ($\approx$70\%):} El perfil y los carros lineales soportan directamente el peso vertical de la carga (8.96\,kg) a través de las ruedas DELRIN. Esta carga genera únicamente esfuerzos de compresión y flexión en el perfil, sin afectar directamente al sistema de transmisión.
    
    \item \textbf{Carga dinámica del motor ($\approx$30\%):} El motor debe vencer las fuerzas necesarias para acelerar y mover horizontalmente la masa efectiva del sistema (3.84\,kg). Esta masa efectiva es menor que la carga total debido a que:
    \begin{enumerate}
        \item Las ruedas del carro transforman el rozamiento estático en rodadura, reduciendo significativamente la resistencia al movimiento
        \item El perfil V-Slot absorbe las cargas verticales, dejando al motor únicamente las fuerzas inerciales horizontales y las fricciones residuales del sistema
    \end{enumerate}
\end{itemize}

Esta distribución asume un coeficiente de fricción por rodadura de $\mu_{\text{rod}} \approx 0.001$ para ruedas DELRIN sobre aluminio, lo que implica que la fuerza horizontal necesaria es aproximadamente el 30\% del peso total.

\paragraph{Verificación del porcentaje adoptado}
Para validar el porcentaje del 30\%, se analiza la relación entre la fuerza de fricción por rodadura y el peso total:

\[\frac{F_{\text{fricción}}}{W_{\text{total}}} = \mu_{\text{rod}} = \frac{F_{\text{rod}}}{m \cdot g}\]
Con un coeficiente de fricción por rodadura típico de $\mu_{\text{rod}} = 0.001$ (DELRIN sobre aluminio):

\[F_{\text{rod}} = 0.001 \cdot 12.8\,\text{kg} \cdot 9.81\,\text{m/s}^2 \approx 0.125\,\text{N}\]

Esta fuerza representa solo el 0.1\% de la carga. Sin embargo, al incluir la aceleración, fricciones de rodamientos, tensión de correa y un factor de seguridad, la fuerza total aumenta significativamente. El 30\% de la masa total es una estimación conservadora que incluye todos estos efectos y proporciona un margen adecuado para variaciones en las condiciones de operación.

\paragraph{Análisis de fuerzas}
Para determinar el torque motor requerido, se consideran las siguientes fuerzas resistivas:

\begin{enumerate}
    \item \textbf{Fricción por rodadura:}
    \begin{equation}
    F_{\text{rod}} = \mu_{\text{rodadura}} \cdot m \cdot g \cdot \frac{1}{r_{\text{rueda}}}
    \end{equation}
    donde $\mu_{\text{rodadura}}$ depende del material de las ruedas DELRIN sobre aluminio.
    
    \item \textbf{Fuerza de aceleración:}
    \begin{equation}
    F_{\text{acel}} = (m_{\text{carga}} + m_{\text{correa}}) \cdot a_{\text{max}}
    \end{equation}
    Incluye la masa de la carga útil y la masa lineal de la correa GT2.
    
    \item \textbf{Fricción en rodamientos de poleas:}
    \begin{equation}
    F_{\text{friccion\_rod}} = T_{\text{correa}} \cdot \mu_{\text{rodamientos}} \cdot n
    \end{equation}
    con $n = 2$ (polea conducida y tensor).
    
    \item \textbf{Fuerza interna de la correa:}
    \begin{equation}
    F_{\text{correa}} = 0.5\,\text{N}
    \end{equation}
    Valor empírico para correas GT2 según especificaciones del fabricante.
    
    \item \textbf{Otras pérdidas:}
    \begin{equation}
    F_{\text{otras}} = 0.1 \cdot (F_{\text{acel}} + F_{\text{rod}})
    \end{equation}
    Se estima un 10\% adicional para pérdidas no modeladas.
\end{enumerate}

\paragraph{Cálculo del torque motor}
Considerando la distribución de cargas analizada anteriormente, donde el motor debe mover efectivamente el 30\% de la masa total (3.84\,kg), la fuerza total requerida en la base del sistema es:
\begin{equation}
F_{\text{tbase}} = 1.46\,\text{N}
\end{equation}

Aplicando un factor de seguridad de 2.5 para contemplar condiciones adversas de operación y degradación del sistema con el uso, el torque motor necesario resulta:
\begin{equation}
T_{\text{motor}} = F_{\text{tbase}} \cdot 2.5 \cdot r_{\text{polea}} \approx 0.04\,\text{Nm}
\label{eq:torque_motor}
\end{equation}

\paragraph{Selección del motor}
Se selecciona un motor paso a paso NEMA 17. Analizando la curva dinámica torque-velocidad (Figura~\ref{fig:Curva_din_nema34}), se verifica que para la velocidad máxima de operación de 250\,mm/s (239\,rpm), el torque disponible supera ampliamente el requerimiento calculado de 0.04\,Nm, garantizando un margen de seguridad adecuado para la operación continua del sistema.

\begin{figure}[H]
    \centering
    \includegraphics[width=0.65\textwidth]{img/Nema17.png}
    \caption{Curva dinámica torque-velocidad del motor paso a paso NEMA 17.}
    \label{fig:Curva_din_nema17}
\end{figure}