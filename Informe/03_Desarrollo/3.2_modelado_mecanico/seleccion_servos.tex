\subsection{Dimensionamiento de Servomotores}

El brazo robótico diseñado para la cosecha de lechugas cuenta con dos grados de libertad, equivalentes a una articulación de hombro y una de codo. El sistema debe ser capaz de extenderse horizontalmente para alcanzar y tomar la carga (lechuga), para luego contraerse y transportarla de manera eficiente.

\subsubsection{Parámetros del Sistema}

Los parámetros físicos y operacionales del brazo robótico se detallan en la Tabla \ref{tab:parametros_brazo}.

\begin{table}[htbp]
\centering
\caption{Parámetros del brazo robótico}
\label{tab:parametros_brazo}
\begin{tabular}{lcc}
\hline
\textbf{Parámetro} & \textbf{Valor} & \textbf{Unidad} \\
\hline
Longitud eslabón 1 ($L_1$) & 16 & cm \\
Longitud eslabón 2 ($L_2$) & 12.5 & cm \\
Masa del brazo ($m_{brazo}$) & 300 & g \\
Masa de la carga ($m_{carga}$) & 400 & g \\
Velocidad angular máxima ($\omega_{\max}$) & 45 & $^{\circ}$/s \\
Tiempo de aceleración ($t_{ac}$) & 0.5 & s \\
\hline
\end{tabular}
\end{table}

El movimiento del brazo involucra tanto desplazamiento horizontal como elevación vertical, partiendo desde una posición horizontal hasta una posición elevada para transportar la carga. Por lo tanto, el análisis debe considerar tanto el torque dinámico necesario para acelerar las masas como el torque estático requerido para sostener el brazo contra la gravedad.

\subsubsection{Dimensionamiento del Motor del Codo}

El servomotor del codo es responsable del movimiento de la carga ubicada en el extremo del segundo eslabón y debe soportarla contra la gravedad.

\subsubsection{Torque Estático (Gravitacional)}

En la posición horizontal, el torque necesario para sostener la carga ubicada en el extremo del eslabón es:

\begin{equation}
\tau_{est,codo} = m_{carga} \cdot g \cdot L_2
\end{equation}

Sustituyendo valores:

\begin{equation}
\tau_{est,codo} = 0.4 \, \text{kg} \cdot 9.81 \, \text{m/s}^2 \cdot 0.125 \, \text{m} = 0.49 \, \text{N} \cdot \text{m}
\end{equation}

\subsubsection{Momento de Inercia}

El momento de inercia de la carga con respecto al eje del codo se calcula como:

\begin{equation}
I_{codo} = m_{carga} \cdot L_2^2
\end{equation}

Sustituyendo valores:

\begin{equation}
I_{codo} = 0.4 \, \text{kg} \cdot (0.125 \, \text{m})^2 = 6.25 \times 10^{-3} \, \text{kg} \cdot \text{m}^2
\end{equation}

\subsubsection{Aceleración Angular}

La aceleración angular necesaria para alcanzar la velocidad máxima en el tiempo especificado es:

\begin{equation}
\alpha = \frac{\omega_{max}}{t_{ac}} = \frac{0.785 \, \text{rad/s}}{0.5 \, \text{s}} = 1.57 \, \text{rad/s}^2
\end{equation}

donde $\omega_{max} = 45° \cdot \frac{\pi}{180°} = 0.785 \, \text{rad/s}$.

\subsubsection{Torque Dinámico}

El torque dinámico requerido se obtiene mediante:

\begin{equation}
\tau_{din,codo} = I_{codo} \cdot \alpha = 6.25 \times 10^{-3} \cdot 1.57 = 9.81 \times 10^{-3} \, \text{N} \cdot \text{m}
\end{equation}

\subsubsection{Torque Total}

El torque total necesario suma los componentes estático y dinámico:

\begin{equation}
\tau_{codo,total} = \tau_{est,codo} + \tau_{din,codo} = 0.49 + 0.00981 = 0.50 \, \text{N} \cdot \text{m}
\end{equation}

Aplicando un factor de seguridad de 1.5 para considerar fricciones, pérdidas y variaciones en la carga:

\begin{equation}
\tau_{codo} = 1.5 \cdot \tau_{codo,total} = 0.75 \, \text{N} \cdot \text{m} \approx 76 \, \text{kg} \cdot \text{cm}
\end{equation}

\subsubsection{Dimensionamiento del Motor del Hombro}

El servomotor del hombro debe proporcionar el torque necesario para mover el brazo completo, soportar su peso y el de la carga contra la gravedad, especialmente en la posición horizontal.

\subsubsection{Torque Estático (Gravitacional)}

En la posición horizontal (peor caso), el torque gravitacional resulta de la suma de los momentos de todas las masas:

\begin{equation}
\tau_{est,hombro} = m_{brazo} \cdot g \cdot \frac{L_1}{2} + m_{carga} \cdot g \cdot L_{total}
\end{equation}

Sustituyendo valores:

\begin{equation}
\tau_{est,hombro} = 0.3 \cdot 9.81 \cdot 0.08 + 0.4 \cdot 9.81 \cdot 0.285 = 0.235 + 1.118 = 1.353 \, \text{N} \cdot \text{m}
\end{equation}

\subsubsection{Momento de Inercia Total}

El momento de inercia total respecto al eje del hombro se compone de tres contribuciones:

\textbf{a) Inercia del primer eslabón:} Considerando una distribución uniforme de masa:

\begin{equation}
I_{eslabón1} = \frac{m_{brazo}}{2} \cdot \frac{L_1^2}{3} = 0.15 \cdot \frac{(0.16)^2}{3} = 1.28 \times 10^{-3} \, \text{kg} \cdot \text{m}^2
\end{equation}

\textbf{b) Inercia del segundo eslabón:} Aproximando su masa concentrada en el codo:

\begin{equation}
I_{eslabón2} = \frac{m_{brazo}}{2} \cdot L_1^2 = 0.15 \cdot (0.16)^2 = 3.84 * 10^{-3} \, \text{kg} \cdot \text{m}^2
\end{equation}

\textbf{c) Inercia de la carga:} Con el brazo completamente extendido ($L_{total} = L_1 + L_2 = 0.285$ m):

\begin{equation}
I_{carga,hombro} = m_{carga} \cdot L_{total}^2 = 0.4 \cdot (0.285)^2 = 3.25 \times 10^{-2} \, \text{kg} \cdot \text{m}^2
\end{equation}

El momento de inercia total resulta:

\begin{equation}
I_{hombro} = I_{eslabón1} + I_{eslabón2} + I_{carga,hombro} = 3.76 \times 10^{-2} \, \text{kg} \cdot \text{m}^2
\end{equation}

\subsubsection{Torque Dinámico}

Utilizando la misma aceleración angular calculada anteriormente:

\begin{equation}
\tau_{din,hombro} = I_{hombro} \cdot \alpha = 3.76 \times 10^{-2} \cdot 1.57 = 5.9 \times 10^{-2} \, \text{N} \cdot \text{m}
\end{equation}

\subsubsection{Torque Total}

El torque total combina los componentes estático y dinámico:

\begin{equation}
\tau_{hombro,total} = \tau_{est,hombro} + \tau_{din,hombro} = 1.353 + 0.059 = 1.412 \, \text{N} \cdot \text{m}
\end{equation}

Con factor de seguridad de 1.5:

\begin{equation}
\tau_{hombro} = 1.5 \cdot \tau_{hombro,total} = 2.12 \, \text{N} \cdot \text{m} \approx 216 \, \text{kg} \cdot \text{cm}
\end{equation}

\subsection{Resultados y Selección de Servomotores}

La Tabla \ref{tab:resultados_torque} resume los torques mínimos requeridos para cada articulación.

\begin{table}[h]
\centering
\caption{Torques mínimos requeridos}
\label{tab:resultados_torque}
\begin{tabular}{lccc}
\hline
\textbf{Articulación} & \textbf{Torque Estático} & \textbf{Torque Total} & \textbf{Con Seguridad} \\
 & \textbf{(kg·cm)} & \textbf{(kg·cm)} & \textbf{(kg·cm)} \\
\hline
Codo & 50 & 51 & 76 \\
Hombro & 138 & 144 & 216 \\
\hline
\end{tabular}
\end{table}

En base a estos requerimientos, se proponen las siguientes opciones de servomotores:

\textbf{Motor del Codo (76 kg·cm):}
\begin{itemize}
    \item Dynamixel MX-64: 60 kg·cm @ 12V (requiere reducción 1.3:1 o torque adicional)
    \item Dynamixel MX-106: 84 kg·cm @ 12V (opción recomendada, margen de seguridad)
    \item Servo industrial con reductor: torque nominal > 80 kg·cm
\end{itemize}

\textbf{Motor del Hombro (216 kg·cm):}
\begin{itemize}
    \item Dynamixel MX-106: 84 kg·cm @ 12V (requiere reducción 3:1)
    \item Servomotor industrial Nema 23 con reductor planetario
    \item Sistema de poleas/engranajes con motor de menor torque (relación 4:1 o 5:1)
\end{itemize}

Dada la magnitud del torque requerido, especialmente en el hombro, se recomienda implementar un sistema de reducción mecánica mediante poleas dentadas o reductores planetarios, lo que permitirá utilizar motores más económicos y con mejor respuesta dinámica.




