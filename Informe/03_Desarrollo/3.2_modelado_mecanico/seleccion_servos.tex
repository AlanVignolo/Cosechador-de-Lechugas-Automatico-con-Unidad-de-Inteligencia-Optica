\subsubsection{Dimensionamiento de servomotores}

Para el dimensionamiento de los servomotores se utilizan las expresiones de torque obtenidas mediante el análisis dinámico por Euler-Lagrange, evaluadas en las configuraciones críticas de operación del sistema.\\

Parámetros del sistema\\
\noindent
Los parámetros físicos y operacionales del brazo robótico se detallan en la Tabla \ref{tab:parametros_brazo}.

\begin{table}[htbp]
\centering
\begin{tabular}{lcc}
\hline
\textbf{Parámetro} & \textbf{Valor} & \textbf{Unidad} \\
\hline
Longitud total eslabón 1 ($L_1$) & 16 & cm \\
Longitud total eslabón 2 ($L_2$) & 12.5 & cm \\
Posición centro de masa eslabón 1 ($l_1$) & 16 & cm \\
Posición centro de masa eslabón 2 ($l_2$) & 12.5 & cm \\
Masa eslabón 1 ($m_1$) & 150 & g \\
Masa eslabón 2 + carga ($m_2$) & 550 & g \\
Momento de inercia eslabón 1 ($I_1$) & $3.2 \times 10^{-5}$ & kg·m$^2$ \\
Momento de inercia eslabón 2 ($I_2$) & $2.15 \times 10^{-5}$ & kg·m$^2$ \\
Velocidad angular máxima ($\omega_{\max}$) & 45 & $^{\circ}$/s \\
Tiempo de aceleración ($t_{ac}$) & 0.5 & s \\
\hline
\end{tabular}
\caption{\textit{Parámetros del brazo robótico}}
\label{tab:parametros_brazo}
\end{table}

Los momentos de inercia se calcularon considerando los eslabones como barras delgadas uniformes respecto a sus centros de masa: $I = \frac{1}{12}mL^2$.

Del análisis dinámico desarrollado previamente, se tienen las expresiones \ref{ec:torqueT1} (hombro) y \ref{ec:torqueT2}(codo) para los torques en cada articulación:

Estas ecuaciones capturan los efectos inerciales, de acoplamiento dinámico entre articulaciones, términos centrífugos y efectos gravitacionales. \\

Configuraciones críticas de análisis\\

Para el dimensionamiento se consideran tres configuraciones críticas que representan las condiciones de carga más exigentes.

Cálculo de torque para la articulación 2 (codo):
\begin{itemize}
    \item \underline{Caso 1: Posición horizontal en reposo}: $\theta_1 = 0°$, $\theta_2 = 0°$, $\dot{\theta}_1 = \dot{\theta}_2 = 0$, $\ddot{\theta}_1 = \ddot{\theta}_2 = 0$\\
     
        La expresión de $\tau_2$ se reduce al término gravitacional:
        \begin{equation}
        \tau_{2,est} = gm_2l_2\cos(\theta_2) = 9.81 \cdot 0.55 \cdot 0.125 \cdot \cos(0°) = 0.674 \, \text{N·m}
        \end{equation}
    \item  \underline{Caso 2: Aceleración máxima}: $\theta_1 = 0°$, $\theta_2 = 0°$, $\dot{\theta}_1 = \dot{\theta}_2 = 0$, $\ddot{\theta}_1 = \ddot{\theta}_2 = 1.57$ rad/s$^2$ \\
    
        \begin{equation}
        \tau_{2,din} = (m_2l_2^2 + I_2)\ddot{\theta}_2 + m_2L_1l_2\ddot{\theta}_1\cos(0) + gm_2l_2\cos(0)
        = 0.705 \, \text{N·m}
        \end{equation}
    \item \underline{Caso 3: Efectos centrífugos}: $\theta_1 = 45°$, $\theta_2 = 45°$, $\dot{\theta}_1 = 0.785$ rad/s, $\ddot{\theta}_1 = \ddot{\theta}_2 = 0$ \\
        
        \begin{equation}
        \tau_{2,cent} = m_2L_1l_2\dot{\theta}_1^2\sin(\theta_1 - \theta_2) + gm_2l_2\cos(\theta_2) = 0.477 \, \text{N·m}
        \end{equation}
\end{itemize}

El torque máximo requerido para el codo es:
\begin{equation}
\tau_{2,max} = \max(0.674, 0.705, 0.477) = 0.705 \, \text{N·m}
\end{equation}

Aplicando un factor de seguridad de 1.5:
\begin{equation}
\tau_{2,req} = 1.5 \cdot 0.705 = 1.058 \, \text{N·m}
\end{equation}
Cálculo de torque para la articulación 1 (hombro):
\begin{itemize}
    \item \underline{Caso 1: Posición horizontal en reposo}: $\theta_1 = 0°$, $\theta_2 = 0°$, $\dot{\theta}_1 = \dot{\theta}_2 = 0$, $\ddot{\theta}_1 = \ddot{\theta}_2 = 0$ \\
        
        \begin{equation}
        \tau_{1,est} = g(m_1l_1 + m_2L_1)\cos(\theta_1) 
        = 1.098 \, \text{N·m}
        \end{equation}

    \item \underline{Caso 2: Aceleración máxima}: $\theta_1 = 0°$, $\theta_2 = 0°$, $\dot{\theta}_1 = \dot{\theta}_2 = 0$, $\ddot{\theta}_1 = \ddot{\theta}_2 = 1.57$ rad/s$^2$\\
        
        \begin{equation}
        \tau_{1,din} = (m_1l_1^2 + m_2L_1^2 + I_1)\ddot{\theta}_1 + m_2L_1l_2\ddot{\theta}_2\cos(0) + g(m_1l_1 + m_2L_1)\cos(0) = 1.144 \, \text{N·m}
        \end{equation}

    \item \underline{Caso 3: Efectos centrífugos}: $\theta_1 = 0°$, $\theta_2 = 45°$, $\dot{\theta}_2 = 0.785$ rad/s, $\ddot{\theta}_1 = \ddot{\theta}_2 = 0$\\
        
        \begin{equation}
        \tau_{1,cent} = - m_2L_1l_2\dot{\theta}_2^2\sin(\theta_1 - \theta_2) + g(m_1l_1 + m_2L_1)\cos(\theta_1)
        = 1.103 \, \text{N·m}
        \end{equation}
\end{itemize}

El torque máximo requerido para el hombro es:

\begin{equation}
\mathbf{\tau_{1,max}} = \max(1.098, 1.144, 1.103) = \mathbf{1.144 \, \textbf{N·m} }
\end{equation}

Aplicando un factor de seguridad de 1.5:

\begin{equation}
\mathbf{\tau_{1,req}} = 1.5 \cdot 1.144 = \mathbf{1.716 \, \textbf{N·m} }
\end{equation}

Para ambos motores se selecciona un motor \textbf{Dynamixel MX-106T/R} que cuenta con un torque de 8N$\cdot$m a 12v y 5A, teniendo un margen de aproximadamente 8 veces para el codo y de aproximadamente 5 veces para el hombro.
