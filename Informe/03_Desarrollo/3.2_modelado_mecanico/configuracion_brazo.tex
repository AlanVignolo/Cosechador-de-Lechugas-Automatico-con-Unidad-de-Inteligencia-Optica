Se cuenta con un brazo de dos grados de libertad controlables para el posicionamiento del efector final que es de tipo pinza.

\subsubsection{Descripción del Sistema}

El brazo robótico consiste en una configuración de dos eslabones articulados mediante juntas rotacionales, con un efector final tipo pinza. Este sistema posee dos grados de libertad, definidos por los ángulos de rotación $\theta_1$ y $\theta_2$ de cada articulación.

\subsubsection{Parámetros Geométricos}

\begin{itemize}
    \item $L_1$: Longitud del primer eslabón
    \item $L_2$: Longitud del segundo eslabón
    \item $\theta_1$: Ángulo de la primera articulación (base)
    \item $\theta_2$: Ángulo de la segunda articulación (codo)
\end{itemize}

%\subsubsection{Cinemática Directa}

%La cinemática directa establece la relación entre las coordenadas articulares $(\theta_1, \theta_2)$ y la posición del efector final $(x, y)$ en el plano.

%La posición del efector final se calcula mediante:

%\begin{align}
%    x &= L_1 \cdot \cos(\theta_1) + L_2 \cdot \cos(\theta_1 + \theta_2) \\
%    y &= L_1 \cdot \sin(\theta_1) + L_2 \cdot \sin(\theta_1 + \theta_2)
%\end{align}

%La orientación del efector final está dada por:

%\begin{equation}
%    \phi = \theta_1 + \theta_2
%\end{equation}

%\subsubsection{Cinemática Inversa}

%La cinemática inversa determina los ángulos articulares necesarios para alcanzar una posición deseada $(x_d, y_d)$. Este problema presenta dos soluciones posibles (configuración de codo arriba y codo abajo).

%Primero se calcula el ángulo $\theta_2$:

%\begin{equation}
%    \cos(\theta_2) = \frac{x_d^2 + y_d^2 - L_1^2 - L_2^2}{2 \cdot L_1 \cdot L_2}
%\end{equation}

%\begin{equation}
%    \theta_2 = \pm \arccos\left(\frac{x_d^2 + y_d^2 - L_1^2 - L_2^2}{2 \cdot L_1 \cdot L_2}\right)
%\end{equation}

%Luego se determina $\theta_1$:

%\begin{equation}
%    \theta_1 = \arctan2(y_d, x_d) - \arctan2(L_2 \cdot \sin(\theta_2), L_1 + L_2 \cdot \cos(\theta_2))
%\end{equation}

\subsubsection{Espacio de Trabajo}

El espacio de trabajo accesible para el efector final es un anillo circular con:

\begin{itemize}
    \item Radio máximo: $R_{max} = L_1 + L_2$ (brazo completamente extendido)
    \item Radio mínimo: $R_{min} = |L_1 - L_2|$ (brazo completamente plegado)
\end{itemize}

%\subsubsection{Jacobiano}

%El Jacobiano relaciona las velocidades articulares con las velocidades del efector final:

%\begin{equation}
%    \begin{bmatrix}
%        \dot{x} \\
%        \dot{y}
%    \end{bmatrix}
%    =
%    \begin{bmatrix}
%        J_{11} & J_{12} \\
 %       J_{21} & J_{22}
  %  \end{bmatrix}
   % \begin{bmatrix}
    %    \dot{\theta}_1 \\
     %   \dot{\theta}_2
    %\end{bmatrix}
%\end{equation}

%Donde:

%\begin{align}
 %   J_{11} &= -L_1 \cdot \sin(\theta_1) - L_2 \cdot \sin(\theta_1 + \theta_2) \\
  %  J_{12} &= -L_2 \cdot \sin(\theta_1 + \theta_2) \\
   % J_{21} &= L_1 \cdot \cos(\theta_1) + L_2 \cdot \cos(\theta_1 + \theta_2) \\
    %J_{22} &= L_2 \cdot \cos(\theta_1 + \theta_2)
%\end{align}

\subsubsection{Gripper Tipo Pinza}

El gripper se ubica en el extremo del segundo eslabón y posee un grado de libertad adicional (no considerado en los 2 GDL principales) para la apertura/cierre de las mandíbulas, controlado por el parámetro $d$ (distancia entre mandíbulas).

\subsubsection{Singularidades}

El sistema presenta singularidades cuando el determinante del Jacobiano es cero:

\begin{equation}
    \det(J) = L_1 \cdot L_2 \cdot \sin(\theta_2) = 0
\end{equation}

Esto ocurre cuando \(\theta_2 = 0^\circ \) o \(\theta_2 = 180^\circ\), es decir, cuando el brazo está completamente extendido o plegado.

