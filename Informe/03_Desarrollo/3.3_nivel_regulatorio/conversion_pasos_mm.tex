La conversión entre pasos de motor y desplazamiento lineal depende de las características mecánicas de cada eje.

\subsubsection{Configuración de Drivers de Motor}

La Tabla \ref{tab:config_drivers} muestra la configuración de los drivers TB6600 para los motores paso a paso.

\begin{table}[H]
\centering
\caption{Configuración de drivers TB6600 y motores NEMA 17}
\label{tab:config_drivers}
\begin{tabular}{|l|c|c|}
\hline
\textbf{Parámetro} & \textbf{Valor} & \textbf{Observación} \\
\hline
Modelo motor & NEMA 17 & 17HS4401 (1.7A nominal) \\
\hline
Pasos/revolución & 200 & 1.8° por paso \\
\hline
Microstepping & 8x & Configurado vía DIP switches del TB6600 \\
\hline
Pasos totales/rev & 1600 & $200 \times 8 = 1600$ \\
\hline
Corriente configurada & 1.5A & 88\% de corriente nominal \\
\hline
Tensión alimentación & 24V DC & Fuente switching regulada \\
\hline
Modo decay & Fast decay & Para mejor respuesta dinámica \\
\hline
\end{tabular}
\end{table}

El microstepping de 8x ofrece balance óptimo entre resolución (mejora 8x vs modo full-step), suavidad de movimiento y velocidad máxima. Configuraciones mayores (16x, 32x) reducirían el torque disponible sin beneficio práctico dada la resolución mecánica limitada por backlash.

\subsubsection{Eje Horizontal (Correa Dentada)}

El eje horizontal utiliza transmisión por correa GT2 con polea de 20 dientes:

\begin{itemize}
    \item Paso de correa GT2: 2mm/diente
    \item Dientes en polea: 20
    \item Avance lineal/revolución: $20 \times 2 = 40$ mm
\end{itemize}

Conversión:
\begin{equation}
\text{pasos/mm}_H = \frac{1600 \text{ pasos}}{40 \text{ mm}} = 40 \text{ pasos/mm}
\end{equation}

\subsubsection{Eje Vertical (Varilla Roscada)}

El eje vertical utiliza varilla roscada TR8x8 (paso 8mm):

\begin{itemize}
    \item Paso de rosca: 8mm/revolución
    \item Avance lineal/revolución: 8mm
\end{itemize}

Conversión:
\begin{equation}
\text{pasos/mm}_V = \frac{1600 \text{ pasos}}{8 \text{ mm}} = 200 \text{ pasos/mm}
\end{equation}

\subsubsection{Resolución del Sistema}

La resolución teórica de posicionamiento es:

\begin{itemize}
    \item Horizontal: $\frac{1}{40} = 0.025$ mm/paso
    \item Vertical: $\frac{1}{200} = 0.005$ mm/paso
\end{itemize}

En la práctica, factores como backlash de la correa (~0.1mm) y flexión mecánica limitan la resolución efectiva a ~0.1mm en ambos ejes.

\subsubsection{Implementación en Firmware}

El firmware define constantes de conversión (40 pasos/mm para eje horizontal, 200 pasos/mm para eje vertical) y funciones que realizan la conversión bidireccional entre unidades. El protocolo de comunicación con el supervisor utiliza milímetros como unidad estándar, simplificando la interfaz y abstracciendo los detalles mecánicos. El firmware traduce automáticamente todas las distancias recibidas a pasos internamente antes de generar las trayectorias.
