Los requerimientos principales que debe cumplir el controlador incluyen: generación de pulsos de alta frecuencia para motores paso a paso operando hasta 10,000 pasos/s, disponibilidad de múltiples temporizadores por hardware para controlar dos ejes de motores paso a paso y dos servomotores de forma concurrente, señales de nivel lógico de 5V compatibles directamente con los drivers TB6600 y servomotores utilizados, un mínimo de 16 pines de entrada/salida digital para comandar steppers (STEP, DIR, ENABLE para dos ejes), servomotores, finales de carrera y actuador de agarre, comunicación UART por hardware a 115200 baudios sin bloquear el bucle de control, y memoria suficiente para almacenar el firmware (aproximadamente 15KB de Flash) y mantener estructuras de datos en tiempo de ejecución (1.2KB de RAM).

Considerando estos requerimientos, se seleccionó un Arduino Mega 2560 como plataforma de control por satisfacer todos los criterios del sistema y proveer recursos adicionales para futuras expansiones. Este microcontrolador está basado en el ATmega2560 de 8 bits de Microchip (anteriormente Atmel), que opera a 16MHz con arquitectura RISC AVR, lo que resulta suficiente para ejecutar bucles de control a frecuencias superiores a 10kHz con latencias predecibles, cumpliendo ampliamente los requisitos de temporización del sistema.

En términos de memoria, cuenta con 256KB de memoria Flash para almacenar el programa (el firmware actual utiliza aproximadamente 15KB, representando solo un 6\% de uso), 8KB de SRAM para variables y estructuras de datos en tiempo de ejecución (uso actual de 1.2KB, equivalente al 15\%), y 4KB de EEPROM para almacenamiento persistente de parámetros de configuración como velocidades máximas, aceleraciones y límites de recorrido.

Los recursos de hardware incluyen 6 temporizadores (cuatro de 16 bits: Timer1, Timer3, Timer4 y Timer5; dos de 8 bits: Timer0 y Timer2), 54 pines digitales de entrada/salida configurables, 16 entradas analógicas con conversores ADC de 10 bits, 4 UARTs por hardware independientes y 15 canales de PWM por hardware. Esta abundancia de recursos permite asignar un temporizador dedicado para la generación de pulsos de steppers (Timer1) y dos temporizadores independientes para los servomotores (Timer4 y Timer5), evitando conflictos de recursos.

Un aspecto crítico de la selección fue la compatibilidad de nivel lógico. El ATmega2560 opera a 5V, mientras que alternativas modernas como los microcontroladores STM32 de 32 bits (basados en ARM Cortex-M) o ESP32 operan a 3.3V. Los drivers de motores paso a paso TB6600 y servomotores seleccionados requieren señales de entrada de 5V. El uso de un microcontrolador de 3.3V requeriría circuitería adicional de level shifting (conversores de nivel lógico bidireccionales), incrementando la complejidad del diseño, el costo de componentes, y potencialmente introduciendo retardos adicionales.

La selección del Arduino Mega 2560 representa un equilibrio óptimo entre capacidad de procesamiento, recursos de hardware, compatibilidad eléctrica directa con los actuadores, precio y facilidad de desarrollo.