\subsubsection{Distribución de Pines del Arduino Mega 2560}

La asignación de pines del microcontrolador se diseñó como se presenta en la Figura \ref{fig:pinout_arduino_mega}, que muestra la disposición física de conexiones en la placa.

\begin{figure}[H]
    \centering
    % TODO: Insertar imagen del pinout del Arduino Mega con las asignaciones marcadas
    \includegraphics[width=0.7\textwidth]{imagenes/pinout_arduino_mega.png}
    \caption{Distribución de pines del Arduino Mega 2560 para el Nivel Regulatorio}
    \label{fig:pinout_arduino_mega}
\end{figure}

El eje horizontal de movimiento (X) utiliza los pines 22, 23 y 24 para las señales STEP, DIR y ENABLE respectivamente, mientras que el eje vertical (Y) utiliza los pines 25, 26 y 27 con las mismas funciones. Los servomotores del brazo se conectan a pines con capacidad PWM por hardware: el servomotor 1 utiliza el pin 11 y el servomotor 2 el pin 12. El actuador de agarre, implementado con un motor paso a paso, requiere cuatro señales de fase conectadas a los pines 30 a 33.

Los finales de carrera ocupan los pines 40 a 43. El pin 40 corresponde al límite mínimo del eje horizontal, el pin 41 al límite máximo horizontal, el pin 42 al límite mínimo vertical, y el pin 43 al límite máximo vertical. Estos pines utilizan resistencias pull-up internas activadas por software.

La comunicación serial con la Raspberry Pi utiliza los pines 0 (RX0) y 1 (TX0), correspondientes al USART0 del ATmega2560.