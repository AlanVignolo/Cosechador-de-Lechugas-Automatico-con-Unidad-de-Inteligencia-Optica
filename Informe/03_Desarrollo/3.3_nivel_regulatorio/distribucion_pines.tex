La distribución de pines del Arduino Mega 2560 se diseñó para optimizar el ruteo de señales, minimizar interferencias y facilitar el mantenimiento (Tabla \ref{tab:distribucion_pines}).

\begin{table}[H]
\centering
\caption{Distribución de pines del Arduino Mega 2560}
\label{tab:distribucion_pines}
\begin{tabular}{|l|l|p{6cm}|}
\hline
\textbf{Pin} & \textbf{Función} & \textbf{Descripción} \\
\hline
\multicolumn{3}{|c|}{\textbf{Eje Horizontal (X)}} \\
\hline
22 & STEP\_H & Pulsos de paso para motor horizontal \\
23 & DIR\_H & Dirección de giro motor horizontal \\
24 & ENABLE\_H & Habilitación driver TB6600 horizontal \\
\hline
\multicolumn{3}{|c|}{\textbf{Eje Vertical (Y)}} \\
\hline
25 & STEP\_V & Pulsos de paso para motor vertical \\
26 & DIR\_V & Dirección de giro motor vertical \\
27 & ENABLE\_V & Habilitación driver TB6600 vertical \\
\hline
\multicolumn{3}{|c|}{\textbf{Servomotores del Brazo}} \\
\hline
11 (PWM) & SERVO1 & Control PWM servo 1 (hombro) - Timer1 \\
12 (PWM) & SERVO2 & Control PWM servo 2 (codo) - Timer1 \\
\hline
\multicolumn{3}{|c|}{\textbf{Gripper (Motor 28BYJ-48)}} \\
\hline
30 & GRIPPER\_IN1 & Fase 1 del motor paso a paso \\
31 & GRIPPER\_IN2 & Fase 2 del motor paso a paso \\
32 & GRIPPER\_IN3 & Fase 3 del motor paso a paso \\
33 & GRIPPER\_IN4 & Fase 4 del motor paso a paso \\
\hline
\multicolumn{3}{|c|}{\textbf{Finales de Carrera}} \\
\hline
40 & LIMIT\_H\_MIN & Final de carrera horizontal mínimo (inicio) \\
41 & LIMIT\_H\_MAX & Final de carrera horizontal máximo (fin) \\
42 & LIMIT\_V\_MIN & Final de carrera vertical mínimo (abajo) \\
43 & LIMIT\_V\_MAX & Final de carrera vertical máximo (arriba) \\
\hline
\multicolumn{3}{|c|}{\textbf{Comunicación}} \\
\hline
0 (RX0) & UART\_RX & Recepción serial desde Raspberry Pi \\
1 (TX0) & UART\_TX & Transmisión serial hacia Raspberry Pi \\
\hline
\end{tabular}
\end{table}

\textbf{Criterios de asignación}:

\begin{itemize}
    \item \textbf{Pines PWM para servos}: Se utilizan pines con soporte hardware PWM (Timer1) para generar señales precisas de 50Hz sin consumir ciclos de CPU.
    \item \textbf{Agrupación por función}: Pines de steppers, gripper y límites agrupados en rangos contiguos facilita el cableado y debugging.
    \item \textbf{Reserva de pines analógicos}: Se mantienen libres los 16 pines analógicos (A0-A15) para futuras expansiones (sensores de corriente, temperatura, etc.).
    \item \textbf{Separación de señales críticas}: STEP/DIR alejados de pines de comunicación para minimizar acoplamiento capacitivo y ruido.
\end{itemize}

Esta distribución permite un cableado ordenado con cables de longitud similar (~30cm) desde el Arduino hasta la shield de drivers, minimizando caídas de tensión y susceptibilidad a EMI.
