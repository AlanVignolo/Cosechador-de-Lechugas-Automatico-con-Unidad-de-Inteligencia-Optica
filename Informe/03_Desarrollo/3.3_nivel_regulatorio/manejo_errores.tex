\subsubsection{Manejo de Errores y Protecciones}

El firmware implementa mecanismos robustos de detección y manejo de errores para garantizar operación segura del sistema mecánico.

Antes de ejecutar cualquier comando, el analizador sintáctico valida sintaxis y rangos. Comandos inválidos retornan código de error inmediatamente sin ejecutar acción, previniendo movimientos peligrosos (Tabla \ref{tab:validacion_comandos}).

\begin{table}[H]
\centering
\small
\begin{tabular}{|l|l|l|}
\hline
\textbf{Parámetro} & \textbf{Rango Válido} & \textbf{Error} \\
\hline
Delimitadores & \texttt{<} y \texttt{>} presentes & INVALID\_CMD \\
\hline
Velocidad H & 500 - 15,000 pasos/s & INVALID\_PARAM \\
\hline
Velocidad V & 500 - 12,000 pasos/s & INVALID\_PARAM \\
\hline
Ángulo servo & 10° - 160° & INVALID\_PARAM \\
\hline
Tiempo trayectoria & 0 - 10,000 ms & INVALID\_PARAM \\
\hline
\end{tabular}
\caption{Validaciones de comandos y rangos permitidos}
\label{tab:validacion_comandos}
\end{table}

El firmware implementa supervisor de comunicación: si no recibe señal de sincronización (\texttt{HB:1}) del nivel superior durante un período prolongado, asume pérdida de comunicación y ejecuta detención de emergencia automática, deshabilitando todos los motores y entrando en modo seguro hasta recibir comando de reset.

El buffer UART circular (256 bytes) implementa protección contra overflow verificando espacio disponible antes de cada escritura. Si el buffer está lleno, el byte se descarta y se envía error de overflow. La Tabla \ref{tab:codigos_error} lista los códigos de error definidos.

\begin{table}[H]
\centering
\begin{tabular}{|l|p{9cm}|}
\hline
\textbf{Código} & \textbf{Descripción y Causa} \\
\hline
INVALID\_CMD & Comando no reconocido o sintaxis incorrecta. \\
\hline
INVALID\_PARAM & Parámetros fuera de rango válido. \\
\hline
LIMIT\_HIT & Movimiento bloqueado por activación de final de carrera. \\
\hline
BUFFER\_OVERFLOW & Buffer UART saturado. \\
\hline
TIMEOUT & Operación excedió tiempo límite configurado. \\
\hline
SYSTEM\_ERROR & Error interno del firmware. \\
\hline
\end{tabular}
\caption{Códigos de error del firmware}
\label{tab:codigos_error}
\end{table}

Los finales de carrera se monitorean continuamente. El sistema detecta la activación de cualquier límite y ejecuta inmediatamente la detención de emergencia: deshabilita los drivers (ENABLE = HIGH), detiene la generación de pulsos, envía notificación asíncrona al supervisor indicando el límite activado, y bloquea nuevos comandos de movimiento hacia el lado en el que se encuentra el fin de carrera. Esto previene daños mecánicos por intentos de movimiento en dirección bloqueada.
