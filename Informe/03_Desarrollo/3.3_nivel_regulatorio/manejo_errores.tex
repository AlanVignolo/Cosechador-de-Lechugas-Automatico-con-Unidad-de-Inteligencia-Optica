El firmware implementa mecanismos robustos de detección y manejo de errores para garantizar operación segura del sistema mecánico.

\subsubsection{Detección de Límites Físicos}

Los finales de carrera se monitorean continuamente mediante lectura de pines digitales en el loop principal. El sistema detecta la activación de cualquier límite (nivel lógico bajo) y ejecuta inmediatamente la secuencia de detención de emergencia, enviando simultáneamente una notificación al supervisor indicando cuál límite se activó.

La activación de cualquier límite durante movimiento causa:
\begin{enumerate}
    \item Detención inmediata de todos los motores
    \item Deshabilitación de drivers (pin ENABLE = HIGH)
    \item Envío de evento asíncrono al supervisor
    \item Bloqueo de nuevos comandos de movimiento hasta reset manual
\end{enumerate}

\subsubsection{Validación de Comandos}

Antes de ejecutar cualquier comando, el parser valida sintaxis y rangos según la Tabla \ref{tab:validacion_comandos}.

\begin{table}[H]
\centering
\caption{Validaciones de comandos y rangos permitidos}
\label{tab:validacion_comandos}
\begin{tabular}{|l|l|l|}
\hline
\textbf{Parámetro} & \textbf{Rango Válido} & \textbf{Error si Invalido} \\
\hline
Delimitadores & \texttt{<} y \texttt{>} presentes & INVALID\_CMD \\
\hline
Velocidad H & 500 - 15,000 pasos/s & INVALID\_PARAM \\
\hline
Velocidad V & 500 - 12,000 pasos/s & INVALID\_PARAM \\
\hline
Ángulo servo & 10° - 160° & INVALID\_PARAM \\
\hline
Tiempo trayectoria & 0 - 10,000 ms & INVALID\_PARAM \\
\hline
Número de parámetros & Según comando & INVALID\_CMD \\
\hline
\end{tabular}
\end{table}

Comandos inválidos retornan código de error inmediatamente sin ejecutar acción, previniendo movimientos peligrosos.

\subsubsection{Timeout de Comunicación}

El firmware implementa watchdog de comunicación: si no recibe heartbeat (\texttt{HB:1}) del supervisor en 90 segundos consecutivos, asume pérdida de comunicación y ejecuta EMERGENCY\_STOP automático, deshabilitando todos los motores y entrando en modo seguro hasta recibir comando de reset. Este mecanismo previene movimientos incontrolados si el supervisor falla.

\subsubsection{Protección de Buffer y Códigos de Error}

El buffer UART circular (256 bytes) implementa protección contra overflow verificando espacio disponible antes de cada escritura. Si el buffer está lleno, el byte se descarta y se envía \texttt{<ERROR:BUFFER\_OVERFLOW>}, reiniciando el parser.

La Tabla \ref{tab:codigos_error} lista los códigos de error definidos para diagnóstico sistemático.

\begin{table}[H]
\centering
\caption{Códigos de error del firmware}
\label{tab:codigos_error}
\begin{tabular}{|l|p{9cm}|}
\hline
\textbf{Código} & \textbf{Descripción y Causa} \\
\hline
INVALID\_CMD & Comando no reconocido o sintaxis incorrecta. \\
\hline
INVALID\_PARAM & Parámetros fuera de rango válido (velocidad, ángulo, tiempo). \\
\hline
LIMIT\_HIT & Movimiento bloqueado por activación de final de carrera. \\
\hline
BUFFER\_OVERFLOW & Buffer UART saturado. Supervisor enviando datos muy rápido. \\
\hline
TIMEOUT & Operación excedió timeout configurado. \\
\hline
SYSTEM\_ERROR & Error interno del firmware (fallo inesperado). \\
\hline
\end{tabular}
\end{table}

Estos códigos permiten al supervisor diagnosticar fallas específicas y aplicar estrategias de recuperación apropiadas (reintentos, degradación, o parada segura).
