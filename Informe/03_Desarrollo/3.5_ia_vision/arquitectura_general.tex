El sistema de visión por computadora se estructura en tres módulos especializados e independientes, cada uno diseñado para una tarea específica de detección o clasificación. Esta arquitectura modular emplea algoritmos clásicos basados en procesamiento morfológico y análisis de características geométricas, evitando la complejidad computacional y requisitos de entrenamiento asociados con métodos de aprendizaje profundo.\\

\underline{Detector de marcadores de referencia:} Localiza cintas negras adhesivas de 18mm de ancho. Opera en tres contextos: corrección de posición horizontal (centroide de cinta vertical), corrección de posición vertical (arista base de cinta), y escaneo horizontal para registrar posiciones \textbf{X} de plantas en matriz\_cintas.json.\\

\underline{Detector de tubos de cultivo:} Identifica tubos de PVC durante el escaneo vertical del workspace. Registra las posiciones verticales \textbf{Y} de todos los tubos detectados, generando un archivo que permite al sistema ubicar cada fila de cultivo.\\

\underline{Clasificador morfológico de cultivos:} Determina la presencia de planta y su nivel de madurez mediante clasificación binaria (maduro/no maduro). Discrimina entre lechugas maduras, plantas inmaduras, y posiciones vacías.
%El sistema de visión por computadora se estructura en tres módulos especializados que operan de manera durante los procesos de exploración y cosecha. A diferencia de arquitecturas basadas en clasificación multiclase, se optó por detectores binarios especializados por las siguientes razones:
%\begin{enumerate}
%\item Separación temporal de tareas: Los tres detectores operan en fases distintas del ciclo de operación.
        %\begin{itemize}[label=$\bullet$]
        %\item Detector de cintas: corrección de posicionamiento durante aproximación
        %\item Detector de tubos: fase de mapeo vertical del workspace
        %\item Clasificador de cultivos: evaluación de madurez en posiciones ya identificadas
        %\end{itemize}

%\item Diferencias en características visuales: Cada elemento presenta propiedades discriminativas específicas:
        %\begin{itemize}[label=$\bullet$]
        %\item Cintas: contraste extremo en canal V (ecuación 3.66)
        %\item Tubos: detección de bordes con filtrado por saturación baja (Sección 3.6.3)
        %\item Lechugas: segmentación cromática en espacio HSV + análisis morfológico (Sección 3.6.4)
        %\end{itemize}

%\item computacional: Algoritmos basados en procesamiento morfológico y umbralización, ejecutables en tiempo real sin requerir GPU.
%\end{enumerate}
%Los tres detectores se basan en técnicas de visión clásica sin aprendizaje profundo:
%\begin{itemize}[label=$\bullet$]
%    \item Detector de cintas (Sección 3.6.2): Umbralización inversa sobre canal V + análisis de contornos
%    \item Detector de tubos (Sección 3.6.3): Algoritmo de Canny + filtrado por saturación + morfología
%    \item Clasificador de cultivos (Sección 3.6.4): Segmentación por rangos HSV + clasificación por área (ecuaciones 3.18-3.19)
%\end{itemize}


%Esta arquitectura modular permite optimizar cada detector para su tarea específica y facilita el mantenimiento, validación individual y posible sustitución futura por modelos de deep learning sin afectar otros componentes del sistema.

%\subsubsection{Evaluación independiente de detectores}0

%Cada detector fue evaluado de forma independiente según su contexto operativo específico. La Tabla~\ref{tab:contextos_detectores} resume cuándo y cómo se activa cada uno durante la operación del sistema.

%\begin{table}[H]
%\centering
%\label{tab:contextos_detectores}
%\small
%\begin{tabular}{|p{2.8cm}|p{3.2cm}|p{2.5cm}|p{4.5cm}|}
%\hline
%\textbf{Detector} & \textbf{Fase operativa} & \textbf{Frecuencia} & \textbf{Salida} \\ \hline
%Detector de cintas & Corrección de posición + Mapeo horizontal & 6 fps durante movimiento & flag detección o posterior cálculo de coordenada x\\ \hline
%Detector de tubos & Mapeo vertical del workspace & 3 fps durante barrido & flag detección + posterior cálculo de coordenada y \\ \hline
%Clasificador de lechugas & Evaluación de madurez & 1 captura por posición & Clase (lechuga/plantin/vaso) + confianza \\ \hline
%\end{tabular}
%\caption{Contextos de activación de detectores}
%\end{table}