El sistema de visión por computadora se estructura en tres módulos especializados que operan de manera coordinada durante los procesos de exploración y cosecha:\\

\underline{Detector de marcadores de referencia:} Localiza cintas negras adhesivas de 18mm de ancho. Opera en tres contextos: corrección de posición horizontal (centroide de cinta vertical), corrección de posición vertical (arista base de cinta), y escaneo horizontal para registrar posiciones \textbf{X} de plantas en matriz\_cintas.json.\\

\underline{Detector de tubos de cultivo:} Identifica tubos de PVC durante el escaneo vertical del workspace. Registra las posiciones verticales \textbf{Y} de todos los tubos detectados, generando un archivo que permite al sistema ubicar cada fila de cultivo.\\

\underline{Clasificador morfológico de cultivos:} Determina la presencia de planta y el nivel de madurez. Discrimina entre lechugas maduras, plantas inmaduras, y posiciones vacías.

El sistema de visión por computadora se estructura en tres módulos especializados que operan de manera secuencial durante los procesos de exploración y cosecha. A diferencia de arquitecturas basadas en clasificación multiclase, se optó por detectores binarios especializados por las siguientes razones:
\begin{enumerate}
\item Separación temporal de tareas: Los tres detectores operan en fases distintas del ciclo de operación.
        %\begin{itemize}[label=$\bullet$]
        %\item Detector de cintas: corrección de posicionamiento durante aproximación
        %\item Detector de tubos: fase de mapeo vertical del workspace
        %\item Clasificador de cultivos: evaluación de madurez en posiciones ya identificadas
        %\end{itemize}

\item Diferencias en características visuales: Cada elemento presenta propiedades discriminativas específicas:
        %\begin{itemize}[label=$\bullet$]
        %\item Cintas: contraste extremo en canal V (ecuación 3.66)
        %\item Tubos: detección de bordes con filtrado por saturación baja (Sección 3.6.3)
        %\item Lechugas: segmentación cromática en espacio HSV + análisis morfológico (Sección 3.6.4)
        %\end{itemize}

\item computacional: Algoritmos basados en procesamiento morfológico y umbralización, ejecutables en tiempo real sin requerir GPU.
\end{enumerate}
Los tres detectores se basan en técnicas de visión clásica sin aprendizaje profundo:
\begin{itemize}[label=$\bullet$]
    \item Detector de cintas (Sección 3.6.2): Umbralización inversa sobre canal V + análisis de contornos
    \item Detector de tubos (Sección 3.6.3): Algoritmo de Canny + filtrado por saturación + morfología
    \item Clasificador de cultivos (Sección 3.6.4): Segmentación por rangos HSV + clasificación por área (ecuaciones 3.18-3.19)
\end{itemize}


Esta arquitectura modular permite optimizar cada detector para su tarea específica y facilita el mantenimiento, validación individual y posible sustitución futura por modelos de deep learning sin afectar otros componentes del sistema.

%\subsubsection{Evaluación independiente de detectores}

Cada detector fue evaluado de forma independiente según su contexto operativo específico. La Tabla~\ref{tab:contextos_detectores} resume cuándo y cómo se activa cada uno durante la operación del sistema.

\begin{table}[H]
\centering
\caption{Contextos de activación de detectores}
\label{tab:contextos_detectores}
\small
\begin{tabular}{|p{2.8cm}|p{3.2cm}|p{2.5cm}|p{4.5cm}|}
\hline
\textbf{Detector} & \textbf{Fase operativa} & \textbf{Frecuencia} & \textbf{Salida} \\ \hline
Detector de cintas & Corrección de posición + Mapeo horizontal & 6 fps durante movimiento & Coordenadas (x,y) del centroide + flag detección \\ \hline
Detector de tubos & Mapeo vertical del workspace & 3 fps durante barrido & Coordenada Y del tubo + flag detección \\ \hline
Clasificador de lechugas & Evaluación de madurez & 1 captura por posición & Clase (madura/inmadura/vacío) + confianza \\ \hline
\end{tabular}
\end{table}

\noindent
Métricas de desempeño por detector:\\
\noindent
\begin{enumerate}
    \item Detector de cintas negras.\\
\noindent
Evaluado sobre 35 imágenes de validación (no usadas en calibración de umbrales):

\begin{table}[H]
\centering
\begin{tabular}{|l|r|}
\hline
\textbf{Métrica} & \textbf{Valor} \\ \hline
Tasa de detección correcta & 94.3\% (33/35) \\ \hline
Falsos negativos & 5.7\% (2/35) \\ \hline
Falsos positivos & 0\% (0/35) \\ \hline
Error de posicionamiento medio & 3.2 px ($\approx$1.8 mm) \\ \hline
Error máximo de posicionamiento & 8.1 px ($\approx$4.5 mm) \\ \hline
\end{tabular}
\caption{Desempeño del detector de cintas}
\label{tab:metricas_cintas}
\end{table}

Matriz de detección para cintas:

\begin{table}[H]
\centering
\begin{tabular}{cc|c|c|}
\cline{3-4}
& & \multicolumn{2}{c|}{\textbf{Predicción}} \\ \cline{3-4}
& & Detectada & No detectada \\ \hline
\multicolumn{1}{|c|}{\multirow{2}{*}{\textbf{Real}}} & Presente & 33 & 2 \\ \cline{2-4}
\multicolumn{1}{|c|}{} & Ausente & 0 & -- \\ \hline
\end{tabular}
\caption{Matriz de detección binaria - cintas}
\label{tab:confusion_cintas}
\end{table}

\noindent
Los 2 falsos negativos ocurrieron con iluminación $<$700 lux, por debajo del rango operativo especificado (800-1200 lux).

\paragraph{2. Detector de tubos de PVC}
Evaluado sobre 28 frames de barrido vertical (14 con tubos, 14 sin tubos):

\begin{table}[H]
\centering
\begin{tabular}{|l|r|}
\hline
\textbf{Métrica} & \textbf{Valor} \\ \hline
Sensibilidad (Recall) & 92.9\% (13/14) \\ \hline
Especificidad & 100\% (14/14) \\ \hline
Precisión (Precision) & 100\% (13/13) \\ \hline
Error de localización Y medio & 4.7 mm \\ \hline
Error máximo de localización Y & 12.3 mm \\ \hline
\end{tabular}
\caption{Desempeño del detector de tubos}
\label{tab:metricas_tubos}
\end{table}

\textbf{Matriz de detección para tubos:}

\begin{table}[H]
\centering
\begin{tabular}{cc|c|c|}
\cline{3-4}
& & \multicolumn{2}{c|}{\textbf{Predicción}} \\ \cline{3-4}
& & Tubo detectado & No tubo \\ \hline
\multicolumn{1}{|c|}{\multirow{2}{*}{\textbf{Real}}} & Tubo & 13 & 1 \\ \cline{2-4}
\multicolumn{1}{|c|}{} & No tubo & 0 & 14 \\ \hline
\end{tabular}
\caption{Matriz de detección binaria - tubos}
\label{tab:confusion_tubos}
\end{table}

\noindent
El único falso negativo se debió a reflejo especular que saturó el canal V del sensor.

\paragraph{3. Clasificador de lechugas}
Evaluado sobre 25 imágenes de validación independientes:

\begin{table}[H]
\centering
\begin{tabular}{|l|c|c|c|c|}
\hline
\textbf{Clase} & \textbf{Precisión} & \textbf{Recall} & \textbf{F1-Score} & \textbf{Muestras} \\ \hline
Lechuga madura & 90.0\% & 90.0\% & 0.90 & 10 \\ \hline
Plantín/inmadura & 77.8\% & 87.5\% & 0.82 & 8 \\ \hline
Vaso vacío & 100\% & 85.7\% & 0.92 & 7 \\ \hline
\textbf{Promedio ponderado} & \textbf{88.2\%} & \textbf{88.0\%} & \textbf{0.88} & \textbf{25} \\ \hline
\end{tabular}
\caption{Métricas por clase del clasificador de lechugas}
\label{tab:metricas_lechugas}
\end{table}

\textbf{Matriz de confusión del clasificador de lechugas:}

\begin{table}[H]
\centering
\begin{tabular}{cc|c|c|c|}
\cline{3-5}
& & \multicolumn{3}{c|}{\textbf{Predicción}} \\ \cline{3-5}
& & Madura & Inmadura & Vacío \\ \hline
\multicolumn{1}{|c|}{\multirow{3}{*}{\textbf{Real}}} & Madura & 9 & 1 & 0 \\ \cline{2-5}
\multicolumn{1}{|c|}{} & Inmadura & 1 & 7 & 0 \\ \cline{2-5}
\multicolumn{1}{|c|}{} & Vacío & 0 & 1 & 6 \\ \hline
\end{tabular}
\caption{Matriz de confusión multiclase - lechugas}
\label{tab:confusion_lechugas}
\end{table}

\noindent
\textbf{Análisis de errores:}
\begin{itemize}[label=$\bullet$]
    \item \textbf{Cintas}: Los 2 falsos negativos ocurrieron con iluminación insuficiente ($<$700 lux)
    \item \textbf{Tubos}: El falso negativo se debió a reflejo especular que saturó el sensor
    \item \textbf{Lechugas}: 
    \begin{itemize}
        \item Confusión madura $\leftrightarrow$ inmadura (2 casos): plantas en estado de transición con área $\approx$18,000 px cercana al umbral de decisión
        \item Confusión vacío $\rightarrow$ inmadura (1 caso): sombra proyectada generó contorno espurio
    \end{itemize}
\end{itemize}

\paragraph{Validación de repetibilidad}
Para evaluar la consistencia, se realizaron 10 capturas consecutivas de la misma escena:

\begin{table}[H]
\centering
\begin{tabular}{|l|c|c|}
\hline
\textbf{Detector} & \textbf{Consistencia} & \textbf{Varianza espacial} \\ \hline
Cintas & 100\% (10/10) & $\sigma_x = 0.8$ px, $\sigma_y = 1.2$ px \\ \hline
Tubos & 100\% (10/10) & $\sigma_y = 2.1$ px \\ \hline
Lechugas & 90\% (9/10) & N/A (categórico) \\ \hline
\end{tabular}
\caption{Repetibilidad de detectores}
\label{tab:repetibilidad}
\end{table}

\noindent
La variación en el clasificador de lechugas ocurrió en una muestra con área fluctuando entre 18,200-18,800 px debido a variaciones de iluminación, confirmando la necesidad de zona de incertidumbre alrededor de umbrales.

\paragraph{Tiempos de procesamiento}
Medidos en Raspberry Pi 5 (procesador ARM Cortex-A76 quad-core a 2.4 GHz):

\begin{table}[H]
\centering
\begin{tabular}{|l|c|c|}
\hline
\textbf{Detector} & \textbf{Tiempo medio} & \textbf{Desviación estándar} \\ \hline
Cintas & 32 ms & 4 ms \\ \hline
Tubos & 48 ms & 7 ms \\ \hline
Lechugas & 41 ms & 5 ms \\ \hline
\end{tabular}
\caption{Tiempos de procesamiento por detector}
\label{tab:tiempos_procesamiento}
\end{table}

\noindent
Todos los detectores cumplen requisitos de tiempo real para sus frecuencias operativas declaradas en la Tabla~\ref{tab:contextos_detectores}.

\end{enumerate}