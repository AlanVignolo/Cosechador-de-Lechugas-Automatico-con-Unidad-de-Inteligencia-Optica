El sistema de visión por computadora se estructura en tres módulos especializados que operan de manera coordinada durante los procesos de exploración y cosecha:\\

\underline{Detector de tubos de cultivo:} Identifica tubos de PVC mediante análisis de bordes y saturación durante el escaneo vertical del workspace. Utiliza el algoritmo de Canny combinado con filtrado en el canal de saturación (HSV) para detectar los bordes superior e inferior de cada tubo. Registra las posiciones verticales \textbf{Y} de todos los tubos detectados, generando un archivo configuracion\_tubos.json que permite al sistema ubicar cada fila de cultivo.\\

\underline{Detector de cintas de referencia:} Localiza cintas negras adhesivas de 18mm de ancho mediante umbralización inversa del canal V (HSV). Opera en tres contextos: corrección de posición horizontal (centroide de cinta vertical), corrección de posición vertical (arista base de cinta), y escaneo horizontal para registrar posiciones \textbf{X} de plantas en matriz\_cintas.json.\\

\underline{Clasificador morfológico de cultivos:} Determina el estado de madurez mediante análisis del área de vegetación verde segmentada. Aplica umbralización en el espacio HSV para extraer píxeles verdes, refinamiento morfológico (cierre y apertura), detección de contornos, y clasificación por umbral de área estadísticamente determinado. Discrimina entre lechugas maduras (área mayor al umbral), plantas inmaduras, y posiciones vacías.
