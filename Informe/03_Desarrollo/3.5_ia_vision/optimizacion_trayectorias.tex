\subsubsection{Formulación del Problema de Optimización}

Una vez completado el mapeo del entorno y habiendo identificado las estaciones que contienen lechugas disponibles para cosecha, el sistema debe determinar la secuencia óptima de visita que minimice el tiempo total de operación. Este problema se modela formalmente como una variante del problema del viajante (Travelling Salesman Problem, TSP) adaptada a las características específicas del robot cartesiano.

Sea $\mathcal{P} = \{P_1, P_2, ..., P_N\}$ el conjunto de $N$ estaciones objetivo identificadas durante el mapeo, donde cada $P_i = (x_i, y_i)$ representa las coordenadas de una estación con lechuga. El objetivo es encontrar una permutación $\pi = [\pi_1, \pi_2, ..., \pi_N]$ que defina la secuencia de visita y minimice el tiempo total de recorrido.

El tiempo total de operación se compone de dos términos: el tiempo acumulado de desplazamiento entre estaciones consecutivas y el tiempo acumulado de operación en cada estación. Formalmente:

\begin{equation}
J_{total}(\pi) = \sum_{i=1}^{N-1} t_{mov}(P_{\pi_i}, P_{\pi_{i+1}}) + \sum_{i=1}^{N} t_{op}(P_{\pi_i})
\end{equation}

El tiempo de operación en cada estación $t_{op}$ incluye la corrección visual de posición, la verificación de presencia de lechuga, la ejecución de la cosecha y el depósito del producto. Este tiempo es aproximadamente constante (alrededor de 8 segundos) independientemente de la posición de la estación. Por lo tanto, el término $\sum_{i=1}^{N} t_{op}(P_{\pi_i})$ es constante para cualquier permutación $\pi$, y la optimización se reduce a:

\begin{equation}
\min_{\pi} \sum_{i=1}^{N-1} t_{mov}(P_{\pi_i}, P_{\pi_{i+1}})
\end{equation}

Este problema es NP-difícil en su formulación general, lo que significa que no existe algoritmo conocido que garantice encontrar la solución óptima en tiempo polinomial para instancias arbitrarias. Sin embargo, para los tamaños de instancia relevantes para este sistema (típicamente $N \leq 50$ estaciones con lechuga), es factible emplear heurísticas que producen soluciones de alta calidad en tiempos de cómputo aceptables.

\subsubsection{Modelo de Tiempo de Movimiento}

El cálculo del tiempo de desplazamiento entre dos estaciones debe considerar que el robot tiene velocidades características diferentes en cada eje. El eje horizontal (X) emplea transmisión por correa dentada y alcanza velocidad de 100 mm/s, mientras que el eje vertical (Y) emplea husillo de bolas y alcanza velocidad de 50 mm/s.

Dado que ambos ejes se mueven simultáneamente, el tiempo de desplazamiento entre dos puntos $P_i = (x_i, y_i)$ y $P_j = (x_j, y_j)$ está determinado por el eje que requiere mayor tiempo:

\begin{equation}
t_{mov}(P_i, P_j) = \max\left(\frac{|\Delta x|}{v_x}, \frac{|\Delta y|}{v_y}\right)
\end{equation}

donde $\Delta x = x_j - x_i$, $\Delta y = y_j - y_i$, $v_x = 100$ mm/s y $v_y = 50$ mm/s.

Esta expresión asume que el tiempo está dominado por el movimiento puro, despreciando aceleración y deceleración. Esta aproximación es razonable dado que las distancias típicas entre estaciones (150 mm horizontal, 200 mm vertical) son suficientes para alcanzar velocidad de régimen permanente.

Sin embargo, para la optimización se emplea una función de costo modificada que considera el desgaste diferencial de los actuadores. El husillo del eje Y presenta mayor desgaste mecánico que la correa del eje X debido a las cargas verticales que soporta. Para balancear el uso de ambos ejes y prolongar la vida útil del sistema, se introduce ponderación en la función de costo:

\begin{equation}
C_{mov}(P_i, P_j) = \alpha \cdot \frac{|\Delta x|}{v_x} + \beta \cdot \frac{|\Delta y|}{v_y}
\end{equation}

Los factores de ponderación se establecen como $\alpha = 0.6$ y $\beta = 1.0$, otorgando mayor peso a los movimientos verticales. Esta ponderación incentiva al algoritmo de optimización a preferir rutas que minimicen desplazamientos en Y, reduciendo el desgaste del husillo sin comprometer significativamente el tiempo total.

\subsubsection{Algoritmo Heurístico del Vecino Más Cercano}

El algoritmo del vecino más cercano constituye una heurística constructiva que genera soluciones de buena calidad con complejidad computacional manejable. El algoritmo construye la ruta incrementalmente, en cada paso seleccionando la estación no visitada más cercana a la posición actual según la función de costo definida.

El procedimiento inicia en la posición actual del robot, que típicamente corresponde a la última posición alcanzada durante el mapeo. En cada iteración, se calculan los costos de desplazamiento desde la posición actual hacia todas las estaciones no visitadas. Se selecciona la estación con menor costo, se agrega a la ruta y se marca como visitada. La posición actual se actualiza a las coordenadas de la estación recién agregada. El proceso se repite hasta que todas las estaciones han sido visitadas.

Formalmente, el algoritmo se describe como:

Inicializar la posición actual $P_{actual}$ como la posición del robot.
Inicializar el conjunto de estaciones restantes $\mathcal{R} = \mathcal{P}$.
Inicializar la ruta $\pi$ como vacía.

Mientras $\mathcal{R} \neq \emptyset$:
\begin{equation}
P_{próximo} = \arg\min_{P \in \mathcal{R}} C_{mov}(P_{actual}, P)
\end{equation}

Agregar $P_{próximo}$ a $\pi$.
Eliminar $P_{próximo}$ de $\mathcal{R}$.
Actualizar $P_{actual} = P_{próximo}$.

La complejidad computacional de este algoritmo es $O(N^2)$ donde $N$ es el número de estaciones, dado que en cada una de las $N$ iteraciones se evalúan los costos hacia las estaciones restantes. Para $N = 50$, esto representa 2500 evaluaciones de costo, operación completada en menos de 10 milisegundos en la Raspberry Pi 4.

\subsubsection{Refinamiento mediante Optimización 2-opt}

La solución generada por el vecino más cercano puede mejorarse mediante optimización local 2-opt. Esta técnica considera pares de aristas en la ruta y evalúa si intercambiarlas reduce el costo total. Específicamente, dada una ruta $\pi = [P_1, P_2, ..., P_N]$, se consideran aristas $(P_i, P_{i+1})$ y $(P_j, P_{j+1})$ con $i < j$. Se calcula el costo actual de estas aristas:

\begin{equation}
C_{actual} = C_{mov}(P_i, P_{i+1}) + C_{mov}(P_j, P_{j+1})
\end{equation}

Y el costo alternativo si se intercambian las conexiones:

\begin{equation}
C_{alternativo} = C_{mov}(P_i, P_j) + C_{mov}(P_{i+1}, P_{j+1})
\end{equation}

Si $C_{alternativo} < C_{actual}$, la ruta se mejora invirtiendo el segmento entre $i+1$ y $j$. Este proceso se repite iterativamente sobre todos los pares de aristas hasta que no se encuentren mejoras adicionales.

La optimización 2-opt puede reducir el costo de la ruta en 5 a 15 por ciento comparada con la solución del vecino más cercano, con tiempo de ejecución adicional típicamente inferior a 100 milisegundos para instancias de 50 estaciones.

\subsubsection{Heurística Específica de Agrupamiento por Filas}

Una estrategia alternativa explota la geometría estructurada del sistema de cultivo. Dado que las estaciones se organizan en filas horizontales bien definidas, se implementa una heurística que agrupa estaciones por fila y planifica la ruta como secuencia de barridos horizontales.

El algoritmo identifica primero todas las filas únicas presentes en el conjunto de estaciones objetivo. Para cada estación $(x_i, y_i)$, se calcula su fila correspondiente redondeando la coordenada Y al múltiplo de 200 más cercano, dado que las filas están separadas 200 milímetros. Las estaciones se agrupan por fila, y las filas se ordenan por su coordenada Y.

Para cada fila, las estaciones se ordenan por coordenada X. La dirección de ordenamiento alterna entre filas: las filas con índice par se ordenan de menor a mayor X (izquierda a derecha), mientras que las filas con índice impar se ordenan de mayor a menor X (derecha a izquierda). Esta alternancia implementa el patrón serpiente que minimiza movimientos verticales de retorno.

La ruta final concatena las estaciones de todas las filas en el orden establecido. Este método garantiza que todos los movimientos largos ocurren en dirección horizontal (eje rápido), mientras que los movimientos verticales (eje lento) son mínimos e inevitables: solamente los necesarios para transitar entre filas.

Experimentalmente, esta heurística produce rutas con costo típicamente 30 a 40 por ciento menor que el vecino más cercano simple, y frecuentemente iguales o superiores a las rutas obtenidas mediante vecino más cercano seguido de 2-opt. Su ventaja adicional es la simplicidad conceptual y la predictibilidad del patrón de movimiento resultante.

\subsubsection{Resultados Comparativos}

Se evaluaron los tres enfoques de planificación en múltiples instancias representativas. Para una instancia típica de 30 estaciones distribuidas uniformemente en el espacio de trabajo, los resultados fueron:

Orden aleatorio (sin optimización): tiempo estimado de 285 segundos.

Vecino más cercano: tiempo estimado de 198 segundos, representando mejora de 30 por ciento respecto al orden aleatorio.

Vecino más cercano con refinamiento 2-opt: tiempo estimado de 175 segundos, mejora de 39 por ciento respecto al orden aleatorio.

Agrupamiento por filas: tiempo estimado de 165 segundos, mejora de 42 por ciento respecto al orden aleatorio.

La implementación final emplea el método de agrupamiento por filas dado su desempeño superior y menor complejidad computacional. El tiempo de cómputo para planificar la ruta de 50 estaciones es inferior a 50 milisegundos, despreciable comparado con el tiempo de ejecución de la ruta que típicamente supera los 400 segundos.

\subsubsection{Eficiencia Operativa Global}

El tiempo total de operación del sistema comprende tres fases: mapeo del entorno, planificación de la ruta, y ejecución de la cosecha. Para un sistema completo de 72 estaciones donde aproximadamente 40 contienen lechugas, los tiempos son:

Mapeo completo: 210 segundos (3.5 minutos).

Planificación de ruta para 40 estaciones: 0.15 segundos (despreciable).

Ejecución de cosecha: 40 estaciones multiplicadas por 8 segundos por estación resulta en 320 segundos.

Tiempo total: aproximadamente 530 segundos, equivalente a 8.8 minutos.

Esta eficiencia representa una mejora significativa comparada con operación manual, donde un operador humano requiere típicamente 25 a 30 minutos para cosechar la misma cantidad de lechugas. El sistema automatizado opera aproximadamente 3.4 veces más rápido que un operador humano, demostrando el valor de la optimización de trayectorias en la eficiencia global del sistema.