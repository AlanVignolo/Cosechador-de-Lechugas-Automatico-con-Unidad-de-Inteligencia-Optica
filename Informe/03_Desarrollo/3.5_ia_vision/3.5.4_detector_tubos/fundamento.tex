\subsubsection{Fundamento y motivación}

El sistema de mapeo autónomo requiere identificar la presencia de tubos de cultivo en dirección vertical para registrar la ubicación de estaciones. A diferencia del detector de cintas que explota contraste cromático mediante umbralización simple, la detección de tubos presenta mayor complejidad debido a que los tubos de PVC blanco presentan baja saturación similar al fondo del entorno.

La estrategia implementada combina detección de bordes mediante el algoritmo de Canny con filtrado por canal de saturación (S) en el espacio HSV. Esta aproximación explota dos propiedades: aunque el tubo no presente contraste cromático significativo, sus bordes superior e inferior generan gradientes de intensidad detectables, y el PVC blanco presenta saturación muy baja (color cercano al gris neutral) que lo diferencia de elementos con color como plantas verdes.

El objetivo específico del detector es identificar las posiciones verticales (coordenadas Y) de los bordes superior e inferior del tubo. Esta información es suficiente para confirmar la presencia de un tubo en el campo de visión y registrar su ubicación en configuracion\_tubos.json, permitiendo al sistema de mapeo construir el inventario de filas de cultivo.

