\subsubsection{Aplicación y resultados}

El detector de tubos se emplea durante la fase de mapeo autónomo para identificar las coordenadas verticales de las filas de estaciones. El robot ejecuta un barrido vertical completo capturando imágenes a intervalos regulares mientras se desplaza a velocidad reducida. Cuando el detector identifica un tubo, envia un flag indicando inicio y final de detección, para que luego el nivel regulatorio envie las posiciones Y correspondientes al sistema de mapeo. Con estas posiciones se saca el valor medio entre el inicio y fin de detección para registrar la coordenada Y de la fila. El error medio de localización es de 4.7mm, con un error máximo de 12.3mm, valores aceptables considerando que el ajuste final de posición se realiza mediante corrección visual cuando el brazo se aproxima a la planta específica.

El sistema implementa un mecanismos de confirmación temporal que requiere detección en al menos 3 de 5 frames consecutivos para rechazar detecciones espurias por reflejos o sombras.

El resultado final es un vector $\mathbf{F}_{filas} = [Y_1, Y_2, ..., Y_n]^T$ con las coordenadas Y de todas las filas detectadas, que se combina con las coordenadas X del detector de marcadores para construir la matriz completa de posiciones de estaciones.
