\subsubsection{Aplicación y resultados}

El detector de tubos se emplea durante la fase de mapeo autónomo para identificar las coordenadas verticales de las filas de estaciones. El robot ejecuta un barrido vertical completo capturando imágenes a intervalos regulares de 50 mm mientras se desplaza a velocidad reducida de 30 mm/s.

El sistema implementa dos mecanismos de filtrado: debouncing espacial de 120 mm para evitar detectar el mismo tubo múltiples veces, y confirmación temporal que requiere detección en al menos 3 de 5 frames consecutivos para rechazar detecciones espurias por reflejos o sombras.

El resultado final es un vector $\mathbf{F}_{filas} = [Y_1, Y_2, ..., Y_n]^T$ con las coordenadas Y de todas las filas detectadas (típicamente 5 a 7 filas), que se combina con las coordenadas X del detector de marcadores para construir la matriz completa de posiciones de estaciones.