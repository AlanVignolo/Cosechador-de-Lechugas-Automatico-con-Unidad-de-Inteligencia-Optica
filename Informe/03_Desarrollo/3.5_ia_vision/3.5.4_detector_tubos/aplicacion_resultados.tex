\subsubsection{Aplicación y resultados}

El detector de tubos se emplea durante la fase de mapeo autónomo para identificar las coordenadas verticales de las filas de estaciones. El robot ejecuta un barrido vertical completo capturando imágenes a intervalos regulares mientras se desplaza a velocidad reducida. Cuando el detector identifica un tubo, envia un flag indicando inicio y final de detección, para que luego el nivel regulatorio envie las posiciones Y correspondientes al sistema de mapeo. Con estas posiciones se saca el valor medio entre el inicio y fin de detección para registrar la coordenada Y de la fila.

El sistema implementa un mecanismos de confirmación temporal que requiere detección en al menos 3 de 5 frames consecutivos para rechazar detecciones espurias por reflejos o sombras.

El resultado final es un vector $\mathbf{F}_{filas} = [Y_1, Y_2, ..., Y_n]^T$ con las coordenadas Y de todas las filas detectadas, que se combina con las coordenadas X del detector de marcadores para construir la matriz completa de posiciones de estaciones.
\subsubsection{Métricas de desempeño}
Evaluado sobre 28 frames de barrido vertical (14 con tubos, 14 sin tubos):

\begin{table}[H]
\centering
\begin{tabular}{|l|r|}
\hline
\textbf{Métrica} & \textbf{Valor} \\ \hline
Sensibilidad (Recall) & 92.9\% (13/14) \\ \hline
Especificidad & 100\% (14/14) \\ \hline
Precisión (Precision) & 100\% (13/13) \\ \hline
Error de localización Y medio & 4.7 mm \\ \hline
Error máximo de localización Y & 12.3 mm \\ \hline
\end{tabular}
\caption{Desempeño del detector de tubos}
\label{tab:metricas_tubos}
\end{table}

Matriz de detección para tubos:

\begin{table}[H]
\centering
\begin{tabular}{cc|c|c|}
\cline{3-4}
& & \multicolumn{2}{c|}{\textbf{Predicción}} \\ \cline{3-4}
& & Tubo detectado & No tubo \\ \hline
\multicolumn{1}{|c|}{\multirow{2}{*}{\textbf{Real}}} & Tubo & 13 & 1 \\ \cline{2-4}
\multicolumn{1}{|c|}{} & No tubo & 0 & 14 \\ \hline
\end{tabular}
\caption{Matriz de detección binaria - tubos}
\label{tab:confusion_tubos}
\end{table}

\noindent
El único falso negativo se debió a reflejo especular que saturó el canal V del sensor.
