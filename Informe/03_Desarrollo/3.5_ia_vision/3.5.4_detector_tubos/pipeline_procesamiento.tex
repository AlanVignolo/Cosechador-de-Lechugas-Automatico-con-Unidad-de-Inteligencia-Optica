El algoritmo de detección de tubos implementa una secuencia de cuatro etapas que progresivamente extraen y refinan información estructural hasta identificar el tubo objetivo.\\

\underline{Etapa 1 - Detección de bordes mediante Canny}: Se aplica el algoritmo de Canny sobre la imagen convertida a escala de grises. Se aplica filtrado gaussiano ($5 \times 5$ píxeles), cálculo de gradientes mediante Sobel, supresión de no-máximos y doble umbralización con histéresis ($T_{bajo} = 20$, $T_{alto} = 172$). El resultado es una imagen binaria con los bordes del tubo y elementos de cultivo.\\

\underline{Etapa 2 - Filtrado por canal de saturación}: Para discriminar bordes del fondo, se explota que el PVC blanco presenta baja saturación. Se extrae el canal S del espacio HSV, se invierte ($S_{inv} = 255 - S$) y se umbraliza ($T_S = 150$). Los bordes de Canny se filtran mediante AND binario con esta máscara.\\

\underline{Etapa 3 - Refinamiento morfológico}: Para conectar segmentos de borde fragmentados por discontinuidades, se aplica una secuencia de operaciones morfológicas para refinar la detección.

\textbf{Aplicación de kernels de convolución}\\
\noindent
Las operaciones morfológicas se implementan mediante convolución con elementos estructurantes (kernels) específicos descriptos en la sección 2.2.1. Cada kernel se aplica mediante convolución según la ecuación 2.2, donde el elemento estructurante $B$ define la forma y alcance del vecindario considerado. La secuencia típica de aplicación es:

\begin{enumerate}
    \item Se realiza una operación morfológica de apertura (ec. \ref{eq:apertura_morf}) con \texttt{kernel (5×5, 1 iter.)} para eliminar ruido puntual mientras se preservan estructuras tubulares grandes.
    \item Se realiza una operación morfológica de cierre (ec. \ref{eq:cierre_morf}) con \texttt{kernel (15×5, 1 iter.)} con orientación horizontal preferencial para conectar segmentos de líneas horizontales del tubo.
    \item Se realiza una operación morfológica de cierre (ec. \ref{eq:cierre_morf}) con \texttt{kernel (20×7, 1 iter.)} para cerrar gaps internos en regiones detectadas.
    \item Se realiza una operación morfológica de dilatación (ec. \ref{eq:dilatacion_morf}) con \texttt{kernel (7×7, 2 iter.)} para engrosar los bordes detectados por Canny y facilitar la conexión de regiones.
\end{enumerate}

Los kernels rectangulares con mayor dimensión horizontal (15×5, 20×7) favorecen específicamente la detección de líneas horizontales características de los bordes superior e inferior del tubo PVC, mientras minimizan la conexión con estructuras verticales del entorno.


\underline{Etapa 4 - Detección de contornos y scoring}: Se ejecuta el algoritmo de Suzuki-Abe filtrando contornos por área mínima de 250 píxeles. Se calcula la relación de aspecto $AR = w/h$ y se descartan contornos con $AR \geq 0.9$ (elementos horizontales). Los contornos verticales se evalúan mediante scoring que combina: relación de aspecto (favorece $AR < 0.5$), área normalizada (favorece tamaños intermedios) y homogeneidad de saturación (favorece color uniforme). El contorno con mayor puntuación se selecciona como tubo objetivo.

La Figura \ref{fig:proceso_tubos} muestra las etapas del procesamiento del detector de tubos.

\begin{figure}[H]
\centering
\begin{subfigure}[b]{0.48\textwidth}
    \centering
    \includegraphics[width=0.7\textwidth]{imagenes/detector_tubos_1_original.png}
    \caption{Imagen RGB original}
\end{subfigure}
\hfill
\begin{subfigure}[b]{0.48\textwidth}
    \centering
    \includegraphics[width=0.7\textwidth]{imagenes/detector_tubos_2_canny.png}
    \caption{Bordes Canny (T=20-172)}
\end{subfigure}

\vspace{0.3cm}

\begin{subfigure}[b]{0.48\textwidth}
    \centering
    \includegraphics[width=0.7\textwidth]{imagenes/detector_tubos_3_canal_s.png}
    \caption{Máscara de baja saturación}
\end{subfigure}
\hfill
\begin{subfigure}[b]{0.48\textwidth}
    \centering
    \includegraphics[width=0.7\textwidth]{imagenes/detector_tubos_5_morfologia.png}
    \caption{Morfología de cierre (kernel 3×9)}
\end{subfigure}

\vspace{0.3cm}

\caption{\textit{Procesamiento del detector de tubos mostrando las transformaciones sucesivas hasta la identificación de bordes del tubo}}
\label{fig:proceso_tubos}
\end{figure}

El algoritmo implementa mecanismos de validación ante casos ambiguos. Sin contornos válidos, retorna valores nulos indicando ausencia de tubo. Con múltiples candidatos similares selecciona el de mayor score. 

\begin{figure}[H]
    \centering
    \includegraphics[width=0.4\textwidth]{imagenes/detector_tubos_6_lineas.png}
    \caption{\textit{Líneas superior e inferior detectadas}}
    \label{fig:detector_tubos_lineas}
\end{figure}

\subsubsection{Métricas de desempeño}
Evaluado sobre 28 frames de barrido vertical (14 con tubos, 14 sin tubos). Las Tablas \ref{tab:metricas_tubos} y \ref{tab:confusion_tubos} resumen el desempeño del detector:

\begin{table}[H]
\centering
\begin{tabular}{|l|r|}
\hline
\textbf{Métrica} & \textbf{Valor} \\ \hline
Sensibilidad (Recall) & 92.9\% (13/14) \\ \hline
Especificidad & 100\% (14/14) \\ \hline
Precisión (Precision) & 100\% (13/13) \\ \hline
Error de localización Y medio & 4.7 mm \\ \hline
Error máximo de localización Y & 12.3 mm \\ \hline
\end{tabular}
\caption{Desempeño del detector de tubos}
\label{tab:metricas_tubos}
\end{table}

Matriz de detección para tubos:

\begin{table}[H]
\centering
\begin{tabular}{cc|c|c|}
\cline{3-4}
& & \multicolumn{2}{c|}{\textbf{Predicción}} \\ \cline{3-4}
& & Tubo detectado & No tubo \\ \hline
\multicolumn{1}{|c|}{\multirow{2}{*}{\textbf{Real}}} & Tubo & 13 & 1 \\ \cline{2-4}
\multicolumn{1}{|c|}{} & No tubo & 0 & 14 \\ \hline
\end{tabular}
\caption{Matriz de detección binaria - tubos}
\label{tab:confusion_tubos}
\end{table}

\noindent
El único falso negativo se debió a reflejo especular que saturó el canal V del sensor.
