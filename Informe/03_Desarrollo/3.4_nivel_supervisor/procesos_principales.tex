El sistema implementa dos procesos de alto nivel que orquestan la operación del robot: detección y mapeo del workspace, y cosecha automatizada. Ambos workflows integran movimientos XY, control del brazo, visión artificial, y manejo de excepciones.

\textbf{Proceso 1: Detección y Mapeo del Workspace}

Este proceso ejecuta un escaneo sistemático para identificar y registrar las posiciones de todos los tubos de cultivo y plantas presentes.

\begin{table}[H]
\centering
\small
\begin{tabular}{|l|p{11cm}|}
\hline
\textbf{Fase} & \textbf{Descripción} \\
\hline
Inicialización & Verifica posición segura del brazo. Ejecuta homing de ejes XY estableciendo origen en esquina superior derecha del workspace. Separa ligeramente de límites (backoff 5-10mm). \\
\hline
Escaneo vertical & Movimiento descendente con velocidad 0.30 m/s capturando frames continuamente. Detector de ArUco identifica tubos y registra posiciones Y. Genera archivo de configuración de tubos. \\
\hline
Escaneo horizontal & Para cada tubo detectado: posiciona sistema en altura del tubo, ejecuta barrido horizontal a velocidad  reducida con detector de cintas. Registra posiciones X de plantas en matriz de cintas. \\
\hline
Retorno & Resincroniza posición consultando firmware. Ejecuta movimiento diagonal directo a origen (0,0). Resetea tracking global. \\
\hline
\end{tabular}
\caption{Fases del proceso de detección y mapeo}
\label{tab:proceso_deteccion}
\end{table}

El proceso genera dos salidas persistentes: archivo de configuración de tubos con posiciones Y, y matriz de cintas con posiciones X de cada planta.

\textbf{Proceso 2: Cosecha Automatizada}

Este proceso ejecuta la recolección sistemática de plantas maduras utilizando los mapas generados previamente.

\begin{table}[H]
\centering
\small
\begin{tabular}{|l|p{5.5cm}|p{5cm}|}
\hline
\textbf{Etapa} & \textbf{Condiciones de inicio} & \textbf{Acciones principales} \\
\hline
Inicialización & Mapas de detección disponibles. Sistema referenciado. & Verifica estado del brazo. Calcula posiciones operacionales (borde seguro, depósito). Inicializa flags internos. \\
\hline
Navegación a planta & Lista de cintas disponible. Brazo en posición segura. & Calcula deltas hacia posición objetivo. Ejecuta movimiento XY. Verifica llegada con tolerancia <5mm. \\
\hline
Clasificación & Posicionado sobre cinta. Cámara disponible. & Captura imagen. Ejecuta inferencia con red neuronal. Interpreta clase y confianza. \\
\hline
Recolección & Clasificación = LECHUGA (confianza >0.85). Gripper vacío. & Corrección fina de posición con visión. Transición brazo a posición recoger. Cierre de gripper. Elevación a posición transporte. \\
\hline
Deposición & Lechuga en gripper. Posicionado en zona depósito. & Transición brazo a posición depositar. Apertura de gripper. Retorno a posición transporte. Reset de flags. \\
\hline
Finalización & Todas las cintas procesadas. & Movimiento a borde seguro. Transición brazo a posición movimiento. Retorno a origen. Registro de estadísticas. \\
\hline
\end{tabular}
\caption{Etapas del proceso de cosecha automatizada}
\label{tab:proceso_cosecha}
\end{table}

\begin{figure}[H]
    \centering
    \includegraphics[width=0.6\textwidth]{imagenes/workflow_deteccion.png}
    \caption{Diagrama del proceso de detección y mapeo}
    \label{fig:workflow_deteccion}
\end{figure}

\begin{figure}[H]
    \centering
    \includegraphics[width=0.6\textwidth]{imagenes/workflow_cosecha.png}
    \caption{Diagrama del proceso de cosecha automatizada}
    \label{fig:workflow_cosecha}
\end{figure}
