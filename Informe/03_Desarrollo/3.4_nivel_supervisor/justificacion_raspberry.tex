\subsubsection{Selección de la Raspberry Pi 5}

El nivel supervisor requiere capacidad de procesamiento simultáneo para inferencia de redes neuronales, gestión de máquina de estados compleja, procesamiento de visión por computadora y comunicación con múltiples subsistemas. Los requerimientos principales incluyen: ejecución de modelos de clasificación de plantas sobre imágenes de 1920x1080 píxeles con latencia inferior a 200ms, procesamiento multi-hilo para gestión concurrente de comunicación UART, captura de cámara e inferencia de IA sin bloqueos, y algoritmos de visión por computadora (detección de marcadores ArUco, segmentación por color, análisis morfológico) sobre imágenes de alta resolución.

La Raspberry Pi 5 con 8GB de RAM fue seleccionada por satisfacer estos requerimientos. El procesador ARM Cortex-A76 quad-core a 2.4GHz (60\% más rápido que RPi 4) ejecuta inferencias de redes neuronales compactas (MobileNet, EfficientNet-Lite) a 5-8 fps. Los 8GB de memoria LPDDR4X-4267 permiten cargar simultáneamente modelos de IA, buffers de imágenes y estructuras de datos sin swapping. La GPU VideoCore VII provee aceleración para operaciones de visión. La conectividad incluye USB 3.0 para cámara (hasta 200 MB/s) y UART hardware para comunicación con el nivel regulatorio. El consumo es de 8-10W típico bajo carga, permitiendo operación continua con disipación pasiva.

La RPi 5 representa el balance óptimo entre capacidad computacional, flexibilidad de desarrollo y costo (\$80 USD). Alternativas como microcontroladores carecen de capacidad para procesamiento de IA, mientras que plataformas más potentes (Jetson Nano, Intel NUC) resultan sobredimensionadas y costosas. El soporte nativo de Linux con Python 3.x y bibliotecas optimizadas (OpenCV-ARM, TensorFlow Lite) facilita el desarrollo e integración del sistema.
