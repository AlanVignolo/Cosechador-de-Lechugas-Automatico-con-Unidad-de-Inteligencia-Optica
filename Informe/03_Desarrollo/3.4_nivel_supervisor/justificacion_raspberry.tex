La elección de la Raspberry Pi 5 como plataforma para el Nivel Supervisor se fundamenta en requerimientos críticos de procesamiento de visión artificial, gestión concurrente de estados y capacidad de ejecución de redes neuronales en tiempo real.

\subsubsection{Requerimientos del Sistema Supervisor}

El nivel supervisor debe ejecutar simultáneamente múltiples tareas computacionalmente intensivas:

\begin{itemize}
    \item \textbf{Inferencia de redes neuronales}: Ejecución de modelos de clasificación de plantas sobre imágenes de 1920x1080 píxeles con latencia <200ms por frame. Requiere capacidad de cómputo de punto flotante y memoria suficiente para cargar pesos del modelo (~50-100MB).

    \item \textbf{Gestión de máquina de estados compleja}: Orquestación de FSM con 7 estados principales, validación de transiciones, sincronización de subsistemas heterogéneos (movimiento, visión, manipulador) con latencias <50ms para garantizar respuesta en tiempo real.

    \item \textbf{Procesamiento multi-hilo}: Gestión concurrente de comunicación UART (thread dedicado para RX), captura de frames de cámara (thread de adquisición), procesamiento de IA (thread de inferencia) y ejecución de workflows sin bloqueos mutuos.

    \item \textbf{Visión por computadora}: Algoritmos de detección de marcadores ArUco, segmentación por color, detección de contornos y análisis morfológico ejecutándose en pipeline sobre imágenes RGB de alta resolución.
\end{itemize}

\subsubsection{Especificaciones de la Raspberry Pi 5}

La Raspberry Pi 5 (configuración de 8GB RAM) provee las capacidades necesarias:

\textbf{Procesador}: CPU ARM Cortex-A76 quad-core a 2.4GHz (60\% más rápido que RPi 4), con set de instrucciones ARMv8.2-A que incluye extensiones SIMD para aceleración de operaciones vectoriales. Suficiente para ejecutar inferencias de redes neuronales compactas (MobileNet, EfficientNet-Lite) a 5-8 fps.

\textbf{Memoria}: 8GB LPDDR4X-4267 permite cargar simultáneamente modelos de IA, buffers de imágenes y estructuras de datos del sistema sin swapping. El sistema operativo y aplicaciones utilizan ~2GB, dejando 6GB disponibles para procesamiento.

\textbf{GPU}: VideoCore VII con arquitectura mejorada provee aceleración para decodificación de video H.264/H.265 y operaciones OpenGL ES 3.1, útil para pre-procesamiento de imágenes.

\textbf{Conectividad}: USB 3.0 para cámara de alta velocidad (hasta 200 MB/s), UART hardware para comunicación con nivel regulatorio, GPIO para expansiones. Dual-band WiFi 802.11ac y Gigabit Ethernet (no utilizados actualmente pero disponibles para supervisión remota futura).

\textbf{Consumo}: 8-10W típico bajo carga, alimentación via USB-C 5V/3A. Permite operación continua sin refrigeración activa con disipador pasivo.

\subsubsection{Justificación de Selección}

La Raspberry Pi 5 es la plataforma mínima viable que cumple todos los requerimientos:

\begin{itemize}
    \item Capacidad de ejecutar inferencias de redes neuronales en tiempo real (<200ms/frame)
    \item Suficiente potencia para gestionar FSM compleja con múltiples threads concurrentes
    \item Soporte nativo de Python 3.x con bibliotecas optimizadas (OpenCV-ARM, TensorFlow Lite, NumPy)
    \item Sistema operativo Linux completo que facilita desarrollo, testing y debugging
    \item Costo accesible (\$80 USD) con prestaciones adecuadas para la aplicación
\end{itemize}

Alternativas como microcontroladores (Arduino, STM32) carecen de capacidad de procesamiento de IA. Plataformas más potentes (Jetson Nano, Intel NUC) resultan sobredimensionadas y costosas para los requerimientos del proyecto. La RPi 5 representa el balance óptimo entre capacidad computacional, flexibilidad de desarrollo y costo.
