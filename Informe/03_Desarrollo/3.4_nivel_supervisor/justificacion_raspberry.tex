El nivel supervisor requiere capacidad de procesamiento para algoritmos de visión por computadora y comunicación con el nivel regulatorio. Los requerimientos principales incluyen: ejecución de algoritmos de clasificación morfológica de plantas sobre imágenes de 1920x1080 píxeles, gestión de eventos mediante callbacks para comunicación UART y captura de cámara, y algoritmos de visión por computadora (detección de contornos, segmentación por color, análisis morfológico) sobre imágenes de alta resolución.

La Raspberry Pi 5 con 8GB de RAM fue seleccionada por satisfacer estos requerimientos y proporcionar una base sólida para futuras mejoras. El procesador ARM Cortex-A76 quad-core a 2.4GHz ejecuta algoritmos de procesamiento de imágenes con suficiente capacidad. Los 8GB de memoria LPDDR4X-4267 permiten cargar simultáneamente algoritmos de visión, buffers de imágenes y estructuras de datos sin intercambio a disco. La GPU VideoCore VII provee aceleración para operaciones de visión mediante OpenCV. La conectividad incluye USB 3.0 para cámara y UART hardware para comunicación con el nivel regulatorio. 

La RPi 5 representa el balance óptimo entre capacidad computacional, flexibilidad de desarrollo y costo. Alternativas como microcontroladores carecen de capacidad para procesamiento de IA, mientras que plataformas más potentes (Jetson Nano, Intel NUC) resultan sobredimensionadas y costosas. El soporte nativo de Linux con Python y bibliotecas optimizadas facilita el desarrollo e integración del sistema.
