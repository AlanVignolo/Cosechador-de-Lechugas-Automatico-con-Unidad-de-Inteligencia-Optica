El supervisor opera mediante una máquina de estados finitos que gobierna el comportamiento del robot durante todas las operaciones. Esta arquitectura permite transiciones deterministas entre estados, sincronización de subsistemas, y manejo robusto de excepciones con recuperación automática.

Además el sistema define ocho estados operacionales principles (Figura \ref{fig:maquina_estados_supervisor}):

\begin{itemize}
    \item \textbf{Inicialización}: Arranque del sistema. Ejecuta homing de ejes XY, calibra dimensiones del workspace, inicializa cámara y modelos de inteligencia artificial, y verifica comunicación con nivel regulatorio.

    \item \textbf{Exploración}: Escaneo sistemático del workspace mediante barrido vertical y horizontal. Genera mapas persistentes y retorna a origen.

    \item \textbf{Navegación}: Movimiento del sistema XY hacia objetivo identificado. Calcula trayectoria, envía comandos al nivel regulatorio, y verifica llegada a destino.

    \item \textbf{Clasificación}: Análisis de planta con visión artificial. Captura imagen, ejecuta algoritmo de inteligencia artificial, e interpreta resultado según umbrales de confianza configurados.
    
    \item \textbf{Corrección de posición}: Ajuste fino de posición XY basado en retroalimentación visual. Utiliza detección de marcadores para centrar la planta en el campo de visión del gripper.  

    \item \textbf{Movimiento del brazo}: Control del brazo robótico para recolección, movimiento o depósito. 

    \item \textbf{Espera}: Estado inactivo aguardando comandos. Monitorea conexión con nivel regulatorio mediante heartbeat periódico y procesa comandos recibidos por interfaz.

    \item \textbf{Emergencia}: Parada segura ante fallas detectadas. Detiene movimientos inmediatamente, intenta llevar brazo a posición segura, y preserva estado del sistema para diagnóstico.
\end{itemize}

\begin{figure}[H]
    \centering
    \includegraphics[width=0.85\textwidth]{imagenes/maquina_estados_supervisor.png}
    \caption{Máquina de estados del nivel supervisor}
    \label{fig:maquina_estados_supervisor}
\end{figure}

Las transiciones entre estados están reguladas por condiciones que garantizan seguridad y consistencia operacional. Las condiciones típicas incluyen verificación de sistema referenciado antes de movimientos, validación de posición dentro del workspace, disponibilidad de cámara, y compatibilidad de estado del brazo.