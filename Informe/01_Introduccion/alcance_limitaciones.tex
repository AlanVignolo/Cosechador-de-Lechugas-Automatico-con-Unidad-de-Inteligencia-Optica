El presente proyecto abarca el diseño, desarrollo e implementación de un prototipo de sistema robótico para cosecha automatizada de lechugas en cultivos hidropónicos verticales. El alcance del trabajo se define en los siguientes aspectos:

\underline{Sistema mecánico:} Diseñar e implementar un sistema de movimiento que permita alcanzar las lechugas en el espacio de trabajo, con capacidad de manipulación para su recolección. El diseño debe ser viable estructuralmente y cumplir con los requerimientos de precisión y estabilidad necesarios para las operaciones.

\underline{Sistema de percepción:} Desarrollar capacidades de detección y clasificación que permitan al sistema identificar la ubicación de tubos, cultivos y determinar su estado de madurez. La solución debe operar bajo condiciones controladas de iluminación propias de ambientes de cultivo hidropónico en interiores.

\underline{Sistema de control:} Implementar el control necesario para coordinar los movimientos del robot, gestionar la secuencia de operaciones, y ejecutar los procesos de detección y cosecha de forma autónoma.

\underline{Procesos operativos.:} Que el sistema cumpla con el objetivo de explorar el espacio de cultivo, identificar lechugas maduras, y ejecutar su recolección depositándolas en un contenedor designado. El proceso contempla hasta la deposición del cultivo cosechado.

\underline{Validación experimental:} Que el sistema sea capaz de cumplir con las pruebas planteadas para verificar su funcionamiento en condiciones reales de operación, caracterizando su desempeño mediante métricas apropiadas de precisión, exactitud y eficiencia.