% Objetivos del Proyecto

El presente proyecto tiene como propósito general el diseño, desarrollo e implementación de un sistema cosechador automático de lechugas para cultivos hidropónicos verticales, integrando tecnologías de visión artificial, control de movimiento y arquitecturas de control jerárquico. Para alcanzar este propósito, se establecen los siguientes objetivos específicos:

\subsection*{Objetivo General}

Desarrollar un prototipo funcional de robot cosechador automático con capacidad de detección por visión artificial y control coordinado de movimiento, validando la viabilidad técnica de la automatización en sistemas de cultivo hidropónico vertical.

\subsection*{Objetivos Específicos}

\subsubsection*{1. Sistema de Inteligencia Artificial y Visión por Computadora}

Implementar un sistema de visión artificial capaz de:

\begin{itemize}
    \item Detectar de manera precisa y confiable la ubicación espacial de lechugas en el entorno de cultivo hidropónico vertical
    \item Clasificar el estado de madurez de los cultivos para determinar su aptitud para cosecha
    \item Identificar elementos de referencia (tubos de cultivo, marcadores) para la navegación y mapeo del espacio de trabajo
\end{itemize}

\subsubsection*{2. Sistema de Control de Movimiento Coordinado}

Desarrollar una arquitectura de control que permita:

\begin{itemize}
    \item Gestionar coordinadamente el movimiento simultáneo de los ejes horizontal (X) y vertical (Z), configurando un sistema cartesiano de posicionamiento bidimensional
    \item Integrar el control de bajo nivel (nivel regulatorio) en microcontrolador Arduino para la actuación directa sobre motores y sensores
    \item Implementar el control de alto nivel (nivel supervisor) en Raspberry Pi para la toma de decisiones, gestión de secuencias operativas y coordinación con el sistema de visión artificial
    \item Garantizar sincronización temporal entre movimientos, trayectorias suaves y respuesta adecuada ante señales de seguridad (finales de carrera)
\end{itemize}

\subsubsection*{3. Validación Experimental del Prototipo}

Construir y validar un prototipo físico funcional que permita:

\begin{itemize}
    \item Verificar experimentalmente la efectividad de los algoritmos de detección y clasificación por visión artificial en condiciones reales de operación
    \item Evaluar el desempeño del sistema de control propuesto en términos de precisión de posicionamiento y velocidad de operación 
    \item Comprobar la integración funcional entre el diseño mecánico, el sistema de actuación y el software de control
    \item Identificar limitaciones, puntos críticos y oportunidades de mejora para futuras iteraciones del diseño
    \item Demostrar la viabilidad técnica y operativa del concepto de cosecha automatizada en sistemas hidropónicos verticales
\end{itemize}