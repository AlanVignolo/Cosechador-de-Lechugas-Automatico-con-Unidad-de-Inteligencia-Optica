% Contexto y Motivación

El crecimiento demográfico global y la urbanización acelerada han generado una creciente demanda de alimentos frescos en centros urbanos, mientras que la superficie cultivable disponible disminuye progresivamente. En este contexto, la hidroponía vertical emerge como una tecnología innovadora que permite la producción agrícola intensiva en espacios reducidos, optimizando el uso de recursos hídricos y eliminando la dependencia del suelo cultivable tradicional.

Los sistemas hidropónicos verticales organizan los cultivos en estructuras de múltiples niveles, maximizando la densidad de producción por metro cuadrado de superficie. Esta configuración resulta especialmente ventajosa para cultivos de hoja verde como lechugas, espinacas y aromáticas, que presentan ciclos de crecimiento cortos y requerimientos nutricionales moderados. La hidroponía vertical puede implementarse tanto en invernaderos especializados como en estructuras urbanas existentes, transformando espacios no productivos en sistemas de generación de alimentos.

Una característica distintiva de los invernaderos verticales es su capacidad para aprovechar superficies de difícil acceso para operadores humanos. Paredes de galpones industriales, fachadas de edificios, espacios interiores con altura considerable y estructuras verticales en general pueden convertirse en áreas de cultivo productivas. Sin embargo, esta ventaja espacial presenta simultáneamente un desafío operativo: la cosecha manual en sistemas verticales de gran altura requiere el uso de escaleras, plataformas elevadoras o andamios, lo cual incrementa los riesgos laborales, reduce la eficiencia operativa y aumenta los costos de mano de obra.

La dificultad de acceso manual a niveles superiores de cultivo limita la viabilidad económica de instalaciones hidropónicas de gran escala vertical. Los trabajadores deben realizar movimientos repetitivos en posiciones ergonómicamente desfavorables, con riesgo de caídas desde altura y fatiga acumulativa. Además, el tiempo requerido para posicionar equipos de acceso y desplazarse entre diferentes niveles de cultivo reduce significativamente la productividad del proceso de cosecha.

En este contexto, la automatización robótica representa una solución tecnológica que permite superar las limitaciones de accesibilidad inherentes a los sistemas verticales. Un sistema cosechador automático elimina la necesidad de que operadores humanos accedan físicamente a todos los niveles de cultivo, permitiendo aprovechar la altura completa de las estructuras disponibles sin comprometer la seguridad ni la eficiencia. La automatización además posibilita la operación continua o en horarios optimizados, mejorando la productividad global del sistema.

El proyecto se desarrolla como respuesta a la necesidad de mejorar la competitividad y escalabilidad de la agricultura urbana vertical, proponiendo una solución que combina eficiencia operativa, seguridad laboral y aprovechamiento óptimo del espacio disponible en entornos urbanos e industriales.
