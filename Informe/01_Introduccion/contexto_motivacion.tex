% Contexto y Motivación

El crecimiento demográfico y la urbanización han generado mayor demanda de alimentos frescos en centros urbanos, mientras la superficie cultivable disponible disminuye. La hidroponía vertical emerge como solución tecnológica que permite producción agrícola intensiva en espacios reducidos, optimizando recursos hídricos y eliminando la dependencia del suelo tradicional.

Los sistemas hidropónicos verticales organizan cultivos en estructuras de múltiples niveles, maximizando la densidad de producción por metro cuadrado. Esta configuración resulta especialmente ventajosa para cultivos de hoja verde como lechugas, con ciclos de crecimiento cortos y requerimientos moderados. La hidroponía vertical puede implementarse en invernaderos especializados o estructuras urbanas existentes, transformando espacios no productivos en áreas de cultivo.

Sin embargo, esta ventaja espacial presenta un desafío operativo crítico: la cosecha manual en sistemas verticales de gran altura requiere escaleras, plataformas elevadoras o andamios, incrementando riesgos laborales, reduciendo la eficiencia operativa y aumentando costos. Los trabajadores realizan movimientos repetitivos en posiciones ergonómicamente desfavorables, con riesgo de caídas y fatiga acumulativa. El tiempo requerido para posicionar equipos y desplazarse entre niveles reduce significativamente la productividad.

La automatización robótica representa una solución que supera las limitaciones de accesibilidad de los sistemas verticales. Un cosechador automático elimina la necesidad de acceso físico humano a todos los niveles, aprovechando la altura completa sin comprometer seguridad ni eficiencia. La automatización posibilita operación continua u optimizada, mejorando la productividad global del sistema.

El proyecto surge como respuesta a la necesidad de mejorar la competitividad y escalabilidad de la agricultura urbana vertical, combinando eficiencia operativa, seguridad laboral y aprovechamiento óptimo del espacio en entornos urbanos e industriales.
